% !TeX root = ../ALPHA_doc.tex
\subsection{Model Calculation Flow}

Figure~\ref{fig:model_flow} shows the processing flow of the AgLU model.  
The model begins with the processing of the original data sources, 
including extraction of necessary information, aggregation of gridded datasets, 
and conversion of data formats.  
Next, the processed data are adjusted through parameter estimation via regression analysis, 
data calibration to ensure equilibrium in the base year, 
and coefficient calibration for model consistency.  

Using these prepared data, the model then performs future scenario calculations.  
For each year, equilibrium solutions are iteratively computed based on 
the base-year data, the results of the previous year, and the scenario assumptions.  
The aggregated annual results provide the final outputs for each scenario.  
Additionally, model results are converted into the submission templates 
of AgMIP (Agricultural Model Intercomparison and Improvement Project) 
and IAMC (Integrated Assessment Modeling Consortium) 
to facilitate data exchange and comparison.

\begin{figure}[H]
    \centering
    \includegraphics[width=\textwidth]{fig7_1}
    \caption{Processing flow of the AgLU model}
    \label{fig:model_flow}
\end{figure}

\bigskip

\subsection{Programs within the Model}

This section provides an overview of the major programs used in the AgLU model.  
Each component plays a specific role in data processing, parameter calibration, and scenario simulation.

\medskip

% Refined tcolorbox style for a more formal academic tone
\newtcolorbox{programbox}[2][]{
  enhanced,
  sharp corners,
  colback=Accent!3!white,          
  colframe=Accent!70!black,        
  boxrule=0.5pt,                   
  left=7pt, right=7pt, top=5pt, bottom=5pt,
  title=\texttt{#2},
  coltitle=Primary!85!black,       
  fonttitle=\ttfamily\bfseries\small,
  attach boxed title to top left={xshift=0pt,yshift*=-1mm},
  boxed title style={
    colback=Accent!12!white,       
    colframe=Accent!60!black,
    boxrule=0pt,
    sharp corners,
    top=1pt, bottom=1pt,
  },
  before skip=8pt, after skip=8pt,
  #1
}


\begin{programbox}{0\_csv\_to\_gdx.R}
Converts datasets in \texttt{csv} or \texttt{xlsx} format into \texttt{gdx} format.  
This script processes raw datasets such as FAOSTAT and MAC Curve sources.
\end{programbox}

\begin{programbox}{0\_grid\_data.R}
Aggregates gridded datasets by country and production unit, and exports the results in \texttt{gdx} format.  
It processes geospatial data from sources such as GAEZ, Livestock System, and LUH2.
\end{programbox}

\begin{programbox}{1\_data\_import.gms}
Integrates and reformats the \texttt{gdx}-formatted datasets into structures compatible with the model’s computation framework.  
It performs country code conversions, consolidates data into AgLU’s food and land-use classifications,  
and includes submodules that output historical data in AgMIP and IAMC template formats.
\end{programbox}

\begin{programbox}{1-5\_regression.gms}
Performs parameter regression analyses based on the original datasets to estimate elasticities and adjustment coefficients.
\end{programbox}

\begin{programbox}{2\_basetear.gms}
Adjusts baseline datasets to ensure equilibrium consistency in the base year.
\end{programbox}

\begin{programbox}{3\_equilibrium.gms}
Solves for annual equilibrium states for future years, using base-year data and previous results as inputs.
\end{programbox}

\begin{programbox}{4\_combine\_gdx.gms}
Aggregates annual simulation outputs and compiles them into scenario-level result files.
\end{programbox}

\begin{programbox}{5\_scenario\_combine.gms}
Combines outputs from all scenarios to produce the comprehensive simulation dataset.
\end{programbox}

\begin{programbox}{Visualization.R}
Generates graphical outputs and visualization figures from the simulation results.
\end{programbox}


\bigskip

\subsection{Execution Method}

The AgLU model can be executed by running the shell script 
\texttt{AgLU/shell/execution.sh} on either Cygwin or Ubuntu environments.  
Model configurations are defined in \texttt{AgLU/shell/model\_setting.sh}.  
As summarized in Table~\ref{tab:model_setting}, the configurable items include 
the selection of executable programs, scenario settings, and computational environment settings.

\begin{table}[H]
\centering
\caption{Model configuration parameters in \texttt{model\_setting.sh}}
\label{tab:model_setting}
\scriptsize
\begin{xltabular}{\textwidth}{@{} l X l @{}}
\toprule
\textbf{Setting item} & \textbf{Description} & \textbf{Default} \\
\midrule
\multicolumn{3}{@{}l}{\textbf{Model switches}} \\[2pt]
\midrule
\texttt{data\_download} & Executes a script to download data from submodules when the repository is first cloned.  
Once downloaded, this can be turned off. & on \\
\texttt{data\_import} & Executes \texttt{1\_data\_import.gms} and \texttt{1-5\_regression.gms}. & on \\
\texttt{baseyear\_run} & Executes \texttt{2\_baseyear.gms}. & on \\
\texttt{model\_run} & Executes \texttt{3\_equilibirium.gms}. & on \\
\texttt{combine\_gdx} & Executes \texttt{4\_combine\_gdx.gms}. & on \\
\texttt{scenario\_combine} & Executes \texttt{5\_scenario\_combine.gms}. & on \\
\texttt{Visualization} & Executes \texttt{Visualization.R}. & on \\
\texttt{data\_update} & Executes \texttt{0\_csv\_to\_gdx.R}.  
If no data update is needed, this may be turned off. & off \\
\texttt{grid\_data} & Executes \texttt{0\_grid\_data.R}.  
If no data update is needed, this may be turned off.  
If executed, the folder \texttt{L/ModelData/AgLU/grid} must be copied to \texttt{AgLU/data}. & off \\
\texttt{Scenario\_compare} & Enables comparison across scenarios in plots. & off \\
\texttt{Model\_compare} & Enables comparison across models in plots. & off \\
\texttt{historical\_check} & Compares simulated and historical data in plots. & off \\
\texttt{aggregate} & Performs aggregation of results into 17-region, 10-region, and 5-region groupings. & on \\
\texttt{AGMIP} & Outputs results in AGMIP and IAMC\_Template formats. & on \\
\midrule
\multicolumn{3}{@{}l}{\textbf{Scenario settings}} \\[2pt]
\midrule
\texttt{socioeconomic\_setting} & Specifies the socioeconomic scenario.  
A file with the scenario name should exist in \texttt{AgLU/setting/socioeconomic}. & SSP2\_BaU \\
\texttt{climate\_setting} & Specifies the climate scenario.  
A file with the scenario name should exist in \texttt{AgLU/setting/climate}. & NoCC \\
\texttt{runlist} & Allows individual scenario specification without using the socioeconomic–climate matrix.  
Scenario names are defined as combined strings (e.g., \texttt{SSP2\_BaU-NoCC}).  
If this item is specified, the two settings above are ignored. & -- \\
\texttt{start\_year} & Starting year of simulation. & 2015 \\
\texttt{target\_year} & Ending year of simulation. & 2100 \\
\midrule
\multicolumn{3}{@{}l}{\textbf{Execution environment}} \\[2pt]
\midrule
\texttt{run\_single} & Runs the model for one scenario at a time. & on \\
\texttt{run\_paralell} & Runs scenarios in parallel. & off \\
\texttt{iteration\_number} & Number of iterations per year.  
Larger values improve convergence but increase computation time. & 0 \\
\texttt{NCPU} & Maximum number of CPUs used for parallel computation. & 8 \\
\texttt{mem\_check} & Performs memory usage checks (Windows only). & off \\
\texttt{mem\_threshold} & Memory usage threshold (\%) at which parallel computation pauses when \texttt{mem\_check} is enabled. & 80 \\
\bottomrule
\end{xltabular}
\end{table}

\subsection{Execution Environment Requirements}

The model can be executed under either Cygwin or Ubuntu with the following dependencies installed:

\begin{itemize}[leftmargin=2.5em]
  \item \textbf{Cygwin} (required packages: \texttt{zip}, \texttt{unzip})
  \item \textbf{Ubuntu}
  \item \textbf{R 4.3.2} (required libraries: 
  \texttt{tidyverse}, \texttt{gdxrrw}, \texttt{openxlsx}, \texttt{readxl}, 
  \texttt{ggplot2}, \texttt{furrr}, \texttt{ggpmisc}, \texttt{patchwork}, 
  \texttt{sf}, \texttt{raster}, \texttt{ncdf4}, \texttt{exactextractr}, 
  \texttt{rmapshaper}, \texttt{grDevices})
  \item \textbf{GAMS 38}
  \item \textbf{GitHub}
\end{itemize}
