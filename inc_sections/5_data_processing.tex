% !TeX root = ../ALPHA_doc.tex
This section explains the calibration and processing of data used in the model.  
Exogenous variables and parameters were estimated using multiple datasets, adjusted to 
obtain equilibrium in the base year, and parameters were estimated by regression analysis.  
\subsection{Data Sources}
Table~\ref{tab:data_sources} lists the datasets used in the model.

\begin{xltabular}{\linewidth}{l X X}
  \caption{List of data sources}\label{tab:data_sources}\\
  \toprule
  \textbf{Data Source} & \textbf{Dataset Name} & \textbf{URL} \\
  \midrule
  \endfirsthead

  \caption[]{List of data sources (continued)}\\
  \toprule
  \textbf{Data Source} & \textbf{Dataset Name} & \textbf{URL} \\
  \midrule
  \endhead

  \midrule
  \multicolumn{3}{r}{\footnotesize Continued on next page} \\
  \endfoot

  \bottomrule
  \endlastfoot

  FAOSTAT & Food Balances   & \nolinkurl{https://www.fao.org/faostat/en/\#data/FBS} \\
  FAOSTAT & Production      & \nolinkurl{https://www.fao.org/faostat/en/\#data/QCL} \\
  FAOSTAT & Trade           & \nolinkurl{https://www.fao.org/faostat/en/\#data/TCL} \\
  FAOSTAT & Producer Prices & \nolinkurl{https://www.fao.org/faostat/en/\#data/PP} \\
  FAOSTAT & Land Use        & \nolinkurl{https://www.fao.org/faostat/en/\#data/RL} \\
  FAOSTAT & Emissions       & \nolinkurl{https://www.fao.org/faostat/en/\#data/GCE} \\
  AQUASTAT & Major hydrological basins of the world & \nolinkurl{https://data.apps.fao.org/catalog//iso/7707086d-af3c-41cc-8aa5-323d8609b2d1} \\
  FAO and IIASA & GAEZ v4   & \nolinkurl{https://gaez.fao.org/} \\
  IIASA & SSP 3.0          & \nolinkurl{https://data.ece.iiasa.ac.at/ssp/\#/about} \\
  GTAP  & GTAP 10          & \nolinkurl{https://jgea.org/ojs/index.php/jgea/article/view/77} \\
  IFPRI & MAPSPAM          & \nolinkurl{https://doi.org/10.7910/DVN/SWPENT} \\
  World Bank & World Bank Open Data & \nolinkurl{https://data.worldbank.org/} \\
  EDGAR & EDGAR v8.0       & \nolinkurl{https://edgar.jrc.ec.europa.eu/dataset_ghg80} \\
  Herrero Acosta et~al.~(2018) & Livestock production systems & \nolinkurl{https://doi.org/10.4225/08/5aa068b33fe06} \\
  Muhammad et~al.~(2011)   & International Evidence on Food Consumption Patterns & \nolinkurl{http://199.135.94.241/publications/pub-details/?pubid=47581} \\
  T.~P.~Robinson et~al.~(2018) & Global distribution of ruminant livestock production systems V5 & \nolinkurl{https://doi.org/10.7910/DVN/WPDSZE} \\
  Harmsen et~al.~(2019)    & CH\textsubscript{4} and N\textsubscript{2}O MAC curves & \nolinkurl{https://doi.org/10.1016/j.envsci.2019.05.013} \\
  Hurtt et~al.~(2019)      & Land Use Harmonization 2 & \nolinkurl{https://luh.umd.edu/index.shtml} \\
\end{xltabular}

\subsection{Calibration and processing of data}
\subsubsection{Adjustment of demand-side data}

First, the adjustment of demand-side data was conducted. 
In the model, if a country–commodity pair has missing data for the base year 2015, it is assumed that consumption does not occur thereafter. 
Therefore, for country–commodity pairs without 2015 data in the Food Balance Sheets (FBS), the average value of subsequent years was used as the 2015 consumption level.

\begin{equation}
QH_{(c,cty,2015)} = \frac{\sum_{yr} QH^{FAO}_{(c,cty,yr)}}{N_{yr}} 
\quad \text{for } QH_{(c,cty,2015)} = \text{NA}
\end{equation}

where \( QH^{FAO} \) is household consumption data from FAO, 
and \( N_{yr} \) is the number of years for which data exist.

In the FBS data, household consumption includes the amount of food wasted at the household level. 
To distinguish this, waste rates by seven commodities and seven regions were obtained from FAO (2011). 
Using these coefficients, household consumption and food waste were estimated as follows:

\begin{align}
QWASTE_{(c,cty)} &= QH^{FAO}_{(c,cty)} \times WASTERATE_{(c,cty)} \\
QH_{(c,cty)} &= QH^{FAO}_{(c,cty)} \times (1 - WASTERATE_{(c,cty)})
\end{align}

For model simplification, the commodities used as livestock feed, processing materials, and biofuel inputs were limited. 
Wheat, rice, maize, other cereals, roots and tubers, sugar crops, pulses, oil crops, and vegetables were used as livestock feed. 
Wheat, rice, maize, other cereals, and fruits were used as raw materials for alcoholic beverages. 
Sugar crops were used as inputs for sugars, oil crops for vegetable oils, and maize, sugar crops, and vegetable oils for biofuels. 
Feed demand, intermediate input demand, and biofuel demand for other commodities were set to zero.

\begin{align}
QF_{(c \notin c_{feed},cty)} &= 0 \\
QBIO_{(c \notin c_{bio},cty)} &= 0 \\
QI_{(c \notin c_{int},cty)} &= 0
\end{align}

As a result of these adjustments, the total domestic demand changed. 
If the sum of all demand components was less than the total demand from the data, 
the residual was added as “other demand”:

\begin{align}
QOTH_{(c,cty)} &= QDEM_{(c,cty)} - 
  \nonumber\\[-0.3em]
  &\quad
  \big(QH_{(c,cty)} + QF_{(c,cty)} + QI_{(c,cty)} + QSEED_{(c,cty)} + 
  \nonumber\\[-0.3em]
  &\quad
  QTOUR_{(c,cty)} + QLOSS_{(c,cty)} + QWASTE_{(c,cty)}\big)
\end{align}


\subsubsection{Adjustment of crop production data}
Next, the adjustment of crop production data was conducted. 
From the FBS data, the production quantities of primary crops for each country were obtained. 
These were disaggregated into production units and production technologies using the GAEZ data as follows:

\begin{equation}
QSWAT_{(acro,bas,cty,water)} = 
QSUP^{FAO}_{(acro,cty)} \times 
\frac{QS^{GAEZ}_{(acro,bas,cty,water)}}{\sum_{(bas,water)} QS^{GAEZ}_{(acro,bas,cty,water)}}
\end{equation}

where \( QSUP^{FAO} \) denotes production quantity data from FAO, 
and \( QS^{GAEZ} \) represents production data from GAEZ.

From the LUH2 dataset, the cultivated land area for each production unit was obtained. 
Using the FAOSTAT data on national land use, these cultivated lands were divided into harvested and fallow lands as follows:

\begin{align}
QLCROCLT_{(bas,cty)} &= 
QLCRO^{LUH2}_{(bas,cty)} \times 
\frac{QLCROCLT^{FAO}_{cty}}{QLCROCLT^{FAO}_{cty} + QLCROFAL^{FAO}_{cty}} \\
QLCROFAL_{(bas,cty)} &= 
QLCRO^{LUH2}_{(bas,cty)} \times 
\frac{QLCROFAL^{FAO}_{cty}}{QLCROCLT^{FAO}_{cty} + QLCROFAL^{FAO}_{cty}}
\end{align}

where \( QLCRO^{LUH2} \) denotes cultivated area data from LUH2, 
\( QLCROCLT^{FAO} \) is harvested land area data from FAO, 
and \( QLCROFAL^{FAO} \) is fallow land area data from FAO.

Furthermore, the harvested land was disaggregated by crop type and production technology 
using GAEZ data as follows:

\begin{equation}
QLWAT_{(acro,bas,cty,water)} = 
QLCROCLT_{(bas,cty)} \times 
\frac{QL^{GAEZ}_{(acro,bas,cty,water)}}{\sum_{(acro,water)} QL^{GAEZ}_{(acro,bas,cty,water)}}
\end{equation}

where \( QLWAT \) is the cropland area by crop type and production technology, 
and \( QL^{GAEZ} \) is the crop-specific land area from GAEZ.

From the MAPSPAM dataset, harvested and physical area data were obtained for each crop, 
production unit, and production technology. 
Using these data, the cropping intensity was estimated as follows:

\begin{equation}
CROPINTESNSITY_{(acro,bas,cty,water)} = 
\frac{QLHARVESTED^{MAPSPAM}_{(acro,bas,cty,water)}}{QLPHYSICAL^{MAPSPAM}_{(acro,bas,cty,water)}}
\end{equation}

where \( QLHARVESTED^{MAPSPAM} \) is the harvested area data from MAPSPAM, 
and \( QLPHYSICAL^{MAPSPAM} \) is the physical area data from MAPSPAM.

Crop yields were calculated using the obtained production quantities, harvested areas, 
and cropping intensities as follows:

\begin{equation}
YLD_{(acro,bas,cty,water)} = 
\frac{QSWAT_{(acro,bas,cty,water)}}{QLWAT_{(acro,bas,cty,water)} \times CROPINTSNSITY_{(acro,bas,cty,water)}}
\end{equation}


\subsubsection{Adjustment of livestock production data}

Next, the adjustment of livestock production data was conducted. 
From FAO data, the numbers of livestock animals and slaughtered (or milked) animals were obtained. 
Using these, the production rate was calculated. 
A minimum threshold of 0.3 was set for the production rate; if it fell below this value, 
it was assumed that a portion of livestock was not involved in production activities.

\begin{align}
ANIMALPRODRATE_{(aliv,cty)} &= 
\frac{ANIMAL^{FAO}_{(aliv,cty)}}{ANIMALSTOCK^{FAO}_{(aliv,cty)}} \\
ANIMALRES_{(aliv,cty)} &= 
ANIMALSTOCK^{FAO}_{(aliv,cty)} - 
\frac{ANIMAL^{FAO}_{(aliv,cty)}}{0.3} 
\quad (\text{if } ANIMALPRODRATE_{(aliv,cty)} < 0.3)
\end{align}

where \( ANIMAL^{FAO} \) is the number of slaughtered (or milked) animals from FAO, 
and \( ANIMALSTOCK^{FAO} \) is the total livestock population.

When non-productive livestock exist, the corresponding unused pasture area was estimated as:

\begin{equation}
QLPASRES_{(aliv,cty)} = 
QLPAS^{LUH2}_{(aliv,cty)} \times 
\frac{ANIMALRES_{(aliv,cty)}}{ANIMALSTOCK^{FAO}_{(aliv,cty)}}
\end{equation}

where \( QLPASRES \) denotes the unused pasture area, 
and \( QLPAS^{LUH2} \) is the pasture area from LUH2.

Country-level livestock production data from the FBS were disaggregated 
into production units, climate zones, and production systems 
using the dataset from Herrero Acosta et al. (2018). 
Similarly, LUH2 pasture area data were disaggregated by livestock type, 
climate zone, and production system.

\begin{align}
QSSYS_{(aliv,bas,cty,cz,sys)} &= 
QSUP^{FAO}_{(aliv,cty)} \times 
\frac{QS^{HERRERO}_{(aliv,bas,cty,cz,sys)}}
{\sum_{(bas,cz,sys)} QS^{HERRERO}_{(aliv,bas,cty,cz,sys)}} \\
QLSYS_{(aliv,bas,cty,cz,sys)} &= 
\big(QLPAS^{LUH2}_{(cty,bas)} - QLPASRES_{(aliv,cty)}\big) 
\times 
\frac{QGRASS^{HERRERO}_{(aliv,bas,cty,cz,sys)}}
{\sum_{(aliv,cz,sys)} QGRASS^{HERRERO}_{(aliv,bas,cty,cz,sys)}}
\end{align}

where \( QS^{HERRERO} \) denotes production data 
and \( QGRASS^{HERRERO} \) represents pasture use data 
from Herrero Acosta et al. (2018).

Next, the feed efficiency and concentrate feed share were estimated as follows:

\begin{align}
FEEDEFFICIENCY_{(aliv,bas,cty,cz,sys)} &= 
\frac{QF^{HERRERO}_{(aliv,bas,cty,cz,sys)} + QGRASS^{HERRERO}_{(aliv,bas,cty,cz,sys)}}
{QS^{HERRERO}_{(aliv,bas,cty,cz,sys)}} \\
FEEDSHARE_{(aliv,bas,cty,cz,sys)} &= 
\frac{QF^{HERRERO}_{(aliv,bas,cty,cz,sys)}}
{QF^{HERRERO}_{(aliv,bas,cty,cz,sys)} + QGRASS^{HERRERO}_{(aliv,bas,cty,cz,sys)}}
\end{align}

where \( QF^{HERRERO} \) is the amount of concentrate feed from Herrero Acosta et al. (2018).

Using these values and the FBS feed demand data, 
the quantities of pasture and concentrate feed use were estimated. 
The amount of concentrate feed input was adjusted with a correction coefficient 
to ensure consistency with the FBS data. 
Consequently, the feed efficiency and concentrate feed share were also adjusted. 
If the estimated pasture use exceeded the available supply, 
it was replaced with the available amount.

\begin{align}
QFEEDSYS^{temp}_{(aliv,cfeed,bas,cty,cz,sys)} &= 
\frac{QSSYS_{(aliv,bas,cty,cz,sys)}}{ANIMALPRODRATE_{(aliv,cty)}} 
\times FEEDEFFICIENCY_{(aliv,bas,cty,cz,sys)} 
\times FEEDSHARE_{(aliv,bas,cty,cz,sys)} 
\times \frac{QF^{FAO}_{(cfeed,cty)}}{\sum_{cfeed} QF^{FAO}_{(cfeed,cty)}} \\
qfcoef_{(cfeed,cty)} &= 
\frac{QF^{FAO}_{(cfeed,cty)}}{\sum_{(aliv,bas,cz,sys)} QFEEDSYS^{temp}_{(aliv,cfeed,bas,cty,cz,sys)}} \\
QFEEDSYS_{(aliv,cfeed,bas,cty,cz,sys)} &= 
QFEEDSYS^{temp}_{(aliv,cfeed,bas,cty,cz,sys)} \times qfcoef_{(cfeed,cty)} \\
QGRASSSYS_{(aliv,bas,cty,cz,sys)} &= 
\frac{QSSYS_{(aliv,bas,cty,cz,sys)}}{ANIMALPRODRATE_{(aliv,cty)}} 
\times FEEDEFFICIENCY_{(aliv,bas,cty,cz,sys)} 
\times (1 - FEEDSHARE_{(aliv,bas,cty,cz,sys)})
\end{align}

If the estimated pasture demand exceeded the available supply, 
it was replaced with the available amount:

\begin{equation}
QGRASSSYS_{(aliv,bas,cty,cz,sys)} = 
QLSYS_{(aliv,bas,cty,cz,sys)} \times GRASSYLD_{(bas,cty)}
\quad \text{if } 
QLSYS_{(aliv,bas,cty,cz,sys)} \times GRASSYLD_{(bas,cty)} 
< QGRASSSYS_{(aliv,bas,cty,cz,sys)}
\end{equation}

where:
\begin{itemize}
\item \( QFEEDSYS^{temp} \): estimated feed demand by production system,  
\item \( QF^{FAO} \): feed demand data from FAO,  
\item \( qfcoef \): adjustment coefficient for consistency with FAO feed demand,  
\item \( QFEEDSYS \): adjusted feed demand consistent with FAO data,  
\item \( QGRASSSYS \): pasture demand by production system.
\end{itemize}


\subsubsection{Adjustment of food trade}

Next, the adjustment of food trade data was conducted. 
In order to express trade through complementary conditions, 
the net import and net export quantities were derived as follows:

\begin{align}
QM_{(c,cty)} &= QM^{FAO}_{(c,cty)} - QE^{FAO}_{(c,cty)} 
\quad (\text{if } QM^{FAO}_{(c,cty)} > QE^{FAO}_{(c,cty)}) \\
QE_{(c,cty)} &= QE^{FAO}_{(c,cty)} - QM^{FAO}_{(c,cty)} 
\quad (\text{if } QE^{FAO}_{(c,cty)} > QM^{FAO}_{(c,cty)})
\end{align}

where \( QM^{FAO} \) denotes the import quantity data from FAO, 
and \( QE^{FAO} \) denotes the export quantity data from FAO.

In the FBS dataset, global trade is not balanced; 
that is, the total export quantity and total import quantity of the world do not match.  
Therefore, the global discrepancy between total exports and imports was treated as a residual.  
This residual can be interpreted as trade with countries or regions 
outside the model scope, as well as losses occurring during international food transport.

\begin{equation}
QRES_{c} = \sum_{cty} QM_{(c,cty)} - \sum_{cty} QE_{(c,cty)}
\end{equation}



\subsubsection{Estimation of land constraints}

Next, the estimation method for land areas unsuitable for agricultural use 
due to soil and topographical constraints is described.  
From the GAEZ dataset, area data for 57 agro-ecological zone (AEZ) classes, 
classified according to climate, soil, and topographical conditions, 
were obtained.  
Among these, land classes with strong constraints were selected 
and considered unsuitable for agricultural use.  
The area of such constrained land was calculated as follows:

\begin{equation}
QLLIM_{(bas,cty)} = 
QLTOT^{LUH2}_{(bas,cty)} \times 
\frac{\sum_{aezlim} QL^{GAEZ}_{(aezlim,bas,cty)}}
{\sum_{aez} QL^{GAEZ}_{(aez,bas,cty)}}
\end{equation}

where:
\begin{itemize}
\item \( QLTOT^{LUH2} \): total land area from LUH2,
\item \( QL^{GAEZ} \): area data by AEZ class from GAEZ,
\item \( aez \): the set of 57 AEZ classes,
\item \( aezlim \in aez \): AEZ classes with severe land-use constraints.
\end{itemize}


\subsubsection{Adjustment of price data}

From the FAO dataset, producer prices, import prices, and export prices 
for each country and commodity were obtained. 
Since the price data contained many outliers and missing values, 
appropriate preprocessing was conducted.  

First, outlier values were treated as missing when the producer price 
was greater than ten times or less than one-tenth of the country-level median.

\begin{equation}
PP_{(c,cty)} \leftarrow \text{NA} 
\quad \text{if } 
\big( PP_{(c,cty)} > 10 \times \text{median}_{cty}(PP_{(c,cty)}) 
\ \cup \ 
PP_{(c,cty)} < 0.1 \times \text{median}_{cty}(PP_{(c,cty)}) \big)
\end{equation}

Next, missing values were complemented.  
A world average price was calculated from the available country-level producer prices, 
weighted by production quantities.  
A country-specific correction coefficient was then derived 
based on the deviation of each country's prices from the world average.  
This coefficient was used to impute the missing prices.  
The same procedure was applied for import and export prices.

\begin{align}
PP_{ave,c} &= 
\frac{\sum_{cty} PP_{(c,cty)} \times QSUP_{(c,cty)}}{\sum_{cty} QSUP_{(c,cty)}} \\
PP_{coef,cty} &= 
\frac{\sum_{c} PP_{(c,cty)}}{PP_{ave,c}} \\
PP_{(c,cty)} &= 
PP_{ave,c} \times PP_{coef,cty}
\end{align}

where \( PP_{ave} \) denotes the global average producer price,  
and \( PP_{coef} \) denotes the country-specific adjustment coefficient.

\bigskip

Next, using the GTAP10 dataset, exogenous variables related to prices were estimated.  
The Producer Support Estimate (PSE) represents the monetary transfers 
from consumers and taxpayers to producers resulting from agricultural policies.  
In this study, the PSE is expressed as the proportionate impact of 
subsidies and taxes on producer prices.  
Similarly, the Consumer Support Estimate (CSE) represents 
the transfers from or to consumers, 
expressed as the proportionate impact of subsidies and taxes on consumer prices.

\begin{align}
PSE_{(i,cty)} &= 
\frac{
OSEP^{GTAP}_{(i,cty)} 
+ \sum_i ISEP^{GTAP}_{(i,ii,cty)} 
- \sum_{endw} FTRV^{GTAP}_{(i,endw,cty)} 
+ \sum_{endw} FBEP^{GTAP}_{(i,endw,cty)}
}{
VOA^{GTAP}_{(i,cty)}
} \\
CSE_{(i,cty)} &= 
\frac{VDPM^{GTAP}_{(i,cty)} - VDPA^{GTAP}_{(i,cty)}}{VDPM^{GTAP}_{(i,cty)}}
\end{align}

where:
\begin{itemize}
\item \( OSEP^{GTAP} \): subsidies/taxes on production,  
\item \( ISEP^{GTAP} \): subsidies/taxes on intermediate inputs,  
\item \( FTRV^{GTAP} \): taxes on factor inputs,  
\item \( FBEP^{GTAP} \): subsidies on factor inputs,  
\item \( VOA^{GTAP} \): production value,  
\item \( VDPM^{GTAP}, VDPA^{GTAP} \): household consumption at market and consumer prices, respectively.
\end{itemize}

\bigskip

To link producer (farm-gate) prices with consumer (market) prices, 
the market margin ratio \( MMJ \) was estimated.  
Intermediate inputs and factor inputs corresponding to market margins 
were identified from the GTAP input–output tables, 
and their total was divided by the production value to obtain the ratio.  
Similarly, trade-related costs \( MMM \) and \( MME \) were estimated from GTAP data,  
using the difference between import values at CIF prices and export values at FOB prices.

\begin{align}
MMJ_{(i,cty)} &= 
\frac{
\sum_{immj} VDFA^{GTAP}_{(i,immj,cty)} 
+ \sum_{immj} VIFA^{GTAP}_{(i,immj,cty)} 
+ \sum_{immj} EVFA^{GTAP}_{(i,immj,cty)}
}{
\Big(
\sum_{ii} VDF^{A,GTAP}_{(i,ii,cty)} 
+ \sum_{ii} VIF^{A,GTAP}_{(i,ii,cty)} 
+ \sum_{endw} EVF^{A,GTAP}_{(i,endw,cty)} 
- 
(\sum_{immj} VDF^{A,GTAP}_{(i,immj,cty)} 
+ \sum_{immj} VIF^{A,GTAP}_{(i,immj,cty)} 
+ \sum_{immj} EVF^{A,GTAP}_{(i,immj,cty)})
\Big)
} \\
MMM_{(i,ccty)} &= 
\frac{\sum_{cty} VIWS^{GTAP}_{(i,cty,ccty)} - \sum_{cty} VXWD^{GTAP}_{(i,cty,ccty)}}
{\sum_{cty} VXWD^{GTAP}_{(i,cty,ccty)}} \\
MME_{(i,cty)} &= 
\frac{\sum_{ccty} VIWS^{GTAP}_{(i,cty,ccty)} - \sum_{ccty} VXWD^{GTAP}_{(i,cty,ccty)}}
{\sum_{ccty} VXWD^{GTAP}_{(i,cty,ccty)}}
\end{align}

where:
\begin{itemize}
\item \( VDFA^{GTAP}, VIFA^{GTAP}, EVFA^{GTAP} \): intermediate and factor input values at consumer prices,  
\item \( VIWS^{GTAP}, VXWD^{GTAP} \): import (CIF) and export (FOB) values,  
\item \( immj \in (i \cup endw) \): goods and factors corresponding to market margins.
\end{itemize}

\bigskip

The import tariff (\( TM \)) and export tariff (\( TE \)) were then calculated.  
The import tariff rate was obtained as the ratio of import value at domestic prices 
to that at international prices, 
and the export tariff rate as the ratio of export value at international prices 
to that at domestic prices.

\begin{align}
TM_{(i,ccty)} &= 
\frac{\sum_{cty} VIMS^{GTAP}_{(i,cty,ccty)}}{\sum_{cty} VIWS^{GTAP}_{(i,cty,ccty)}} \\
TE_{(i,cty)} &= 
\frac{\sum_{ccty} VXWD^{GTAP}_{(i,cty,ccty)}}{\sum_{ccty} VXMD^{GTAP}_{(i,cty,ccty)}}
\end{align}

where:
\begin{itemize}
\item \( VIMS^{GTAP} \): import value at domestic (market) prices,  
\item \( VXMD^{GTAP} \): export value at domestic (market) prices.
\end{itemize}

\bigskip

Finally, coefficients for decomposing producer prices by input factor 
were estimated.  
Each agricultural commodity is produced using land, feed, raw materials, 
and other production factors.  
Using the GTAP input–output tables, the cost share of each factor 
in the producer price was estimated.  
By multiplying the producer price by these shares, 
the base-year prices of each factor were obtained.

\begin{align}
LND_{(i,cty)} &= 
\frac{
\sum_{ilnd} VDF^{A,GTAP}_{(i,ilnd,cty)} 
+ \sum_{ilnd} VIF^{A,GTAP}_{(i,ilnd,cty)} 
+ \sum_{ilnd} EVF^{A,GTAP}_{(i,ilnd,cty)}
}{
\Big(
\sum_{ii} VDF^{A,GTAP}_{(i,ii,cty)} 
+ \sum_{ii} VIF^{A,GTAP}_{(i,ii,cty)} 
+ \sum_{endw} EVF^{A,GTAP}_{(i,endw,cty)} 
-
(\sum_{immj} VDF^{A,GTAP}_{(i,immj,cty)} 
+ \sum_{immj} VIF^{A,GTAP}_{(i,immj,cty)} 
+ \sum_{immj} EVF^{A,GTAP}_{(i,immj,cty)})
\Big)
} \\[6pt]
FED_{(i,cty)} &= 
\frac{
\sum_{ifed} VDF^{A,GTAP}_{(i,ifed,cty)} 
+ \sum_{ifed} VIF^{A,GTAP}_{(i,ifed,cty)} 
+ \sum_{ifed} EVF^{A,GTAP}_{(i,ifed,cty)}
}{
\Big(
\sum_{ii} VDF^{A,GTAP}_{(i,ii,cty)} 
+ \sum_{ii} VIF^{A,GTAP}_{(i,ii,cty)} 
+ \sum_{endw} EVF^{A,GTAP}_{(i,endw,cty)} 
-
(\sum_{immj} VDF^{A,GTAP}_{(i,immj,cty)} 
+ \sum_{immj} VIF^{A,GTAP}_{(i,immj,cty)} 
+ \sum_{immj} EVF^{A,GTAP}_{(i,immj,cty)})
\Big)
} \\[6pt]
IMD_{(i,cty)} &= 
\frac{
\sum_{iimd} VDF^{A,GTAP}_{(i,iimd,cty)} 
+ \sum_{iimd} VIF^{A,GTAP}_{(i,iimd,cty)} 
+ \sum_{iimd} EVF^{A,GTAP}_{(i,iimd,cty)}
}{
\Big(
\sum_{ii} VDF^{A,GTAP}_{(i,ii,cty)} 
+ \sum_{ii} VIF^{A,GTAP}_{(i,ii,cty)} 
+ \sum_{endw} EVF^{A,GTAP}_{(i,endw,cty)} 
-
(\sum_{immj} VDF^{A,GTAP}_{(i,immj,cty)} 
+ \sum_{immj} VIF^{A,GTAP}_{(i,immj,cty)} 
+ \sum_{immj} EVF^{A,GTAP}_{(i,immj,cty)})
\Big)
} \\[6pt]
OFC_{(i,cty)} &= 
1 - (LND_{(i,cty)} + FED_{(i,cty)} + IMD_{(i,cty)})
\end{align}

where:
\begin{itemize}
\item \( LND \): share of land cost in the producer price,  
\item \( FED \): share of feed cost in the producer price,  
\item \( IMD \): share of raw material cost in the producer price,  
\item \( OFC \): share of other production factors in the producer price,  
\item \( ilnd, ifed, iimd \in (i \cup endw) \): goods and factors corresponding to land, feed, and raw materials,  
\item The denominator represents the total value of domestic and imported intermediate inputs 
      and factor inputs, excluding those corresponding to market margin sectors.
\end{itemize}

