% !TeX root = ../ALPHA_doc.tex
\subsection{Parameter estimation through regression analysis}

This section presents the results of regression analyses used to estimate the parameters of the model.  
In the following equations, variables with an overline represent the dependent and independent variables used in the regressions.

\bigskip

Since price elasticities in the model were assumed to vary with per-capita GDP, 
the following regression equation was estimated:

\begin{equation}
\overline{elhp}_{(c,cty)} = \alpha^{pe}_{c} + \beta^{pe}_{c} \cdot \log(\overline{GDPPC}_{cty})
\end{equation}

\bigskip

For livestock products, it was assumed that production systems change with economic development.  
Thus, the parameters determining the production share were estimated using logistic regression analysis.  
In this estimation, dummy variables were introduced to represent climatic zones.

\begin{equation}
\log \left(
\frac{
\overline{SYSSHARE}_{(aliv,bas,cty,cz,sys)}
}{
1 - \overline{SYSSHARE}_{(aliv,bas,cty,cz,sys)}
}
\right)
= 
\delta^{sys}_{(aliv,cz,sys)} + \rho^{sys}_{(aliv,sys)} \cdot \log(\overline{GDPPC}_{cty})
\end{equation}

\bigskip

It was further assumed that feed requirements and the share of concentrate feed 
also change with the level of economic development.  
Regression analyses were conducted using the following equations:

\begin{align}
\log \left( \overline{FEEDEFFICIENCY}_{(aliv,bas,cty,cz,sys)} \right)
&= 
\alpha^{fe}_{(aliv,cz,sys)} + \beta^{fe}_{(aliv,cz,sys)} \cdot \log(\overline{GDPPC}_{cty}) \\[6pt]
\log \left(
\frac{
\overline{FEEDSHARE}_{(aliv,bas,cty,cz,sys)}
}{
1 - \overline{FEEDSHARE}_{(aliv,bas,cty,cz,sys)}
}
\right)
&=
\delta^{fs}_{(aliv,cz,sys)} + \rho^{fs}_{(aliv,sys)} \cdot \log(\overline{GDPPC}_{cty})
\end{align}

\bigskip

Following Harmsen et al.\ (2019), the elasticity parameter 
\(\sigma\) related to the adoption of emission-reduction technologies 
was estimated using the following regression equation:

\begin{equation}
\log \left(
1 - 
\frac{
\overline{RR}_{(src,gas,cty)}
}{
\overline{MRP}_{(src,gas,cty)}
}
\right)
=
\sigma_{(src,gas,cty)} \cdot \log(\overline{PRCCAR}_{cty} + 1)
\end{equation}

\bigskip

\subsection{Results of parameter estimation}

This section presents the results of the regression analyses conducted to estimate the model parameters.  
Variables with an overline in the equations represent the data used in the regressions.

\subsubsection{Relationship between price elasticity and per-capita GDP}

Figure~\ref{fig:gdp_elasticity} illustrates the relationship between per-capita GDP and price elasticity.  
Table~\ref{tab:price_elasticity} shows the estimated parameters and the coefficient of determination ($R^2$).  

Since the model assumes that price elasticity varies with per-capita GDP, 
the following regression equation was estimated:

\begin{equation}
\overline{elhp}_{(c,cty)} = \alpha^{pe}_{c} + \beta^{pe}_{c} \cdot \log(\overline{GDPPC}_{cty})
\end{equation}

For all commodities, the estimated elasticities increased with higher levels of per-capita GDP.  
For most commodities, high coefficients of determination ($R^2 \approx 0.8$) were obtained.  
However, for commodities such as sugars, eggs, and spices, $R^2$ values were lower, around 0.5 or 0.3.  
The $p$-values for all regressions were below 0.001, indicating statistically significant relationships 
between the variables.  
Furthermore, the 95\% confidence intervals for $\beta^{pe}$ did not include zero, 
confirming the statistical significance of the estimated coefficients.

% \begin{figure}[H]
%     \centering
%     \includegraphics[width=0.75\textwidth]{figs/figure3_1_gdp_elasticity.png}
%     \caption{Relationship between per-capita GDP and price elasticity}
%     \label{fig:gdp_elasticity}
% \end{figure}

\bigskip

\begin{table}[H]
    \centering
    \caption{Estimated parameters of price elasticity from regression analysis 
    (values in parentheses denote 95\% confidence intervals)}
    \label{tab:price_elasticity}

    \csvreader[
        tabular={l r r r r r l},
        table head=\toprule
            Commodity & $\alpha_{pe}$ & $\beta_{pe}$ & \multicolumn{2}{c}{$\beta_{pe}$ (95\% CI)} & $R^2$ & $p$-value \\
            \cmidrule(lr){4-5}
            & & & lower & upper & & \\
        \midrule,
        late after line=\\,
        table foot=\bottomrule
    ]{data/reg_table1.csv}%
    {commodity=\commodity, alphaPe=\alpha, betaPe=\beta, betaLow=\betalow, betaHigh=\betahigh, R2=\Rtwo, pValue=\pvalue}%
    {%
        \commodity &
        \alpha &
        \beta &
        \betalow &
        \betahigh &
        \Rtwo &
        \pvalue
    }
\end{table}


\bigskip

\subsubsection{Relationship between livestock production system share and per-capita GDP}

Figure~\ref{fig:sys_share_gdp} illustrates the relationship between per-capita GDP and 
the proportion of pasture-based production by livestock type and climatic zone.  
Table~\ref{tab:sys_share_gdp} presents the estimated parameters.

The following regression model was estimated:

\begin{equation}
\log\left(
\frac{\overline{SYSSHARE}_{(aliv,bas,cty,cz,sys)}}{1 - \overline{SYSSHARE}_{(aliv,bas,cty,cz,sys)}}
\right)
=
\delta^{sys}_{(aliv,cz,sys)} + \rho^{sys}_{(aliv,sys)} \cdot \log(\overline{GDPPC}_{cty})
\end{equation}

As per-capita GDP increases, the share of pasture-based production decreases for all livestock products.  
Adjusted coefficients of determination were moderate (0.2–0.3), and all $p$-values were below 0.001,  
indicating statistically significant relationships.  
The 95\% confidence intervals for $\rho^{sys}$ did not include zero,  
confirming the significance of the coefficients.

% \begin{figure}[H]
%     \centering
%     \includegraphics[width=0.75\textwidth]{figure3_2_sys_share_gdp.png}
%     \caption{Relationship between per-capita GDP and share of pasture-based production}
%     \label{fig:sys_share_gdp}
% \end{figure}

\begin{table}[H]
\centering
\caption{Estimated parameters for pasture-based production share (95\% confidence intervals in parentheses)}
\label{tab:sys_share_gdp}
\csvreader[
    tabular={l l r r r r r l},
    table head=\toprule
        Commodity & $\rho_{sys}$ & \multicolumn{4}{c}{$\delta_{sys}$ by climate zone} & $R^2$ & $p$-value \\
        \cmidrule(lr){3-6}
        & & Arid & Temperate & Humid & HyperArid & & \\
    \midrule,
    late after line=\\,
    table foot=\bottomrule
]{data/reg_table2.csv}%
{commodity=\commodity,rhoSys=\rho,deltaArid=\arid,deltaTemp=\temp,deltaHumid=\humid,deltaHyperArid=\hyper,R2=\Rtwo,pValue=\pval}%
{\commodity & \rho & \arid & \temp & \humid & \hyper & \Rtwo & \pval}
\end{table}


\subsubsection{Relationship between feed efficiency and per-capita GDP}

Figure~\ref{fig:feed_eff_gdp} shows the relationship between per-capita GDP and feed efficiency 
by livestock type, climate zone, and production system.  
Table~\ref{tab:feed_eff_gdp} presents the estimated regression parameters.

The regression model used is expressed as:

\begin{equation}
\log(\overline{FEEDEFFICIENCY}_{(aliv,bas,cty,cz,sys)}) =
\alpha^{fe}_{(aliv,cz,sys)} + \beta^{fe}_{(aliv,sys)} \cdot \log(\overline{GDPPC}_{cty})
\end{equation}

Higher per-capita GDP is associated with lower feed requirements.  
For ruminants, the adjusted $R^2$ ranged from 0.66 to 0.97, indicating high explanatory power,  
while for poultry and other livestock it was relatively low (0.22 and 0.05, respectively).  
The coefficients $\beta^{fe}$ were smaller for ruminants, suggesting stronger efficiency improvements 
with economic development.  
All $p$-values were below 0.001, and the 95\% confidence intervals for $\alpha^{fe}$ 
did not include zero, confirming statistical significance.

% \begin{figure}[H]
%     \centering
%     \includegraphics[width=0.75\textwidth]{figure3_3_feed_efficiency.png}
%     \caption{Relationship between per-capita GDP and feed efficiency}
%     \label{fig:feed_eff_gdp}
% \end{figure}

\begin{table}[H]
\centering
\caption{Estimated parameters of feed efficiency (95\% confidence intervals in parentheses)}
\label{tab:feed_eff_gdp}
\csvreader[
    tabular={l l l r r r r r r l},
    table head=\toprule
        Commodity & System & $\beta_{fe}$ & \multicolumn{5}{c}{$\alpha_{fe}$ by climate zone} & $R^2$ & $p$-value \\
        \cmidrule(lr){4-8}
        & & & Arid & Temperate & Humid & HyperArid & Total & & \\
    \midrule,
    late after line=\\,
    table foot=\bottomrule
]{data/reg_table3.csv}%
{commodity=\commodity,system=\system,betaFe=\beta,alphaArid=\arid,alphaTemp=\temp,alphaHumid=\humid,alphaHyperArid=\hyper,alphaTotal=\total,R2=\Rtwo,pValue=\pval}%
{\commodity & \system & \beta & \arid & \temp & \humid & \hyper & \total & \Rtwo & \pval}
\end{table}

\subsubsection{Relationship between concentrate feed share and per-capita GDP}

Figure~\ref{fig:feedshare_gdp} shows the relationship between per-capita GDP and 
the share of concentrate feed by livestock type, climatic zone, and production system.  
Table~\ref{tab:feedshare_gdp} reports the estimated regression parameters.  
Poultry and other livestock are excluded from the analysis because they do not consume pasture.

The following regression model was estimated:

\begin{equation}
\log \left(
\frac{
\overline{FEEDSHARE}_{(aliv,bas,cty,cz,sys)}
}{
1 - \overline{FEEDSHARE}_{(aliv,bas,cty,cz,sys)}
}
\right)
=
\delta^{fs}_{(aliv,cz,sys)} + \rho^{fs}_{(aliv,sys)} \cdot \log(\overline{GDPPC}_{cty})
\end{equation}

The results indicate that higher per-capita GDP is associated with 
a higher share of concentrate feed and a lower share of pasture.  
Adjusted coefficients of determination range from 0.74 to 0.90, 
indicating a relatively high explanatory power.  
All $p$-values are below 0.001, suggesting statistically significant relationships 
between the variables.  
Furthermore, the 95\% confidence intervals for $\rho^{fs}$ do not include zero, 
which confirms the statistical significance of the estimated coefficients.

% \begin{figure}[H]
%     \centering
%     \includegraphics[width=0.75\textwidth]{figure3_4_feedshare_gdp.png}
%     \caption{Relationship between per-capita GDP and concentrate feed share}
%     \label{fig:feedshare_gdp}
% \end{figure}


\begin{table}[H]
\centering
\caption{Estimated parameters of concentrate feed share 
(95\% confidence intervals in parentheses)}
\label{tab:feedshare_gdp}

\csvreader[
    tabular={l l l r r r r r l},
    table head=\toprule
        Commodity & System & $\rho_{fs}$ & \multicolumn{4}{c}{$\delta_{fs}$ by climate zone} & $R^2$ & $p$-value \\
        \cmidrule(lr){4-7}
        & & & Arid & Temperate & Humid & HyperArid & & \\
    \midrule,
    late after line=\\,
    table foot=\bottomrule
]{data/reg_table4.csv}%
{commodity=\commodity,system=\system,rhoFs=\rho,deltaArid=\arid,deltaTemp=\temp,deltaHumid=\humid,deltaHyperArid=\hyper,R2=\Rtwo,pValue=\pval}%
{\commodity & \system & \rho & \arid & \temp & \humid & \hyper & \Rtwo & \pval}
\end{table}

