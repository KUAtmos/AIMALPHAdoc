% !TeX root = ../ALPHA_doc.tex
% \subsection{About this document}
This document describes the model structure and system of equations of AIM-ALPHA (the
Asia-pacific Integrated Model - Agricultural Land-use Partial equilibrium model for Harmonized Analysis), 
a partial equilibrium model for agriculture and land use. The document is organized as follows:
\begin{enumerate}[label=\arabic*)]
  \item Model Structure (\cref{sec:structure})
  \item System of Equations (\cref{sec:variables,sec:equations})
  \item Data Processing (\cref{sec:data})
  \item Parameter Estimation (\cref{sec:param})
  \item Model Calculation Flow and Execution Method (\cref{sec:calc})
\end{enumerate}
\medskip
This document aims to enhance transparency and reproducibility of the model’s analytical components, provide clear instructions for model execution and configuration, 
facilitate linkage with other models, and support further applications in global and national assessments of sustainable land-use transitions.

% \subsection{Background}
% Future projections of agriculture and land use play an essential role in understanding global sustainability challenges. 
% Increasing food demand, dietary shifts, and the growing use of bioenergy are intensifying pressures on land resources, 
% while agriculture and land-use activities remain major sources of greenhouse gas (GHG) emissions and biodiversity loss. 
% At the same time, these sectors are highly vulnerable to climate change through altered temperature and precipitation patterns 
% that affect crop yields and food availability. Consequently, designing sustainable pathways for agriculture and land management 
% requires an integrated understanding of the interactions among socioeconomic drivers, land-use dynamics, and environmental outcomes.

% Model-based approaches provide a powerful framework for exploring such interlinked systems. A variety of global and regional models 
% have been developed to evaluate the implications of alternative socioeconomic and climate scenarios for food security, mitigation, 
% and ecosystem conservation. However, many existing models operate at aggregated regional scales and provide limited representation 
% of country-specific agricultural systems, land-use competition, and trade linkages. This limits their applicability for detailed 
% policy analysis or assessment of subnational dynamics.

% To address these limitations, the AIM-ALPHA model has been developed 
% as a high-resolution partial equilibrium model covering the agriculture and land-use sectors at national and subnational river basin levels. 
% The model is designed to complement the broader AIM (Asia-Pacific Integrated Model) framework by offering a detailed, bottom-up 
% representation of agricultural markets, land-use allocation, and environmental interactions.

% This formulation document provides a comprehensive description of the model structure, mathematical formulation, and computational 
% framework of AIM-ALPHA. It aims to enhance transparency and reproducibility of the model’s analytical components, facilitate linkage 
% with other models in the AIM framework, and support further applications in global and national assessments of sustainable land-use transitions.
