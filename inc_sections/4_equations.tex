% !TeX root = ../ALPHA_doc.tex
This section presents the mathematical formulation of the model and its explanation.  
In the following, \textbf{endogenous variables} are denoted by uppercase letters,  
\textbf{exogenous variables} by uppercase letters with an overline, and  
\textbf{parameters} by lowercase or Greek letters.

\subsection{Demand Block}
This block describes the equations related to demand determination.  
In the demand block, food demand is determined by factors such as prices, population, and GDP.

\paragraph{Total Demand}
% The total domestic demand (\( QDEM_{c,cty} \)) is the sum of all types of demand, including household consumption, feed, processing, biofuel, seed, tourism, food loss, food waste, and other demand.
\begin{align}
  QDEM_{c,cty}
  &= QH_{c,cty} + QF_{c,cty} + QI_{c,cty} + QBIO_{c,cty} \nonumber\\
  &\quad + QSEED_{c,cty} + QTOUR_{c,cty} + QLOSS_{c,cty}
    + QWASTE_{c,cty} + QOTH_{c,cty}
  \tag{QDEMEQ}\label{eq:QDEMEQ}
\end{align}

\paragraph{Household Consumption Demand}
\begin{align}
  QH_{c,cty}
  &= \overline{QHBASE}_{c,cty}
     \cdot
     \left(\frac{\overline{GDPPC}_{cty}}{\overline{GDPPCBASE}_{cty}}\right)^{elhg_{c,cty}}
     \nonumber\\
  &\quad \cdot
     \prod_{cc}\!
     \left(
       \frac{\bigl(1-\overline{CSE}_{cc,cty}\bigr)\cdot PC_{cc,cty}}
            {\bigl(1-\overline{CSEBASE}_{cc,cty}\bigr)\cdot \overline{PCBASE}_{cc,cty}}
     \right)^{elhp_{c,cc,cty}}
     \cdot
     \frac{\overline{POP}_{cty}}{\overline{POPBASE}_{cty}}
  \tag{QHEQ}\label{eq:QHEQ}
\end{align}

\paragraph{Feed Demand}
\begin{align}
  QF_{cfeed,cty}
  &= \sum_{bas}\sum_{aliv} QFEED_{aliv,cfeed,cty}
  \tag{QFEEDEQ}\label{eq:QFEEDEQ}
\end{align}

\paragraph{Intermediate Input Demand for Processing}
\begin{align}
  QI_{cint,cty}
  &= \sum_{apro} QINT_{apro,cint,bas,cty}
  \tag{QINTEQ}\label{eq:QINTEQ}
\end{align}

\paragraph{Biofuel Demand}
\begin{align}
  QBIO_{c,cty}
  &= \overline{QBIOBASE}_{c,cty}
     \cdot
     \prod_{cc}\!
     \left(
       \frac{\bigl(1-\overline{CSE}_{cc,cty}\bigr)\cdot PC_{cc,cty}}
            {\bigl(1-\overline{CSEBASE}_{cc,cty}\bigr)\cdot \overline{PCBASE}_{cc,cty}}
     \right)^{elhp_{c,cc,cty}}
  \tag{QBIOEQ}\label{eq:QBIOEQ}
\end{align}

\paragraph{Seed Demand}
\begin{align}
  QSEED_{c,cty}
  &= \overline{QSEEDBASE}_{c,cty}
     \cdot
     \frac{\overline{POP}_{cty}}{\overline{POPBASE}_{cty}}
     \nonumber\\
  &\quad \cdot
     \prod_{cc}\!
     \left(
       \frac{\bigl(1-\overline{CSE}_{cc,cty}\bigr)\cdot PC_{cc,cty}}
            {\bigl(1-\overline{CSEBASE}_{cc,cty}\bigr)\cdot \overline{PCBASE}_{cc,cty}}
     \right)^{elhp_{c,cc,cty}}
  \tag{QSEEDEQ}\label{eq:QSEEDEQ}
\end{align}

\paragraph{Tourist Consumption Demand}
\begin{align}
  QTOUR_{c,cty}
  &= \overline{QTOURBASE}_{c,cty}
     \cdot
     \prod_{cc}\!
     \left(
       \frac{\bigl(1-\overline{CSE}_{cc,cty}\bigr)\cdot PC_{cc,cty}}
            {\bigl(1-\overline{CSEBASE}_{cc,cty}\bigr)\cdot \overline{PCBASE}_{cc,cty}}
     \right)^{elhp_{c,cc,cty}}
  \tag{QTOUREQ}\label{eq:QTOUREQ}
\end{align}

\paragraph{Loss and Waste}
\begin{align}
  QLOSS_{c,cty}
  &= QDEM_{c,cty} \cdot \overline{LOSSRATE}_{c,cty}
  \tag{QLOSSEQ}\label{eq:QLOSSEQ}\\[.3em]
  QWASTE_{c,cty}
  &= QH_{c,cty} \cdot \overline{WASTERATE}_{c,cty}
  \tag{QWASTEEQ}\label{eq:QWASTEEQ}
\end{align}

\paragraph{Other Demand}
\begin{align}
  QOTH_{c,cty}
  &= \overline{QOTHBASE}_{c,cty}
     \cdot
     \prod_{cc}\!
     \left(
       \frac{\bigl(1-\overline{CSE}_{cc,cty}\bigr)\cdot PC_{cc,cty}}
            {\bigl(1-\overline{CSEBASE}_{cc,cty}\bigr)\cdot \overline{PCBASE}_{cc,cty}}
     \right)^{elhp_{c,cc,cty}}
  \tag{QOTHEQ}\label{eq:QOTHEQ}
\end{align}


\subsection{Production Block}

The production block describes the equations determining the production quantities of each food commodity, the input of intermediate goods and production factors, and the producer prices.  
Primary crops are differentiated by production technologies (irrigated and rainfed), while livestock products are distinguished by climate zones and production systems (pasture-based or mixed).  
Yield improvements due to technological progress and climate impacts are considered for crops, and productivity changes in livestock are captured through feed requirements and feed share.  
For energy crops, production quantities by year and macro-region are given exogenously from AIM-Hub, and allocated to production units according to price and yield.

\subsubsection{Crop Production}

The livestock production block describes equations for determining livestock output, production systems, feed requirements, and associated prices.  
Livestock products are distinguished by climate zones and production systems (pasture-based or mixed).  
Feed use and production efficiency depend on feed composition, feed efficiency, and animal productivity.


\paragraph{Domestic Production}
\begin{align}
  QSUP_{a,cty}
  &= \sum_{bas} QS_{a,bas,cty}
  \tag{QSUPEQ}\label{eq:QSUPEQ}
\end{align}

\paragraph{Crop Production by Technology}
\begin{align}
  QSWAT_{acro,bas,cty,water}
  &= QS_{acro,bas,cty} \cdot watshare_{acro,bas,cty,water}
  \tag{CROWATEREQ}\label{eq:CROWATEREQ}
\end{align}

\paragraph{Producer Price of Crops}
\begin{align}
  PP_{acro,cty}
  &= \Bigl(
      PL_{acro,bas,cty}
      \cdot
      \sum_{water}
      \frac{
        watshare_{acro,bas,cty,water}
      }{
        YLD_{acro,bas,cty,water}
        \cdot
        \overline{CROPINTENSITY}_{acro,bas,cty,water}
      }
      \nonumber\\[-0.3em]
  &\quad
      +\; PO_{acro,bas,cty}
     \Bigl)
     \,/\, (1 + PPCOMP_{acro,bas,cty})
  \tag{PPCROEQ}\label{eq:PPCROEQ}
\end{align}

\paragraph{Land Input for Crop Production}
\begin{align}
  QL_{acro,bas,cty}
  &= QS_{acro,bas,cty}
     \cdot
     \sum_{water}
     \frac{
       watshare_{acro,bas,cty,water}
     }{
       YLD_{acro,bas,cty,water}
       \cdot
       \overline{CROPINTENSITY}_{acro,bas,cty,water}
     }
  \tag{QLCROINPUTEQ}\label{eq:QLCROINPUTEQ}
\end{align}

\paragraph{Crop Yield}
\begin{align}
  YLD_{acro,bas,cty,water}
  &= \overline{YLDBASE}_{acro,bas,cty,water}
     \cdot
     \overline{YLDGR}_{acro,bas,cty,water}
     \nonumber\\
  &\quad
     \cdot
     \overline{YLDCHAN}_{acro,bas,cty,water}
  \tag{YLDEQ}\label{eq:YLDEQ}
\end{align}

\paragraph{Other Production Factor Price}
\begin{align}
  PO_{acro,bas,cty}
  &= \overline{POBASE}_{acro,bas,cty}
     \cdot
     \left(
       \frac{QS_{acro,bas,cty}}
            {\overline{QSPRE}_{acro,bas,cty}}
     \right)^{elop_{acro,cty}}
  \tag{POCROEQ}\label{eq:POCROEQ}
\end{align}


\subsubsection{Livestock Production}

The livestock production block describes equations for determining livestock output, production systems, feed requirements, and associated prices.  
Livestock products are distinguished by climate zones and production systems (pasture-based or mixed).  
Feed use and production efficiency depend on feed composition, feed efficiency, and animal productivity.

\paragraph{Livestock Production by System}
\begin{align}
  QSSYS_{aliv,bas,cty,cz,sys}
  &= QS_{aliv,bas,cty}
     \cdot czshare_{aliv,bas,cty,cz}
     \cdot SYSSHARE_{aliv,bas,cty,cz,sys}
  \tag{LIVSYSEQ}\label{eq:LIVSYSEQ}
\end{align}

\paragraph{Livestock System Share Function}
\begin{align}
  SYSSHARE_{aliv,bas,cty,cz,sys}
  &= \frac{1}{
       1 + \delta^{sys}_{aliv,cz,sys}
       \cdot
       \overline{GDPPC}_{cty}^{\,\rho^{sys}_{aliv,sys}}
     }
     + res_{aliv,bas,cty,cz,sys}
  \tag{SYSSHAREEQ}\label{eq:SYSSHAREEQ}
\end{align}

\paragraph{Producer Price of Livestock}
\begin{align}
  PP_{aliv,cty}
  &= \Bigl(
       PL_{aliv,bas,cty}
       / ANIMALYLD_{aliv,bas,cty}
       \nonumber\\[-0.3em]
  &\quad
       + pfeedcoef_{aliv,cty}
         \cdot
         \sum_{cfeed}
         (1 - \overline{CSE}_{cfeed,cty})
         \cdot
         PC_{cfeed,cty}
         \nonumber\\[-0.3em]
  &\quad
         \cdot
         \sum_{cz,sys}
         \Bigl(
           czshare_{aliv,bas,cty,cz}
           \cdot
           SYSSHARE_{aliv,bas,cty,cz,sys}
           \cdot
           ICL_{aliv,cfeed,bas,cty,cz,sys}
         \Bigr)
         \nonumber\\[-0.3em]
  &\quad
       + PO_{aliv,bas,cty}
     \Bigr)
     / (1 + PPCOMP_{aliv,bas,cty})
  \tag{PPLIVEQ}\label{eq:PPLIVEQ}
\end{align}

\paragraph{Grass Input Calculation}
\begin{align}
  Qgrass_{aliv,bas,cty}
  &= 
     \frac{
       QS_{aliv,bas,cty}
     }{
       \overline{ANIMALPRODRATE}_{aliv,cty}
     }
     \nonumber\\[-0.3em]
  &\quad
     \cdot
     \sum_{cz,sys}
     \Bigl(
       czshare_{aliv,bas,cty,cz}
       \cdot
       SYSSHARE_{aliv,bas,cty,cz,sys}
       \nonumber\\[-0.3em]
  &\quad
       \cdot
       \overline{FEEDEFFICIENCY}_{aliv,bas,cty,cz,sys}
       \cdot
       (1 - \overline{FEEDSHARE}_{aliv,bas,cty,cz,sys})
     \Bigr)
  \tag{QGRASSINPUTEQ}\label{eq:QGRASSINPUTEQ}
\end{align}

\paragraph{Available Grassland Area}
\begin{align}
  Qgrass_{aliv,bas,cty}
  &= 
     \Bigl(
       QL_{aliv,bas,cty}
       - 
       \overline{QLRES}_{aliv,bas,cty}
     \Bigr)
     \nonumber\\[-0.3em]
  &\quad
     \cdot
     \overline{GRASSYLD}_{bas,cty}
     \cdot
     grassusagerate_{bas,cty}
  \tag{QLLIVINPUTEQ}\label{eq:QLLIVINPUTEQ}
\end{align}


\paragraph{Livestock Yield}
\begin{align}
  ANIMALYLD_{aliv,bas,cty}
  &= \frac{
       QS_{aliv,bas,cty}
     }{
       QL_{aliv,bas,cty}
       - \overline{QLRES}_{aliv,bas,cty}
     }
  \tag{ANIMALYLDEQ}\label{eq:ANIMALYLDEQ}
\end{align}

\paragraph{Feed Input Calculation}
\begin{align}
  QFEED_{aliv,cfeed,bas,cty}
  &= \frac{
       QS_{aliv,bas,cty}
     }{
       \overline{ANIMALPRODRATE}_{aliv,cty}
     }
     \nonumber\\[-0.3em]
  &\quad
     \cdot
     \sum_{cz,sys}
     \bigl(
       czshare_{aliv,bas,cty,cz}
       \cdot SYSSHARE_{aliv,bas,cty,cz,sys}
       \nonumber\\[-0.3em]
  &\quad
       \cdot ICL_{aliv,cfeed,bas,cty,cz,sys}
     \bigr)
  \tag{QFEEDINPUTEQ}\label{eq:QFEEDINPUTEQ}
\end{align}

\paragraph{Feed Input Coefficient}
\begin{multline}
  ICL_{aliv,cfeed,bas,cty,cz,sys} = \overline{FEEDEFFICIENCY}_{aliv,bas,cty,cz,sys}\\
     \cdot
     \overline{FEEDSHARE}_{aliv,bas,cty,cz,sys}
     \cdot
     \overline{GRAINSHARE}_{aliv,cfeed,bas,cty,cz,sys}
  \tag{ICLEQ}\label{eq:ICLEQ}
\end{multline}

\paragraph{Other Production Factor Price (Livestock)}
\begin{align}
  PO_{aliv,bas,cty}
  &= \overline{POBASE}_{aliv,bas,cty}
     \cdot
     \left(
       \frac{
         QS_{aliv,bas,cty}
       }{
         \overline{QSPRE}_{aliv,bas,cty}
       }
     \right)^{elop_{aliv,cty}}
  \tag{POLIVEQ}\label{eq:POLIVEQ}
\end{align}

\paragraph{Livestock Production Quantity}
\begin{align}
  QS_{aliv,bas,cty}
  &= \overline{ANIMALWEIGHT}_{aliv,cty}
     \cdot
     (
       ANIMAL_{aliv,bas,cty}
       - \overline{ANIMALRES}_{aliv,bas,cty}
     )
     \nonumber\\[-0.3em]
  &\quad
     \cdot
     \overline{ANIMALPRODRATE}_{aliv,cty}
  \tag{ANIMALEQ}\label{eq:ANIMALEQ}
\end{align}

\subsubsection{Processed Goods produciton}

This section describes the production equations for processed agricultural products.  
Processed products are produced using intermediate inputs and other production factors.

\paragraph{Producer Price of Processed Products}
\begin{align}
  PP_{apro,cty}
  &= pintcoef_{apro,cty}
     \cdot
     \sum_{cint}
     \Bigl(
       (1 - \overline{CSE}_{cint,cty})
       \cdot
       PC_{cint,cty}
       \cdot
       ica_{apro,cint,cty}
     \Bigr)
     \nonumber\\[-0.3em]
  &\quad
     + PO_{apro,cty}
  \tag{PPPROEQ}\label{eq:PPPROEQ}
\end{align}

\paragraph{Intermediate Input Demand for Processed Products}
\begin{align}
  QINT_{apro,cint,cty}
  &= QSUP_{apro,cty}
     \cdot
     ica_{apro,cint,cty}
  \tag{QINTINPUTEQ}\label{eq:QINTINPUTEQ}
\end{align}

\paragraph{Other Production Factor Price (Processed Products)}
\begin{align}
  PO_{apro,cty}
  &= \overline{POBASE}_{apro,cty}
     \cdot
     \left(
       \frac{
         QSUP_{apro,cty}
       }{
         \overline{QSUPRE}_{apro,cty}
       }
     \right)^{elop_{apro,cty}}
  \tag{POPROEQ}\label{eq:POPROEQ}
\end{align}

\subsubsection{Energy Crop Production}
Energy crop production is determined from regional totals provided exogenously by the AIM-Hub model.


\paragraph{Regional Energy Crop Supply Balance}
\begin{align}
  \overline{QENE}_{region}
  &= \sum_{bas,cty \in region}
     QS_{\mathrm{ecr},bas,cty}
  \tag{QSENEEQ}\label{eq:QSENEEQ}
\end{align}

\paragraph{Land Input for Energy Crops}
\begin{align}
  QL_{\mathrm{ecr},bas,cty}
  &= \frac{
       QS_{\mathrm{ecr},bas,cty}
     }{
       YLD_{\mathrm{ecr},bas,cty}
     }
  \tag{QLENEINPUTEQ}\label{eq:QLENEINPUTEQ}
\end{align}

\paragraph{Land Price for Energy Crops}
\begin{align}
  PL_{\mathrm{ecr},bas,cty}
  &= PENE_{region}
  \tag{PLENEEQ}\label{eq:PLENEEQ}
\end{align}



\subsection{Land Use Block}

The land-use block allocates available land among different uses through a Logit function.  
Prices for non-human-used land (forest, other natural land, and fallow land) are fixed at their baseline values.  
A constraint is imposed to ensure that total agricultural land does not exceed the physically available land area due to slope and soil limitations.  
Afforestation areas are given exogenously from AIM-PLUM results.

\paragraph{Vegetation Area}
\begin{align}
  QLVEG_{bas,cty}
  &= 
     \overline{QLTOT}_{bas,cty}
     - 
     \overline{QLVEGN}_{bas,cty}
     -
     \overline{QLAFR}_{bas,cty}
  \tag{QLVEGEQ}\label{eq:QLVEGEQ}
\end{align}

\paragraph{Pasture Area Allocation}
\begin{align}
  QLPAS_{bas,cty}
  &= 
     QLVEG_{bas,cty}
     \nonumber\\[-0.3em]
  &\quad
     \cdot
     \frac{
       \delta_{pas,bas,cty}
       \cdot
       PLPAS_{bas,cty}^{\,\rho_{pas,bas,cty}}
     }{
       \delta_{pas,bas,cty}
       \cdot
       PLPAS_{bas,cty}^{\,\rho_{pas,bas,cty}}
       +
       \delta_{pasn,bas,cty}
       \cdot
       PLPASN_{bas,cty}^{\,\rho_{pasn,bas,cty}}
     }
  \tag{QLPASEQ}\label{eq:QLPASEQ}
\end{align}

\paragraph{Non-pasture Grassland Area}
\begin{align}
  QLPASN_{bas,cty}
  &=
     QLVEG_{bas,cty}
     \nonumber\\[-0.3em]
  &\quad
     \cdot
     \frac{
       \delta_{pasn,bas,cty}
       \cdot
       PLPASN_{bas,cty}^{\,\rho_{pasn,bas,cty}}
     }{
       \delta_{pas,bas,cty}
       \cdot
       PLPAS_{bas,cty}^{\,\rho_{pas,bas,cty}}
       +
       \delta_{pasn,bas,cty}
       \cdot
       PLPASN_{bas,cty}^{\,\rho_{pasn,bas,cty}}
     }
  \tag{QLPASNEQ}\label{eq:QLPASNEQ}
\end{align}

\paragraph{Cropland Area}
\begin{multline}
  QLCRO_{bas,cty} = QLPASN_{bas,cty}\\
  \cdot
  \frac{
    \delta_{cro,bas,cty}\cdot PLCRO_{bas,cty}^{\,\rho_{cro,bas,cty}}
  }{
    \delta_{cro,bas,cty}\cdot PLCRO_{bas,cty}^{\,\rho_{cro,bas,cty}}
    + \delta_{frs,bas,cty}\cdot \overline{PLFRS}_{bas,cty}^{\,\rho_{frs,bas,cty}}
    + \delta_{gra,bas,cty}\cdot \overline{PLGRA}_{bas,cty}^{\,\rho_{gra,bas,cty}}
  }\\
  \tag{QLCROEQ}\label{eq:QLCROEQ}
\end{multline}

\paragraph{Forest Area}
\begin{multline}
  QLFRS_{bas,cty} = QLPASN_{bas,cty}\\
  \cdot
  \frac{
    \delta_{frs,bas,cty}\cdot PLFRS_{bas,cty}^{\,\rho_{frs,bas,cty}}
  }{
    \delta_{cro,bas,cty}\cdot PLCRO_{bas,cty}^{\,\rho_{cro,bas,cty}}
    + \delta_{frs,bas,cty}\cdot \overline{PLFRS}_{bas,cty}^{\,\rho_{frs,bas,cty}}
    + \delta_{gra,bas,cty}\cdot \overline{PLGRA}_{bas,cty}^{\,\rho_{gra,bas,cty}}
  }\\
  \tag{QLFRSEQ}\label{eq:QLFRSEQ}
\end{multline}

\paragraph{Other Natural Land}
\begin{multline}
  QLGRA_{bas,cty} = QLPASN_{bas,cty}\\
  \cdot
  \frac{
    \delta_{gra,bas,cty}\cdot PLGRA_{bas,cty}^{\,\rho_{gra,bas,cty}}
  }{
    \delta_{cro,bas,cty}\cdot PLCRO_{bas,cty}^{\,\rho_{cro,bas,cty}}
    + \delta_{frs,bas,cty}\cdot \overline{PLFRS}_{bas,cty}^{\,\rho_{frs,bas,cty}}
    + \delta_{gra,bas,cty}\cdot \overline{PLGRA}_{bas,cty}^{\,\rho_{gra,bas,cty}}
  }\\
  \tag{QLGRAEQ}\label{eq:QLGRAEQ}
\end{multline}

\paragraph{Harvested Cropland Area}
\begin{multline}
  QLCROCLT_{bas,cty} = QLCRO_{bas,cty}\\
  \cdot
  \frac{
    \delta_{croclt,bas,cty}\cdot PLCROCLT_{bas,cty}^{\,\rho_{croclt,bas,cty}}
  }{
    \delta_{croclt,bas,cty}\cdot PLCROCLT_{bas,cty}^{\,\rho_{croclt,bas,cty}}
    + \delta_{crofal,bas,cty}\cdot \overline{PLCROFAL}_{bas,cty}^{\,\rho_{crofal,bas,cty}}
  }
  \tag{QLCROCLTEQ}\label{eq:QLCROCLTEQ}
\end{multline}

\paragraph{Fallow Cropland Area}
\begin{multline}
  QLCROFAL_{bas,cty} = QLCRO_{bas,cty}\\
  \cdot
  \frac{
    \delta_{crofal,bas,cty}\cdot \overline{PLCROFAL}_{bas,cty}^{\,\rho_{crofal,bas,cty}}
  }{
    \delta_{croclt,bas,cty}\cdot PLCROCLT_{bas,cty}^{\,\rho_{croclt,bas,cty}}
    + \delta_{crofal,bas,cty}\cdot \overline{PLCROFAL}_{bas,cty}^{\,\rho_{crofal,bas,cty}}
  }
  \tag{QLCROFALEQ}\label{eq:QLCROFALEQ}
\end{multline}

\paragraph{Cropland Allocation}
\begin{align}
  QL_{acro,bas,cty}
  &= QLCROCLT_{bas,cty}
     \cdot
     \frac{
       \delta_{acro,bas,cty}\cdot PL_{acro,bas,cty}^{\,\rho_{acro,bas,cty}}
     }{
       \sum_{acro}\delta_{acro,bas,cty}\cdot PL_{acro,bas,cty}^{\,\rho_{acro,bas,cty}}
     }
  \tag{QLCRODISTEQ}\label{eq:QLCRODISTEQ}
\end{align}

\paragraph{Pasture Allocation}
\begin{align}
  QL_{aliv,bas,cty}
  &= QLPAS_{bas,cty}
     \cdot
     \frac{
       \delta_{aliv,bas,cty}\cdot PL_{aliv,bas,cty}^{\,\rho_{aliv,bas,cty}}
     }{
       \sum_{aliv}\delta_{aliv,bas,cty}\cdot PL_{aliv,bas,cty}^{\,\rho_{aliv,bas,cty}}
     }
  \tag{QLLIVDISTEQ}\label{eq:QLLIVDISTEQ}
\end{align}

\paragraph{Land Prices}
\begin{align}
  PLCROCLT_{bas,cty}\cdot QLCROCLT_{bas,cty}
  &= \sum_{acro} PL_{acro,bas,cty}\cdot QL_{acro,bas,cty}
  \tag{PLCROCLTEQ}\label{eq:PLCROCLTEQ}
  \shortintertext{\vspace{0.6\baselineskip}}
  PLCRO_{bas,cty}\cdot QLCRO_{bas,cty}
  &= \bigl(
       PLCROCLT_{bas,cty}\cdot QLCROCLT_{bas,cty}
       \nonumber\\[-0.3em]
  &\quad
       + \overline{PLCROFAL}_{bas,cty}\cdot QLCROFAL_{bas,cty}
     \bigr)
     \nonumber\\[-0.3em]
  &\quad / (1 + PLCOMP_{bas,cty})
  \tag{PLCROEQ}\label{eq:PLCROEQ}
  \shortintertext{\vspace{0.6\baselineskip}}
  PLPASN_{bas,cty}\cdot QLPASN_{bas,cty}
  &= PLCRO_{bas,cty}\cdot QLCRO_{bas,cty}
     \nonumber\\[-0.3em]
  &\quad + \overline{PLFRS}_{bas,cty}\cdot QLFRS_{bas,cty}
     \nonumber\\[-0.3em]
  &\quad + \overline{PLGRA}_{bas,cty}\cdot QLGRA_{bas,cty}
  \tag{PLPASNEQ}\label{eq:PLPASNEQ}
  \shortintertext{\vspace{0.6\baselineskip}}
  PLPAS_{bas,cty}\cdot QLPAS_{bas,cty}
  &= \frac{\sum_{aliv} PL_{aliv,bas,cty}\cdot QL_{aliv,bas,cty}}
          {1 + PLCOMP_{bas,cty}}
  \tag{PLPASEQ}\label{eq:PLPASEQ}
\end{align}

\paragraph{Agricultural Land Constraint}
\begin{align}
  \overline{QLLIM}_{cty}
  &\ge
  QLCRO_{bas,cty} + QLPAS_{bas,cty}
  \;\perp\;
  PLCOMP_{bas,cty} \ge 0
  \tag{QLLIMEQ}\label{eq:QLLIMEQ}
\end{align}


\subsection{Market Block}

This subsection describes the equations in the market block, where consistency between food demand and supply is ensured.  
Trade of food commodities is represented as a complementarity problem.  
When the import price is higher than the border price, the import quantity becomes zero.  
Once the import price decreases to equal the border price, imports begin.  
Similarly, when the border price is higher than the export price, exports become zero, and once the export price rises to match the border price, exports take place.

\paragraph{Domestic Trade Balance}
\begin{align}
  NT_{c,cty}
  &= QSUP_{c,cty} - QDEM_{c,cty} - \overline{QST}_{c,cty}
  \tag{NTEQ}\label{eq:NTEQ}\\[0.6em]
  NT_{c,cty}
  &= QE_{c,cty} - QM_{c,cty}
  \tag{QEQMEQ}\label{eq:QEQMEQ}
\end{align}

\paragraph{International Trade Balance}
\begin{align}
  \sum_{cty} NT_{c,cty} - \overline{QRES}_{c} = 0
  \tag{NT0EQ}\label{eq:NT0EQ}
\end{align}

\paragraph{Import and Export Conditions}
\begin{align}
  PM_{c,cty} &\ge PB_{c,cty}
  \quad \perp \quad
  QM_{c,cty} \ge 0
  \tag{PBPMEQ}\label{eq:PBPMEQ}\\[0.4em]
  PB_{c,cty} &\ge PE_{c,cty}
  \quad \perp \quad
  QE_{c,cty} \ge 0
  \tag{PBPEEQ}\label{eq:PBPEEQ}
\end{align}



\subsection{GHG Emission Block}

This subsection describes the equations in the GHG emission block.  
GHG emissions are calculated based on the production quantities of each food commodity.  
When carbon prices are considered, the amount of mitigation technology adoption and its cost are computed from the MAC (Marginal Abatement Cost) curve.  
Emissions from the land-use sector are calculated according to each land area.

\paragraph{GHG Emissions}
\begin{align}
  GHG_{a,src,gas,cty}
  &= QSUP_{a,cty}
     \cdot
     \overline{EF}_{a,src,gas,cty}
     \cdot
     (1 - RR_{a,src,gas,cty})
  \tag{GHGEQ}\label{eq:GHGEQ}
\end{align}

\paragraph{Land-Use Emissions}
\begin{align}
  GHGLU_{src,gas,bas,cty}
  &= \overline{EF\_LU}_{src,gas,cty}
     \cdot QLFRS_{bas,cty}
     + \overline{EF\_LU}_{src,gas,cty}
       \cdot QLGRA_{bas,cty}
     \nonumber\\[-0.3em]
  &\quad
     + \overline{EF\_LU}_{src,gas,cty}
       \cdot QLCRO_{bas,cty}
     + \overline{EF\_LU}_{src,gas,cty}
       \cdot QLPAS_{bas,cty}
  \tag{GHGLUEQ}\label{eq:GHGLUEQ}
\end{align}

\paragraph{Emission Reduction Rate}
\begin{align}
  RR_{a,src,gas,cty}
  &= \overline{MRP}_{src,gas,cty}
     \cdot
     \Bigl(
       1 -
       \bigl[
         1 / (1 + \overline{PRCCAR}_{cty})
       \bigr]^{\sigma_{src,gas,cty}}
     \Bigr)
  \tag{RREQ}\label{eq:RREQ}
\end{align}

\paragraph{Abatement Cost}
\begin{align}
  AC_{a,cty}
  &= \sum_{src,gas}
     \overline{EF}_{a,src,gas,cty}
     \cdot
     \Biggl[
       \frac{\sigma_{src,gas,cty}}
            {(\sigma_{src,gas,cty}-1)}
       \cdot
       \overline{MRP}_{src,gas,cty}
       \nonumber\\[-0.3em]
  &\quad
       \cdot
       \Bigl(
         1 -
         \bigl(
           1 -
           RR_{a,src,gas,cty} /
           \overline{MRP}_{src,gas,cty}
         \bigr)^{(1 - 1/\sigma_{src,gas,cty})}
       \Bigr)
       - RR_{a,src,gas,cty}
     \Biggr]
  \tag{ACEQ}\label{eq:ACEQ}
\end{align}

\paragraph{Carbon Cost on Remaining Emissions}
\begin{align}
  PCB_{a,cty}
  &= \sum_{src,gas}
     \overline{PRCCAR}_{cty}
     \cdot
     \overline{EF}_{a,src,gas,cty}
     \cdot
     (1 - RR_{a,src,gas,cty})
  \tag{PCBEQ}\label{eq:PCBEQ}
\end{align}


\subsection{Complementarity Conditions}

This subsection describes the complementarity conditions introduced to avoid numerical errors in the Logit function when production quantities in specific production units become zero.  
If the production of a unit becomes zero, its land-use area would also become zero, leading to instability.  
Therefore, a minimum positive value is imposed to ensure that production quantities do not fall to zero.

\paragraph{Minimum Production Share for Trade}
\begin{align}
  \frac{QSUP_{c,cty}}{QM_{c,cty}}
  &\ge 0.01
  \quad \perp \quad
  TMCOMP_{c,cty} \ge 0
  \tag{DEPQMEQ}\label{eq:DEPQMEQ}
\end{align}

\paragraph{Minimum Production Share for Supply}
\begin{align}
  \frac{QS_{a,bas,cty}}{QSUP_{a,cty}}
  &\ge 0.01
  \quad \perp \quad
  PPCOMP_{a,bas,cty} \ge 0
  \tag{QSBOUEQ}\label{eq:QSBOUEQ}
\end{align}
