% !TeX program = lualatex
\documentclass[11pt,a4paper,headings=small,parskip=half]{scrartcl}

%-------------------- Fonts & Micro-typography --------------------
\usepackage[T1]{fontenc}
\usepackage[scaled=.95]{helvet}
\usepackage{newtxtext,newtxmath}  % Times 系の本文+数式(定番)
\usepackage{microtype}

%-------------------- Geometry & Layout --------------------
\usepackage[a4paper,margin=25mm]{geometry}
\usepackage{setspace}
\onehalfspacing
\setlength{\emergencystretch}{2em}

%-------------------- Colors & Links --------------------
\usepackage[dvipsnames]{xcolor}
\definecolor{Primary}{HTML}{1F3A5F} % deep blue-gray
\definecolor{Accent}{HTML}{2F6B5C}  % green-teal
\definecolor{Link}{HTML}{1A73E8}

% ======================================================================
%-------------------- Document variables (metadata) --------------------
% !TeX root = ../AIMALPHA_documentation.tex
\newcommand{\doctitle}{AIM-ALPHA documentation}
\newcommand{\docsubtitle}{Agricultural Land-use Partial Equilibrium Model}
\newcommand{\docversion}{1.0}
\newcommand{\docauthor}{Ryo Totake, Hiroki Yoshida, Tomoko Hasegawa, Shinichiro Fujimori}
\newcommand{\docauthorshort}{R. Totake et al.}
\newcommand{\docorg}{Kyoto University}
\newcommand{\docdate}{\today}

\usepackage[hidelinks]{hyperref}
\hypersetup{
  colorlinks=true,linkcolor=Primary,citecolor=Accent,urlcolor=Link,
  pdftitle={\doctitle},
  pdfauthor={\docauthor}
}
% ======================================================================

%-------------------- Headings (KOMA) --------------------
\addtokomafont{sectioning}{\color{Primary}\bfseries}
\RedeclareSectionCommand[beforeskip=1.1\baselineskip,afterskip=.4\baselineskip]{section}
\RedeclareSectionCommand[beforeskip=.9\baselineskip,afterskip=.3\baselineskip]{subsection}
\RedeclareSectionCommand[beforeskip=.7\baselineskip,afterskip=.2\baselineskip]{subsubsection}

%-------------------- Headers & Footers --------------------
\usepackage{scrlayer-scrpage}
\clearpairofpagestyles
\automark{section} 
\ihead{\footnotesize \doctitle\ Ver.\docversion}
\ohead{\footnotesize \headmark}
\cfoot{\footnotesize\color{Accent}\thepage}
\setheadsepline{.4pt}
\pagestyle{scrheadings}

%-------------------- Math & Notation --------------------
\usepackage{amsmath,mathtools}
\usepackage{bm}
\numberwithin{equation}{section}
\DeclareMathOperator*{\argmin}{arg\,min}
\DeclareMathOperator*{\argmax}{arg\,max}
\DeclareMathOperator{\Var}{Var}
\DeclareMathOperator{\Cov}{Cov}
\newcommand{\dd}{\,\mathrm{d}}
\newcommand{\R}{\mathbb{R}}
\newcommand{\vect}[1]{\bm{#1}}
\newcommand{\mat}[1]{\mathbf{#1}}
\DeclarePairedDelimiter{\abs}{\lvert}{\rvert}
\DeclarePairedDelimiter{\norm}{\lVert}{\rVert}

%-------------------- Cross-references --------------------
\usepackage[nameinlink,capitalise]{cleveref}

%-------------------- Figures, Tables, Captions --------------------
\usepackage{graphicx}
\graphicspath{{figs/}}
\usepackage{booktabs}
\usepackage{siunitx}
\sisetup{detect-all,table-auto-round=true,separate-uncertainty=true}
\usepackage[labelfont=bf]{caption}
\usepackage{subcaption}
\captionsetup{font=small,labelfont=bf}
\usepackage{csvsimple}
\usepackage{enumitem}
\usepackage{xltabular}
\usepackage{array}
\usepackage{xurl}
\usepackage[section]{placeins} 

%-------------------- Theorems in subtle boxes --------------------
\usepackage[most]{tcolorbox}
\tcbset{enhanced,boxrule=.3pt,arc=2mm,colframe=Accent!60,colback=Accent!3}
\newtcbtheorem[number within=section]{theobox}{Theorem}{colback=Accent!2!white}{th}
\newtcbtheorem[number within=section,use counter from=theobox]{lemabox}{Lemma}{colback=Accent!2!white}{lem}
\newtcbtheorem[number within=section]{defbox}{Definition}{colback=Accent!2!white}{def}

%-------------------- Algorithms (pseudocode) --------------------
\usepackage{algorithm}
\usepackage{algpseudocode}
\algrenewcommand\algorithmicrequire{\textbf{Input:}}
\algrenewcommand\algorithmicensure{\textbf{Output:}}

%-------------------- Code listings (optional) --------------------
\usepackage{listings}
\lstset{basicstyle=\ttfamily\small,frame=single,rulecolor=\color{Accent!50},
  keywordstyle=\color{Primary}\bfseries,commentstyle=\itshape\color{Accent!80},
  showstringspaces=false}

%-------------------- Section --------------------
\RedeclareSectionCommand[beforeskip=1.0ex plus 1ex minus .2ex, afterskip=.8ex]{section}
\RedeclareSectionCommand[beforeskip=2.2ex plus 1ex minus .2ex, afterskip=.8ex]{subsection}
\usepackage{etoolbox}
\newcommand\SectionTopSpace{0.5ex} % ← ここでお好み調整(例: 5–8ex)
\makeatletter
\pretocmd{\section}{%
  \clearpage
  \ifdim\pagetotal=0pt\relax
    \vspace*{\SectionTopSpace}
  \fi
}{}{}
\makeatother

\usepackage{float}
% Section size and color
\setkomafont{section}{\Huge\bfseries\color{Primary}}
\setkomafont{subsection}{\Large\bfseries\color{Primary}}
\setkomafont{subsubsection}{\large\bfseries\color{Primary}}

% Underbar for Section Title
\newcommand\SecRuleWidth{1.0\linewidth}   % 線の長さ(.25〜.45あたり好みで)
\newcommand\SecRuleThickness{1.1pt}       % 太さ(0.8〜1.6pt)
\newcommand\SecRuleVSkip{.20ex}           % タイトルと線の間隔
\makeatletter
\let\orig@sectionlinesformat\sectionlinesformat
\renewcommand*\sectionlinesformat[4]{%
  % #1=level name, #2=indent box, #3=number, #4=title
  \ifnum\pdfstrcmp{#1}{section}=0
    \par\noindent
    {\usekomafont{#1}\raggedright #3\enskip #4\par
     \vspace{\SecRuleVSkip}%
     {\color{Accent}\rule{\SecRuleWidth}{\SecRuleThickness}}\par}%
  \else
    \orig@sectionlinesformat{#1}{#2}{#3}{#4}%
  \fi
}
\makeatother

% Update History
\newtcolorbox{changelogentry}[3]{% {date}{author}{version}
  enhanced, breakable,
  colback=white,            % 本体は白
  colframe=Accent!70,       % 枠色(やや濃い)
  boxrule=.9pt,             % 枠の太さ(重め)
  borderline west={2.4pt}{0pt}{Accent}, % 左サイドバー
  sharp corners,
  left=8pt, right=8pt, top=8pt, bottom=8pt,
  title={\textbf{#1} \quad|\quad #2 \quad v#3}, % 上帯のタイトル
  fonttitle=\bfseries\small,
  coltitle=Primary,
  colbacktitle=Accent!10,   % タイトル帯の背景(控えめ)
  boxed title style={
    sharp corners,
    colframe=Accent!70,
    boxrule=.9pt
  }
}

% Reference
\usepackage[backend=biber,style=numeric,sorting=nyt,maxbibnames=6]{biblatex}
\addbibresource{refs.bib}

% ----------------------------------------------------------------




% =====================
%        body
% =====================
%============================================================
\title{\doctitle\\\large \docsubtitle}
\author{\docauthor}
\date{\today}
%============================================================

\begin{document}

% ---- Title Page ----
\begin{titlepage}
\centering
\vspace*{3cm}

{\Huge\bfseries \doctitle\\[0.5em]
\textcolor{Accent}{Agricultural Land-use\\Partial Equilibrium Model}}
% {\Huge\bfseries \doctitle\par}
% \vspace{0.4em}
% {\huge\color{Accent}\docsubtitle\par} 

\vspace{2cm}
{\Large \docauthor\\[0.5em]
\docorg}

\vfill
{\large  Version \docversion{} -- \docdate}
\end{titlepage}
% ---- Imprint / Abstract page (front matter) ----
\clearpage
\thispagestyle{empty} % Without page number and header

{\Large\bfseries \doctitle\par}
{\normalsize \docsubtitle\par}
\vspace{1.2em}

\textbf{Abstract}\par
\noindent
This document describes the model structure and system of equations for
AIM-ALPHA. AIM-ALPHA is a one-year step
sequential dynamic partial equilibrium model targeting agricultural
markets and land use. It performs future projections considering 166
countries on the demand side, 400 production units on the supply side,
23 food commodities, and 5 land use categories. This model is
anticipated to be applicable to research including analyses of food
security issues, climate change impacts, evaluations of climate change
mitigation measures, and assessments of biodiversity impacts related to
land use.

\medskip
\textbf{Keywords:} Integrated assessment model, partial equilibrium, agricultural markets,
land use, climate change mitigation

\vfill

{\color{Accent}\rule{0.22\linewidth}{0.8pt}\par}\vspace{0.8ex}

{\small
\textbf{Authors:} \docauthor\par
\textbf{Affiliation:} \docorg\par
\textbf{Contact:} \href{mailto:totake.ryo.85s@st.kyoto-u.ac.jp}{totake.ryo.85s@st.kyoto-u.ac.jp}\par
\textbf{Version:} \docversion\par 
\textbf{Date:} \docdate\par
\textbf{License:} **\par %\href{https://creativecommons.org/licenses/by/4.0/}{CC BY 4.0}\par
\textcopyright~
\the\year\ \docauthor. All rights reserved.
}

\clearpage
% -------------- Table of Contents -----------------

\clearpage
\tableofcontents
\clearpage

% ------------------- Contents ---------------------
\section{Introduction}\label{sec:intro}
% !TeX root = ../ALPHA_doc.tex
% \subsection{About this document}
This document describes the model structure and system of equations of AIM-ALPHA (the
Asia-pacific Integrated Model - Agricultural Land-use Partial equilibrium model for Harmonized Analysis), 
a partial equilibrium model for agriculture and land use. The document is organized as follows:
\begin{enumerate}[label=\arabic*)]
  \item Model Structure (\cref{sec:structure})
  \item System of Equations (\cref{sec:variables,sec:equations})
  \item Data Processing (\cref{sec:data})
  \item Parameter Estimation (\cref{sec:param})
  \item Model Calculation Flow and Execution Method (\cref{sec:calc})
\end{enumerate}
\medskip
This document aims to enhance transparency and reproducibility of the model’s analytical components, provide clear instructions for model execution and configuration, 
facilitate linkage with other models, and support further applications in global and national assessments of sustainable land-use transitions.

% \subsection{Background}
% Future projections of agriculture and land use play an essential role in understanding global sustainability challenges. 
% Increasing food demand, dietary shifts, and the growing use of bioenergy are intensifying pressures on land resources, 
% while agriculture and land-use activities remain major sources of greenhouse gas (GHG) emissions and biodiversity loss. 
% At the same time, these sectors are highly vulnerable to climate change through altered temperature and precipitation patterns 
% that affect crop yields and food availability. Consequently, designing sustainable pathways for agriculture and land management 
% requires an integrated understanding of the interactions among socioeconomic drivers, land-use dynamics, and environmental outcomes.

% Model-based approaches provide a powerful framework for exploring such interlinked systems. A variety of global and regional models 
% have been developed to evaluate the implications of alternative socioeconomic and climate scenarios for food security, mitigation, 
% and ecosystem conservation. However, many existing models operate at aggregated regional scales and provide limited representation 
% of country-specific agricultural systems, land-use competition, and trade linkages. This limits their applicability for detailed 
% policy analysis or assessment of subnational dynamics.

% To address these limitations, the AIM-ALPHA model has been developed 
% as a high-resolution partial equilibrium model covering the agriculture and land-use sectors at national and subnational river basin levels. 
% The model is designed to complement the broader AIM (Asia-Pacific Integrated Model) framework by offering a detailed, bottom-up 
% representation of agricultural markets, land-use allocation, and environmental interactions.

% This formulation document provides a comprehensive description of the model structure, mathematical formulation, and computational 
% framework of AIM-ALPHA. It aims to enhance transparency and reproducibility of the model’s analytical components, facilitate linkage 
% with other models in the AIM framework, and support further applications in global and national assessments of sustainable land-use transitions.


\section{Model Structure}\label{sec:structure}
% !TeX root = ../ALPHA_doc.tex
This section provides an overview of the AIM-ALPHA model. AIM-ALPHA is a recursive-dynamic partial equilibrium model 
focusing on agricultural markets and land use, operating with an annual time step for the time horizon from 2015 to 2100.
The model is formulated as a mixed complementarity problem (MCP) and handles approximately 200,000 variables. 
The basic model structure is shown in Fig.~\ref{fig:model_structure}.

For the spacial resolution, the demand side covers 166 countries and regions globally, while the production side consists of 400 production units. 
These production units are defined by overlaying the 166 demand regions~\cite{worldbankWorldBankOfficial2025} with FAO’s 230 major river basin boundaries~\cite{faolandandwaterdivisionMajorHydrologicalBasins2011}. 
Figure~\ref{fig:regions} illustrates the geographical coverage of the demand regions and production units.

Food production is categorized into 13 primary crops, 6 livestock products, and 4 processed goods. Table~\ref{tab:commodities} lists the food commodity classifications used in the model. 
The production functions are represented by Leontief-type functions, where each good is produced by combining cropland, feed and pasture, 
and raw materials as inputs. Other factors such as labor, capital, and energy are aggregated into a single production factor. 

Primary crops and livestock products that require land are differentiated by production unit, whereas processed goods, which do not require 
land inputs, are modeled at the level of the 166 demand regions. Land use within each production unit is allocated according to a multi-nested 
logit function that balances total land area, and land shares are determined by relative land prices.

Food demand is driven by population and GDP, with income elasticity considered for future changes. Food prices are determined endogenously 
so that supply and demand are balanced in the domestic market. Within each country, the balance between production, domestic demand, and net trade 
(imports minus exports) is maintained. In the global market, total world imports and exports are also balanced. 


\begin{figure}[tbp]
  \centering
  \includegraphics[width=\linewidth,height=0.78\textheight,keepaspectratio]{fig2_1}
  \caption{Overview of AIM-ALPHA}
  \label{fig:model_structure}
\end{figure}

\begin{figure}[tbp]
  \centering
  \includegraphics[width=\linewidth,height=0.78\textheight,keepaspectratio]{fig2_2}
  \caption{Geographical coverage of countries (166) and production units (400).}
  \label{fig:regions}
\end{figure}

\FloatBarrier

\begin{center}
\begin{xltabular}{\linewidth}{l l l}
  \caption{Classification of food commodities}\label{tab:commodities}\\
    \toprule
    \textbf{Category} & \textbf{Commodity} & \textbf{Model code} \\
    \midrule
    \textbf{Primary crops} & Wheat & \texttt{wht} \\
                           & Rice & \texttt{rce} \\
                           & Maize & \texttt{mze} \\
                           & Other cereals & \texttt{crl} \\
                           & Roots and tubers & \texttt{str} \\
                           & Sugar crops & \texttt{sgr} \\
                           & Pulses & \texttt{pls} \\
                           & Nuts & \texttt{nut} \\
                           & Oil crops & \texttt{ocr} \\
                           & Vegetables & \texttt{vgt} \\
                           & Fruits & \texttt{frt} \\
                           & Stimulant crops & \texttt{stm} \\
                           & Spices & \texttt{spc} \\
    \addlinespace[4pt]
    \textbf{Livestock products} & Beef & \texttt{cmt} \\
                                & Sheep and goat meat & \texttt{rmt} \\
                                & Poultry meat & \texttt{pmt} \\
                                & Other meat & \texttt{omt} \\
                                & Eggs & \texttt{egg} \\
                                & Raw milk & \texttt{mlk} \\
    \addlinespace[4pt]
    \textbf{Processed products} & Sugar products & \texttt{swt} \\
                                & Vegetable oils & \texttt{vol} \\
                                & Alcoholic beverages & \texttt{alc} \\
                                & Dairy products & \texttt{dai} \\
    \bottomrule
\end{xltabular}
\end{center}


\section{Variables and Parameters}\label{sec:variables}
% !TeX root = ../ALPHA_doc.tex
This section describes the variables and parameters used in the model.

\subsection{Indices}
Table~\ref{tab:indices} lists the indices used in the model.

\begin{xltabular}{\linewidth}{l X l X}
  \caption{List of indices}\label{tab:indices}\\
  \toprule
  \textbf{Index} & \textbf{Description} & \textbf{Dimension} & \textbf{Notes} \\
  \midrule
  \endfirsthead

  \caption[]{List of indices (continued)}\\
  \toprule
  \textbf{Index} & \textbf{Description} & \textbf{Dimension} & \textbf{Notes} \\
  \midrule
  \endhead

  \midrule
  \multicolumn{4}{r}{\footnotesize Continued on next page}
  \endfoot

  \bottomrule
  \endlastfoot

  \texttt{cty}    & Country / region & 166 & \\
  \texttt{bas}    & River basin      & 400 & \\
  \texttt{region} & Macro region     & 17  & Same as the regional classification in AIM-Hub~\cite{fujimoriSSP3AIMImplementation2017}. See \cref{sec:appendix_b} \\
  \texttt{c(a)}   & Food commodities (production activities) & 23 & See Table~\ref{tab:commodities} \\
  \texttt{acro}   & Primary crops    & 13  & \\
  \texttt{aliv}   & Livestock products & 6 & \\
  \texttt{apro}   & Processed products & 4 & \\
  \texttt{cfeed}  & Feed inputs      & 10  & \\
  \texttt{cint}   & Intermediate inputs & 8 & \\
  \texttt{cbio}   & Biofuel crops    & 3   & \\
  \texttt{water}  & Crop production technology & 2 & Irrigated / rainfed \\
  \texttt{cz}     & Climate zone     & 4 & Tropical, temperate, arid, hyper-arid \\
  \texttt{sys}    & Livestock production system & 2 & Pasture-based, mixed farming \\
  \texttt{src}    & GHG emission sources & 14 &
    Enteric fermentation, manure management, rice cultivation, synthetic fertilizer application, 
    organic fertilizer application, manure on pasture, cultivation of organic soils, crop residues, 
    crop residue burning, grassland burning, forests, natural wildfires, deforestation, afforestation \\
  \texttt{gas}    & Gas species      & 3 & CO\textsubscript{2}, CH\textsubscript{4}, N\textsubscript{2}O \\
  \texttt{age}    & Forest age      & 85 & 0 to 85 (year 2015 to 2100)\\
\end{xltabular}




\subsection{Endogenous Variables}
Table~\ref{tab:endovars} lists the endogenous variables used in the model.

\begin{xltabular}{\linewidth}{l X l}
  \caption{List of endogenous variables}\label{tab:endovars}\\
  \toprule
  \textbf{Variable} & \textbf{Description} & \textbf{Unit} \\
  \midrule
  \endfirsthead

  \caption[]{List of endogenous variables (continued)}\\
  \toprule
  \textbf{Variable} & \textbf{Description} & \textbf{Unit} \\
  \midrule
  \endhead

  \midrule
  \multicolumn{3}{r}{\footnotesize Continued on next page} \\
  \endfoot

  \bottomrule
  \endlastfoot

  \texttt{QDEM} & Domestic demand & 1000t \\
  \texttt{QH} & Household consumption demand & 1000t \\
  \texttt{QF} & Feed demand & 1000t \\
  \texttt{QI} & Intermediate input demand for processed goods & 1000t \\
  \texttt{QBIO} & Biofuel demand & 1000t \\
  \texttt{QSEED} & Seed demand & 1000t \\
  \texttt{QTOUR} & Tourist consumption demand & 1000t \\
  \texttt{QLOSS} & Food loss amount & 1000t \\
  \texttt{QWASTE} & Food waste amount & 1000t \\
  \texttt{QOTH} & Other demand & 1000t \\
  \texttt{QFEED} & Feed input & 1000t \\
  \texttt{QINT} & Raw material input & 1000t \\
  \texttt{QSUP} & Domestic production & 1000t \\
  \texttt{QS} & Production by river basin & 1000t \\
  \texttt{QSWAT} & Crop production by technology & 1000t \\
  \texttt{QSSYS} & Livestock production by system & 1000t \\
  \texttt{QM} & Imports & 1000t \\
  \texttt{QE} & Exports & 1000t \\
  \texttt{NT} & Net trade & 1000t \\
  \texttt{PP} & Producer price & \$/t \\
  \texttt{PC} & Consumer price & \$/t \\
  \texttt{PB} & Border price & \$/t \\
  \texttt{PM} & Import price & \$/t \\
  \texttt{PE} & Export price & \$/t \\
  \texttt{PW} & World price & \$/t \\
  \texttt{YLD} & Crop yield & t/ha \\
  \texttt{SYSSHARE} & Livestock system share & - \\
  \texttt{QGRASS} & Grass utilization & 1000t \\
  \texttt{ANIMALYLD} & Livestock yield per hectare & t/ha \\
  \texttt{ICL} & Feed input coefficient & t/t \\
  \texttt{ANIMAL} & Number of livestock & 1000 head \\
  \texttt{PO} & Price of other production factors & \$/t \\
  \texttt{QLVEG} & Vegetation area & 1000ha \\
  \texttt{QLPAS} & Pasture area & 1000ha \\
  \texttt{QLPASN} & Non-pasture area & 1000ha \\
  \texttt{QLCRO} & Cropland area & 1000ha \\
  \texttt{QLFRS} & Forest area & 1000ha \\
  \texttt{QLGRA} & Other natural land area & 1000ha \\
  \texttt{QLCROCLT} & Harvested cropland area & 1000ha \\
  \texttt{QLCROFAL} & Fallow cropland area & 1000ha \\
  \texttt{QL} & Total production area & 1000ha \\
  \texttt{PLPAS} & Pasture land price & \$/ha \\
  \texttt{PLPASN} & Non-pasture land price & \$/ha \\
  \texttt{PLCRO} & Cropland price & \$/ha \\
  \texttt{PLFRS} & Forest price & \$/ha \\
  \texttt{PLCROCLT} & Harvested cropland price & \$/ha \\
  \texttt{PL} & Land price & \$/ha \\
  \texttt{PENE} & Energy crop land price & \$/ha \\
  \texttt{GHG} & GHG emissions from agriculture sector & ktCO\textsubscript{2}-eq/yr \\
  \texttt{GHGLU} & GHG emissions from land-use sector & ktCO\textsubscript{2}-eq/yr \\
  \texttt{RR} & Emission reduction rate & - \\
  \texttt{AC} & Abatement cost & \$/t \\
  \texttt{PCB} & Carbon price on residual emissions & \$/t \\
  \texttt{PPCOMP} & Complementary variable for production & - \\
  \texttt{TMCOMP} & Complementary variable for imports & - \\
  \texttt{PLCOMP} & Complementary variable for land price & - \\
\end{xltabular}


\subsection{Exogenous Variables}
Table~\ref{tab:exovars} lists the exogenous variables used in the model. For data sources, see \cref{sec:data}.

\begin{xltabular}{\linewidth}{l X l}
  \caption{List of exogenous variables}\label{tab:exovars}\\
  \toprule
  \textbf{Variable} & \textbf{Description} & \textbf{Data Source} \\
  \midrule
  \endfirsthead

  \caption[]{List of exogenous variables (continued)}\\
  \toprule
  \textbf{Variable} & \textbf{Description} & \textbf{Data Source} \\
  \midrule
  \endhead

  \midrule
  \multicolumn{3}{r}{\footnotesize Continued on next page} \\
  \endfoot

  \bottomrule
  \endlastfoot

  \texttt{QHBASE} & Household consumption demand in the base year & FAOSTAT \\
  \texttt{QBIOBASE} & Biofuel demand in the base year & FAOSTAT \\
  \texttt{QSEEDBASE} & Seed demand in the base year & FAOSTAT \\
  \texttt{QTOURBASE} & Tourist consumption demand in the base year & FAOSTAT \\
  \texttt{QLOSSBASE} & Food loss in the base year & FAOSTAT \\
  \texttt{QOTHBASE} & Other demand in the base year & FAOSTAT \\
  \texttt{QSUPPRE} & Domestic production in the previous year & FAOSTAT \\
  \texttt{QSPRE} & Production by river basin in the previous year & GAEZ v4 \\
  \texttt{QENE} & Demand for dedicated energy crops & AIM-Hub \\
  \texttt{QST} & Change in stock & FAOSTAT \\
  \texttt{QRES} & Residual of international trade balance & FAOSTAT \\
  \texttt{LOSSRATE} & Food loss rate & FAOSTAT \\
  \texttt{WASTERATE} & Food waste rate & FAO (2011) \\
  \texttt{GDPPC} & GDP per capita & SSP~3.0~\cite{crespocuaresmaIncomeProjectionsClimate2017} \\
  \texttt{GDPPCBASE} & GDP per capita in the base year & World Bank \\
  \texttt{POP} & Population & SSP~3.0~\cite{kcWittgensteinCenterWIC2024} \\
  \texttt{POPBASE} & Population in the base year & World Bank \\
  \texttt{MMJ} & Market margin & GTAP~10 \\
  \texttt{PSE} & Producer Support Estimate (PSE) & GTAP~10 \\
  \texttt{CSE} & Consumer Support Estimate (CSE) & GTAP~10 \\
  \texttt{CSEBASE} & CSE in the base year & GTAP~10 \\
  \texttt{TM} & Import tariff rate & GTAP~10 \\
  \texttt{MMM} & Import margin rate & GTAP~10 \\
  \texttt{TE} & Export tariff rate & GTAP~10 \\
  \texttt{MME} & Export margin rate & GTAP~10 \\
  \texttt{YLDBASE} & Crop yield in the base year & FAOSTAT, GAEZ~v4 \\
  \texttt{YLDGR} & Yield change rate due to technological progress & IMPACT \\
  \texttt{YLDCHAN} & Yield change rate due to climate change impacts & GAEZ~v4 \\
  \texttt{CROPINTENSITY} & Cropping intensity & MAPSPAM \\
  \texttt{GRASSYLD} & Grass yield & GAEZ~v4 \\
  \texttt{ANIMALPRODRATE} & Slaughtering / milking rate & FAOSTAT \\
  \texttt{FEEDEFFICIENCY} & Feed conversion efficiency & Herrero et~al.~(2018) \\
  \texttt{FEEDSHARE} & Share of concentrate feed & Herrero et~al.~(2018) \\
  \texttt{GRAINSHARE} & Share of grain feed & FAOSTAT \\
  \texttt{ANIMALWEIGHT} & Livestock body weight & FAOSTAT \\
  \texttt{ANIMALRES} & Number of non-productive livestock & FAOSTAT \\
  \texttt{QLRES} & Pasture area not used for livestock production & FAOSTAT, LUH2 \\
  \texttt{POBASE} & Price of other production factors in the base year & FAOSTAT, GTAP~10 \\
  \texttt{QLTOT} & Total land area & LUH2 \\
  \texttt{QLVEGN} & Non-vegetated land area & LUH2 \\
  \texttt{PLFRSBASE} & Forest land price in the base year & LUH2, GTAP~10 \\
  \texttt{PLGRA} & Price of other natural land & LUH2, GTAP~10 \\
  \texttt{PLCROFAL} & Fallow land price & LUH2, GTAP~10 \\
  \texttt{CARBONSTOCK} & Carbon stock & FRA~\cite{FAO2020FRA} \\
  \texttt{CARBONSEQ} & Carbon sequestration by forest age class & AFOLUB~\cite{hasegawaNationalLandbasedClimate2025} \\
  \texttt{EF} & GHG emission factor in agriculture sector & FAOSTAT \\
  \texttt{EFLU} & GHG emission factor in land-use sector & FAOSTAT \\
  \texttt{PRCCAR} & Carbon price & AIM-Hub \\
  \texttt{MRP} & Maximum reduction potential of abatement technologies & Harmsen et~al.~(2019) \\
\end{xltabular}




\subsection{Parameters}
Table~\ref{tab:params} lists the parameters used in the model.

\begin{xltabular}{\linewidth}{l X X}
  \caption{List of parameters}\label{tab:params}\\
  \toprule
  \textbf{Parameter} & \textbf{Description} & \textbf{Notes} \\
  \midrule
  \endfirsthead

  \caption[]{List of parameters (continued)}\\
  \toprule
  \textbf{Parameter} & \textbf{Description} & \textbf{Notes} \\
  \midrule
  \endhead

  \midrule
  \multicolumn{3}{r}{\footnotesize Continued on next page} \\
  \endfoot

  \bottomrule
  \endlastfoot

  \texttt{elhg} & Income elasticity of consumption & Estimated from the food demand model. See \cref{sec:appendix_c}. \\
  \texttt{elhp} & Price elasticity of consumption & Estimated by regression based on Muhammad et~al.~(2011) \\
  \texttt{ica} & Intermediate input coefficient & \\
  \texttt{elop} & Elasticity of other production factors & \\
  $\delta$ & Coefficient parameter of the logit function & \\
  $\rho$ & Exponent parameter of the logit function & For land use, adopted from GCAM~\cite{calvinGCAMV51Representing2019}; for production systems, estimated by regression based on Herrero et~al.~(2018) \\
  \texttt{res} & Residual parameter & \\
  \texttt{watshare} & Share of primary-crop production technologies & \\
  \texttt{czshare} & Climate zone share & \\
  \texttt{grassusagerate} & Grass (pasture) utilization rate & \\
  \texttt{pintcoef} & Price coefficient for intermediate inputs & \\
  \texttt{pfeedcoef} & Price coefficient for feed inputs & \\
  \texttt{pbres} & Residual between consumer price and border price & \\
  \texttt{pmres} & Residual between import price and world price & \\
  \texttt{peres} & Residual between export price and world price & \\
  \texttt{exr} & Exchange rate & \\
  \texttt{dcffrs} & Discount factor for forest carbon value & Introduces the discounted carbon value of forests, reflecting a 60-year forest lifetime and a 5\% interest rate. \\
  \texttt{affcost} & Afforestation cost & Cost of establishing new forest land, assumed at 250~(100~US\$/acre; 1~acre~=~0.4~ha) based on Gorte R.W. (2009). \\
  \texttt{qlfrs\_age} & Forest area by age class & Records forest area by age class, tracking the afforestation area in the past years.\\
  $\sigma$ & Elasticity of adoption of abatement technologies & Estimated by regression based on Harmsen et~al.~(2019) \\
\end{xltabular}


\section{System of Equations}\label{sec:equations}
% !TeX root = ../AIMALPHA_documentation.tex
This section presents the mathematical formulation of the model and its explanation.  
In the following, \textbf{endogenous variables} are denoted by uppercase letters,  
\textbf{exogenous variables} by uppercase letters with an overline, and  
\textbf{parameters} by lowercase or Greek letters.

\subsection{Demand Block}
This block describes the equations related to demand determination.  
In the demand block, food demand is determined by factors such as prices, population, and GDP.

\paragraph{Total Demand}
% The total domestic demand (\( QDEM_{c,cty} \)) is the sum of all types of demand, including household consumption, feed, processing, biofuel, seed, tourism, food loss, food waste, and other demand.
\begin{align}
  QDEM_{c,cty}
  &= QH_{c,cty} + QF_{c,cty} + QI_{c,cty} + QBIO_{c,cty} \nonumber\\
  &\quad + QSEED_{c,cty} + QTOUR_{c,cty} + QLOSS_{c,cty}
    + QWASTE_{c,cty} + QOTH_{c,cty}
  \tag{QDEMEQ}\label{eq:QDEMEQ}
\end{align}

\paragraph{Household Consumption Demand}
\begin{align}
  QH_{c,cty}
  &= \overline{QHBASE}_{c,cty}
     \cdot
     \left(\frac{\overline{GDPPC}_{cty}}{\overline{GDPPCBASE}_{cty}}\right)^{elhg_{c,cty}}
     \nonumber\\
  &\quad \cdot
     \prod_{cc}\!
     \left(
       \frac{\bigl(1-\overline{CSE}_{cc,cty}\bigr)\cdot PC_{cc,cty}}
            {\bigl(1-\overline{CSEBASE}_{cc,cty}\bigr)\cdot \overline{PCBASE}_{cc,cty}}
     \right)^{elhp_{c,cc,cty}}
     \cdot
     \frac{\overline{POP}_{cty}}{\overline{POPBASE}_{cty}}
  \tag{QHEQ}\label{eq:QHEQ}
\end{align}

\paragraph{Feed Demand}
\begin{align}
  QF_{cfeed,cty}
  &= \sum_{bas}\sum_{aliv} QFEED_{aliv,cfeed,cty}
  \tag{QFEEDEQ}\label{eq:QFEEDEQ}
\end{align}

\paragraph{Intermediate Input Demand for Processing}
\begin{align}
  QI_{cint,cty}
  &= \sum_{apro} QINT_{apro,cint,bas,cty}
  \tag{QINTEQ}\label{eq:QINTEQ}
\end{align}

\paragraph{Biofuel Demand}
\begin{align}
  QBIO_{c,cty}
  &= \overline{QBIOBASE}_{c,cty}
     \cdot
     \prod_{cc}\!
     \left(
       \frac{\bigl(1-\overline{CSE}_{cc,cty}\bigr)\cdot PC_{cc,cty}}
            {\bigl(1-\overline{CSEBASE}_{cc,cty}\bigr)\cdot \overline{PCBASE}_{cc,cty}}
     \right)^{elhp_{c,cc,cty}}
  \tag{QBIOEQ}\label{eq:QBIOEQ}
\end{align}

\paragraph{Seed Demand}
\begin{align}
  QSEED_{c,cty}
  &= \overline{QSEEDBASE}_{c,cty}
     \cdot
     \frac{\overline{POP}_{cty}}{\overline{POPBASE}_{cty}}
     \nonumber\\
  &\quad \cdot
     \prod_{cc}\!
     \left(
       \frac{\bigl(1-\overline{CSE}_{cc,cty}\bigr)\cdot PC_{cc,cty}}
            {\bigl(1-\overline{CSEBASE}_{cc,cty}\bigr)\cdot \overline{PCBASE}_{cc,cty}}
     \right)^{elhp_{c,cc,cty}}
  \tag{QSEEDEQ}\label{eq:QSEEDEQ}
\end{align}

\paragraph{Tourist Consumption Demand}
\begin{align}
  QTOUR_{c,cty}
  &= \overline{QTOURBASE}_{c,cty}
     \cdot
     \prod_{cc}\!
     \left(
       \frac{\bigl(1-\overline{CSE}_{cc,cty}\bigr)\cdot PC_{cc,cty}}
            {\bigl(1-\overline{CSEBASE}_{cc,cty}\bigr)\cdot \overline{PCBASE}_{cc,cty}}
     \right)^{elhp_{c,cc,cty}}
  \tag{QTOUREQ}\label{eq:QTOUREQ}
\end{align}

\paragraph{Loss and Waste}
\begin{align}
  QLOSS_{c,cty}
  &= QDEM_{c,cty} \cdot \overline{LOSSRATE}_{c,cty}
  \tag{QLOSSEQ}\label{eq:QLOSSEQ}\\[.3em]
  QWASTE_{c,cty}
  &= QH_{c,cty} \cdot \overline{WASTERATE}_{c,cty}
  \tag{QWASTEEQ}\label{eq:QWASTEEQ}
\end{align}

\paragraph{Other Demand}
\begin{align}
  QOTH_{c,cty}
  &= \overline{QOTHBASE}_{c,cty}
     \cdot
     \prod_{cc}\!
     \left(
       \frac{\bigl(1-\overline{CSE}_{cc,cty}\bigr)\cdot PC_{cc,cty}}
            {\bigl(1-\overline{CSEBASE}_{cc,cty}\bigr)\cdot \overline{PCBASE}_{cc,cty}}
     \right)^{elhp_{c,cc,cty}}
  \tag{QOTHEQ}\label{eq:QOTHEQ}
\end{align}


\subsection{Production Block}

The production block describes the equations determining the production quantities of each food commodity, the input of intermediate goods and production factors, and the producer prices.  
The production functions are represented by Leontief-type functions, where each good is produced through the combination of specific input factors.

\paragraph{Domestic Production}
\begin{align}
  QSUP_{a,cty}
  &= \sum_{bas} QS_{a,bas,cty}
  \tag{QSUPEQ}\label{eq:QSUPEQ}
\end{align}

\subsubsection{Crop Production}

Primary crops are differentiated by production technologies (irrigated and rainfed).
Yield improvements due to technological progress and climate impacts are considered for crops.

\paragraph{Crop Production by Technology}
\begin{align}
  QSWAT_{acro,bas,cty,water}
  &= QS_{acro,bas,cty} \cdot watshare_{acro,bas,cty,water}
  \tag{CROWATEREQ}\label{eq:CROWATEREQ}
\end{align}

\paragraph{Producer Price of Crops}
\begin{align}
  PP_{acro,cty}
  &= \Bigl(
      PL_{acro,bas,cty}
      \cdot
      \sum_{water}
      \frac{
        watshare_{acro,bas,cty,water}
      }{
        YLD_{acro,bas,cty,water}
        \cdot
        \overline{CROPINTENSITY}_{acro,bas,cty,water}
      }
      \nonumber\\[-0.3em]
  &\quad
      +\; PO_{acro,bas,cty}
     \Bigl)
     \,/\, (1 + PPCOMP_{acro,bas,cty})
  \tag{PPCROEQ}\label{eq:PPCROEQ}
\end{align}

\paragraph{Land Input for Crop Production}
\begin{align}
  QL_{acro,bas,cty}
  &= QS_{acro,bas,cty}
     \cdot
     \sum_{water}
     \frac{
       watshare_{acro,bas,cty,water}
     }{
       YLD_{acro,bas,cty,water}
       \cdot
       \overline{CROPINTENSITY}_{acro,bas,cty,water}
     }
  \tag{QLCROINPUTEQ}\label{eq:QLCROINPUTEQ}
\end{align}

\paragraph{Crop Yield}
\begin{align}
  YLD_{acro,bas,cty,water}
  &= \overline{YLDBASE}_{acro,bas,cty,water}
     \cdot
     \overline{YLDGR}_{acro,bas,cty,water}
     \nonumber\\
  &\quad
     \cdot
     \overline{YLDCHAN}_{acro,bas,cty,water}
  \tag{YLDEQ}\label{eq:YLDEQ}
\end{align}

\paragraph{Other Production Factor Price (Crops)}
\begin{align}
  PO_{acro,bas,cty}
  &= \overline{POBASE}_{acro,bas,cty}
     \cdot
     \left(
       \frac{QS_{acro,bas,cty}}
            {\overline{QSPRE}_{acro,bas,cty}}
     \right)^{elop_{acro,cty}}
  \tag{POCROEQ}\label{eq:POCROEQ}
\end{align}


\subsubsection{Livestock Production}

The livestock production block describes equations for determining livestock output, production systems, feed requirements, and associated prices.  
Livestock products are distinguished by climate zones and production systems (pasture-based or mixed).  
Feed use and production efficiency depend on feed composition, feed efficiency, and animal productivity.

\paragraph{Livestock Production by System}
\begin{align}
  QSSYS_{aliv,bas,cty,cz,sys}
  &= QS_{aliv,bas,cty}
     \cdot czshare_{aliv,bas,cty,cz}
     \cdot SYSSHARE_{aliv,bas,cty,cz,sys}
  \tag{LIVSYSEQ}\label{eq:LIVSYSEQ}
\end{align}

\paragraph{Livestock System Share Function}
\begin{align}
  SYSSHARE_{aliv,bas,cty,cz,sys}
  &= \frac{1}{
       1 + \delta^{sys}_{aliv,cz,sys}
       \cdot
       \overline{GDPPC}_{cty}^{\,\rho^{sys}_{aliv,sys}}
     }
     + res_{aliv,bas,cty,cz,sys}
  \tag{SYSSHAREEQ}\label{eq:SYSSHAREEQ}
\end{align}

\paragraph{Producer Price of Livestock}
\begin{align}
  PP_{aliv,cty}
  &= \Bigl(
       PL_{aliv,bas,cty}
       / ANIMALYLD_{aliv,bas,cty}
       \nonumber\\[-0.3em]
  &\quad
       + pfeedcoef_{aliv,cty}
         \cdot
         \sum_{cfeed}
         (1 - \overline{CSE}_{cfeed,cty})
         \cdot
         PC_{cfeed,cty}
         \nonumber\\[-0.3em]
  &\quad
         \cdot
         \sum_{cz,sys}
         \Bigl(
           czshare_{aliv,bas,cty,cz}
           \cdot
           SYSSHARE_{aliv,bas,cty,cz,sys}
           \cdot
           ICL_{aliv,cfeed,bas,cty,cz,sys}
         \Bigr)
         \nonumber\\[-0.3em]
  &\quad
       + PO_{aliv,bas,cty}
     \Bigr)
     / (1 + PPCOMP_{aliv,bas,cty})
  \tag{PPLIVEQ}\label{eq:PPLIVEQ}
\end{align}

\paragraph{Grass Input Calculation}
\begin{align}
  Qgrass_{aliv,bas,cty}
  &= 
     \frac{
       QS_{aliv,bas,cty}
     }{
       \overline{ANIMALPRODRATE}_{aliv,cty}
     }
     \nonumber\\[-0.3em]
  &\quad
     \cdot
     \sum_{cz,sys}
     \Bigl(
       czshare_{aliv,bas,cty,cz}
       \cdot
       SYSSHARE_{aliv,bas,cty,cz,sys}
       \nonumber\\[-0.3em]
  &\quad
       \cdot
       \overline{FEEDEFFICIENCY}_{aliv,bas,cty,cz,sys}
       \cdot
       (1 - \overline{FEEDSHARE}_{aliv,bas,cty,cz,sys})
     \Bigr)
  \tag{QGRASSINPUTEQ}\label{eq:QGRASSINPUTEQ}
\end{align}

\paragraph{Available Grassland Area}
\begin{align}
  Qgrass_{aliv,bas,cty}
  &= 
     \Bigl(
       QL_{aliv,bas,cty}
       - 
       \overline{QLRES}_{aliv,bas,cty}
     \Bigr)
     \nonumber\\[-0.3em]
  &\quad
     \cdot
     \overline{GRASSYLD}_{bas,cty}
     \cdot
     grassusagerate_{bas,cty}
  \tag{QLLIVINPUTEQ}\label{eq:QLLIVINPUTEQ}
\end{align}


\paragraph{Livestock Yield}
\begin{align}
  ANIMALYLD_{aliv,bas,cty}
  &= \frac{
       QS_{aliv,bas,cty}
     }{
       QL_{aliv,bas,cty}
       - \overline{QLRES}_{aliv,bas,cty}
     }
  \tag{ANIMALYLDEQ}\label{eq:ANIMALYLDEQ}
\end{align}

\paragraph{Feed Input Calculation}
\begin{align}
  QFEED_{aliv,cfeed,bas,cty}
  &= \frac{
       QS_{aliv,bas,cty}
     }{
       \overline{ANIMALPRODRATE}_{aliv,cty}
     }
     \nonumber\\[-0.3em]
  &\quad
     \cdot
     \sum_{cz,sys}
     \bigl(
       czshare_{aliv,bas,cty,cz}
       \cdot SYSSHARE_{aliv,bas,cty,cz,sys}
       \nonumber\\[-0.3em]
  &\quad
       \cdot ICL_{aliv,cfeed,bas,cty,cz,sys}
     \bigr)
  \tag{QFEEDINPUTEQ}\label{eq:QFEEDINPUTEQ}
\end{align}

\paragraph{Feed Input Coefficient}
\begin{multline}
  ICL_{aliv,cfeed,bas,cty,cz,sys} = \overline{FEEDEFFICIENCY}_{aliv,bas,cty,cz,sys}\\
     \cdot
     \overline{FEEDSHARE}_{aliv,bas,cty,cz,sys}
     \cdot
     \overline{GRAINSHARE}_{aliv,cfeed,bas,cty,cz,sys}
  \tag{ICLEQ}\label{eq:ICLEQ}
\end{multline}

\paragraph{Other Production Factor Price (Livestock)}
\begin{align}
  PO_{aliv,bas,cty}
  &= \overline{POBASE}_{aliv,bas,cty}
     \cdot
     \left(
       \frac{
         QS_{aliv,bas,cty}
       }{
         \overline{QSPRE}_{aliv,bas,cty}
       }
     \right)^{elop_{aliv,cty}}
  \tag{POLIVEQ}\label{eq:POLIVEQ}
\end{align}

\paragraph{Livestock Production Quantity}
\begin{align}
  QS_{aliv,bas,cty}
  &= \overline{ANIMALWEIGHT}_{aliv,cty}
     \cdot
     (
       ANIMAL_{aliv,bas,cty}
       - \overline{ANIMALRES}_{aliv,bas,cty}
     )
     \nonumber\\[-0.3em]
  &\quad
     \cdot
     \overline{ANIMALPRODRATE}_{aliv,cty}
  \tag{ANIMALEQ}\label{eq:ANIMALEQ}
\end{align}

\subsubsection{Processed Goods produciton}

This section describes the production equations for processed agricultural products.  
Processed products are produced using intermediate inputs and other production factors.

\paragraph{Producer Price of Processed Products}
\begin{align}
  PP_{apro,cty}
  &= pintcoef_{apro,cty}
     \cdot
     \sum_{cint}
     \Bigl(
       (1 - \overline{CSE}_{cint,cty})
       \cdot
       PC_{cint,cty}
       \cdot
       ica_{apro,cint,cty}
     \Bigr)
     \nonumber\\[-0.3em]
  &\quad
     + PO_{apro,cty}
  \tag{PPPROEQ}\label{eq:PPPROEQ}
\end{align}

\paragraph{Intermediate Input Demand for Processed Products}
\begin{align}
  QINT_{apro,cint,cty}
  &= QSUP_{apro,cty}
     \cdot
     ica_{apro,cint,cty}
  \tag{QINTINPUTEQ}\label{eq:QINTINPUTEQ}
\end{align}

\paragraph{Other Production Factor Price (Processed Products)}
\begin{align}
  PO_{apro,cty}
  &= \overline{POBASE}_{apro,cty}
     \cdot
     \left(
       \frac{
         QSUP_{apro,cty}
       }{
         \overline{QSUPRE}_{apro,cty}
       }
     \right)^{elop_{apro,cty}}
  \tag{POPROEQ}\label{eq:POPROEQ}
\end{align}

\subsubsection{Energy Crop Production}
Energy crop production is determined from regional totals provided exogenously by the AIM-Hub model~\cite{fujimoriSSP3AIMImplementation2017}.


\paragraph{Regional Energy Crop Supply Balance}
\begin{align}
  \overline{QENE}_{region}
  &= \sum_{bas,cty \in region}
     QS_{\mathrm{ecr},bas,cty}
  \tag{QSENEEQ}\label{eq:QSENEEQ}
\end{align}

\paragraph{Land Input for Energy Crops}
\begin{align}
  QL_{\mathrm{ecr},bas,cty}
  &= \frac{
       QS_{\mathrm{ecr},bas,cty}
     }{
       YLD_{\mathrm{ecr},bas,cty}
     }
  \tag{QLENEINPUTEQ}\label{eq:QLENEINPUTEQ}
\end{align}

\paragraph{Land Price for Energy Crops}
\begin{align}
  PL_{\mathrm{ecr},bas,cty}
  &= PENE_{region}
  \tag{PLENEEQ}\label{eq:PLENEEQ}
\end{align}


\subsection{Price Block}

This subsection describes the equations in the price block.  
The price block represents the relationships among various prices in the model.  
Prices are linked through fixed coefficients so that the price used in the demand and supply functions is consistently determined.

\paragraph{Consumer and Producer Prices}
\begin{align}
  (1 + \overline{MMJ}_{a,cty})
  \cdot PP_{a,cty}
  + PCB_{a,cty}
  + AC_{a,cty}
  &= PC_{a,cty}
  \tag{PCEQ}\label{eq:PCEQ}
\end{align}

\paragraph{Consumer and Border Prices}
\begin{align}
  PC_{c,cty}
  &= (1 + \overline{PSE}_{c,cty})
     \cdot PB_{c,cty}
     \cdot (1 + pbres_{c,cty})
  \tag{PBEQ}\label{eq:PBEQ}
\end{align}

\paragraph{Import Price}
\begin{align}
  PM_{c,cty}
  &= PW_{c}
     \cdot exr_{cty}
     \cdot
     (1 + \overline{TM}_{c,cty} + TMCOMP_{c,cty})
     \nonumber\\[-0.3em]
  &\quad
     \cdot (1 + \overline{MMM}_{c,cty})
     \cdot (1 + pmres_{c,cty})
  \tag{PMEQ}\label{eq:PMEQ}
\end{align}

\paragraph{Export Price}
\begin{align}
  PE_{c,cty}
  \cdot (1 + \overline{TE}_{c,cty})
  \cdot (1 + \overline{MME}_{c,cty})
  &= PW_{c}
     \cdot (1 + peres_{c,cty})
     \cdot exr_{cty}
  \tag{PEEQ}\label{eq:PEEQ}
\end{align}


\subsection{Land Use Block}

The land-use block allocates available land among different uses through a Logit function.  
Prices for non-human-used land (forest, other natural land, and fallow land) are fixed at their baseline values.  
A constraint is imposed to ensure that total agricultural land does not exceed the physically available land area due to slope and soil limitations.  

\paragraph{Vegetation Area}
\begin{align}
  QLVEG_{bas,cty}
  &= 
     \overline{QLTOT}_{bas,cty}
     - 
     \overline{QLVEGN}_{bas,cty}
  \tag{QLVEGEQ}\label{eq:QLVEGEQ}
\end{align}

\paragraph{Pasture Area Allocation}
\begin{align}
  QLPAS_{bas,cty}
  &=
    QLVEG_{bas,cty}
    \nonumber\\[-0.3em]
  &\quad\cdot
    \frac{
      \delta^{\text{pas}}_{bas,cty}
      \cdot
      PLPAS_{bas,cty}^{\,\rho^{\text{pas}}_{bas,cty}}
    }{
      \delta^{\text{pas}}_{bas,cty}
      \cdot
      PLPAS_{bas,cty}^{\,\rho^{\text{pas}}_{bas,cty}}
      +
      \delta^{\text{pasn}}_{bas,cty}
      \cdot
      PLPASN_{bas,cty}^{\,\rho^{\text{pasn}}_{bas,cty}}
    }
  \tag{QLPASEQ}\label{eq:QLPASEQ}
\end{align}


\paragraph{Non-pasture Grassland Area}
\begin{align}
  QLPASN_{bas,cty}
  &=
    QLVEG_{bas,cty}
    \nonumber\\[-0.3em]
  &\quad\cdot
    \frac{
      \delta^{\text{pasn}}_{bas,cty}
      \, PLPASN_{bas,cty}^{\,\rho^{\text{pasn}}_{bas,cty}}
    }{
      \delta^{\text{pas}}_{bas,cty}
      \, PLPAS_{bas,cty}^{\,\rho^{\text{pas}}_{bas,cty}}
      +
      \delta^{\text{pasn}}_{bas,cty}
      \, PLPASN_{bas,cty}^{\,\rho^{\text{pasn}}_{bas,cty}}
    }
  \tag{QLPASNEQ}\label{eq:QLPASNEQ}
\end{align}

\paragraph{Cropland Area}
\begin{multline}
  QLCRO_{bas,cty} = QLPASN_{bas,cty}\\
  \cdot
  \frac{
    \delta^{\text{cro}}_{bas,cty}\, PLCRO_{bas,cty}^{\,\rho^{\text{cro}}_{bas,cty}}
  }{
    \delta^{\text{cro}}_{bas,cty}\, PLCRO_{bas,cty}^{\,\rho^{\text{cro}}_{bas,cty}}
    + \delta^{\text{frs}}_{bas,cty}\, PLFRS_{bas,cty}^{\,\rho^{\text{frs}}_{bas,cty}}
    + \delta^{\text{gra}}_{bas,cty}\, \overline{PLGRA}_{bas,cty}^{\,\rho^{\text{gra}}_{bas,cty}}
  }
  \tag{QLCROEQ}\label{eq:QLCROEQ}
\end{multline}

\paragraph{Forest Area}
\begin{multline}
  QLFRS_{bas,cty} = QLPASN_{bas,cty}\\
  \cdot
  \frac{
    \delta^{\text{frs}}_{bas,cty}\, PLFRS_{bas,cty}^{\,\rho^{\text{frs}}_{bas,cty}}
  }{
    \delta^{\text{cro}}_{bas,cty}\, PLCRO_{bas,cty}^{\,\rho^{\text{cro}}_{bas,cty}}
    + \delta^{\text{frs}}_{bas,cty}\, PLFRS_{bas,cty}^{\,\rho^{\text{frs}}_{bas,cty}}
    + \delta^{\text{gra}}_{bas,cty}\, \overline{PLGRA}_{bas,cty}^{\,\rho^{\text{gra}}_{bas,cty}}
  }
  \tag{QLFRSEQ}\label{eq:QLFRSEQ}
\end{multline}

\paragraph{Other Natural Land}
\begin{multline}
  QLGRA_{bas,cty} = QLPASN_{bas,cty}\\
  \cdot
  \frac{
    \delta^{\text{gra}}_{bas,cty}\, PLGRA_{bas,cty}^{\,\rho^{\text{gra}}_{bas,cty}}
  }{
    \delta^{\text{cro}}_{bas,cty}\, PLCRO_{bas,cty}^{\,\rho^{\text{cro}}_{bas,cty}}
    + \delta^{\text{frs}}_{bas,cty}\, PLFRS_{bas,cty}^{\,\rho^{\text{frs}}_{bas,cty}}
    + \delta^{\text{gra}}_{bas,cty}\, \overline{PLGRA}_{bas,cty}^{\,\rho^{\text{gra}}_{bas,cty}}
  }
  \tag{QLGRAEQ}\label{eq:QLGRAEQ}
\end{multline}

\paragraph{Harvested Cropland Area}
\begin{multline}
  QLCROCLT_{bas,cty} = QLCRO_{bas,cty}\\
  \cdot
  \frac{
    \delta^{\text{croclt}}_{bas,cty}\, PLCROCLT_{bas,cty}^{\,\rho^{\text{croclt}}_{bas,cty}}
  }{
    \delta^{\text{croclt}}_{bas,cty}\, PLCROCLT_{bas,cty}^{\,\rho^{\text{croclt}}_{bas,cty}}
    + \delta^{\text{crofal}}_{bas,cty}\, \overline{PLCROFAL}_{bas,cty}^{\,\rho^{\text{crofal}}_{bas,cty}}
  }
  \tag{QLCROCLTEQ}\label{eq:QLCROCLTEQ}
\end{multline}

\paragraph{Fallow Cropland Area}
\begin{multline}
  QLCROFAL_{bas,cty} = QLCRO_{bas,cty}\\
  \cdot
  \frac{
    \delta^{\text{crofal}}_{bas,cty}\, \overline{PLCROFAL}_{bas,cty}^{\,\rho^{\text{crofal}}_{bas,cty}}
  }{
    \delta^{\text{croclt}}_{bas,cty}\, PLCROCLT_{bas,cty}^{\,\rho^{\text{croclt}}_{bas,cty}}
    + \delta^{\text{crofal}}_{bas,cty}\, \overline{PLCROFAL}_{bas,cty}^{\,\rho^{\text{crofal}}_{bas,cty}}
  }
  \tag{QLCROFALEQ}\label{eq:QLCROFALEQ}
\end{multline}


\paragraph{Cropland Allocation}
\begin{align}
  QL_{acro,bas,cty}
  &= QLCROCLT_{bas,cty}
     \cdot
     \frac{
       \delta_{acro,bas,cty}\cdot PL_{acro,bas,cty}^{\,\rho_{acro,bas,cty}}
     }{
       \sum_{acro}\delta_{acro,bas,cty}\cdot PL_{acro,bas,cty}^{\,\rho_{acro,bas,cty}}
     }
  \tag{QLCRODISTEQ}\label{eq:QLCRODISTEQ}
\end{align}

\paragraph{Pasture Allocation}
\begin{align}
  QL_{aliv,bas,cty}
  &= QLPAS_{bas,cty}
     \cdot
     \frac{
       \delta_{aliv,bas,cty}\cdot PL_{aliv,bas,cty}^{\,\rho_{aliv,bas,cty}}
     }{
       \sum_{aliv}\delta_{aliv,bas,cty}\cdot PL_{aliv,bas,cty}^{\,\rho_{aliv,bas,cty}}
     }
  \tag{QLLIVDISTEQ}\label{eq:QLLIVDISTEQ}
\end{align}


\paragraph{Land Prices}
\begin{equation}
  PLCROCLT_{bas,cty}\cdot QLCROCLT_{bas,cty}
  = \sum_{acro} PL_{acro,bas,cty}\cdot QL_{acro,bas,cty}
  \tag{PLCROCLTEQ}\label{eq:PLCROCLTEQ}
\end{equation}

\vspace{0.3\baselineskip}

\begin{multline}
  PLCRO_{bas,cty}\cdot QLCRO_{bas,cty}
  = \bigl(
       PLCROCLT_{bas,cty}\cdot QLCROCLT_{bas,cty}
       \\[-0.3em]
  \quad
       + \overline{PLCROFAL}_{bas,cty}\cdot QLCROFAL_{bas,cty}
     \bigr)
  \quad / (1 + PLCOMP_{bas,cty})
  \tag{PLCROEQ}\label{eq:PLCROEQ}
\end{multline}

\vspace{0.3\baselineskip}

\begin{multline}
  PLPASN_{bas,cty}\cdot QLPASN_{bas,cty}
  = PLCRO_{bas,cty}\cdot QLCRO_{bas,cty}
    \\[-0.3em]
  \quad + PLFRS_{bas,cty}\cdot QLFRS_{bas,cty}
  \quad + \overline{PLGRA}_{bas,cty}\cdot QLGRA_{bas,cty}
  \tag{PLPASNEQ}\label{eq:PLPASNEQ}
\end{multline}

\vspace{0.3\baselineskip}

\begin{equation}
  PLPAS_{bas,cty}\cdot QLPAS_{bas,cty}
  = \frac{\sum_{aliv} PL_{aliv,bas,cty}\cdot QL_{aliv,bas,cty}}
          {1 + PLCOMP_{bas,cty}}
  \tag{PLPASEQ}\label{eq:PLPASEQ}
\end{equation}

\vspace{0.3\baselineskip}

\begin{multline}
  PLFRS_{bas,cty}
  = \overline{PLFRSBASE}_{bas,cty}
  \\[-0.3em]
  \quad
    + \max\Bigl(
        0,\,
        \overline{CARBONSTOCK}_{cty} \cdot
        \overline{PRCCAR}_{cty} \cdot
        44/12 \cdot
        dcffrs_{cty}
        - affcost_{cty}
      \Bigr)
  \tag{PLFRSEQ}\label{eq:PLFRSEQ}
\end{multline}

\paragraph{Agricultural Land Constraint}
\begin{align}
  \overline{QLLIM}_{cty}
  &\ge
  QLCRO_{bas,cty} + QLPAS_{bas,cty}
  \;\perp\;
  PLCOMP_{bas,cty} \ge 0
  \tag{QLLIMEQ}\label{eq:QLLIMEQ}
\end{align}


\subsection{Market Block}

This subsection describes the equations in the market block, where consistency between food demand and supply is ensured.  
Trade of food commodities is represented as a complementarity problem.  
When the import price is higher than the border price, the import quantity becomes zero.  
Once the import price decreases to equal the border price, imports begin.  
Similarly, when the border price is higher than the export price, exports become zero, and once the export price rises to match the border price, exports take place.

\paragraph{Domestic Trade Balance}
\begin{align}
  NT_{c,cty}
  &= QSUP_{c,cty} - QDEM_{c,cty} - \overline{QST}_{c,cty}
  \tag{NTEQ}\label{eq:NTEQ}\\[0.6em]
  NT_{c,cty}
  &= QE_{c,cty} - QM_{c,cty}
  \tag{QEQMEQ}\label{eq:QEQMEQ}
\end{align}

\paragraph{International Trade Balance}
\begin{align}
  \sum_{cty} NT_{c,cty} - \overline{QRES}_{c} = 0
  \tag{NT0EQ}\label{eq:NT0EQ}
\end{align}

\paragraph{Import and Export Conditions}
\begin{align}
  PM_{c,cty} &\ge PB_{c,cty}
  \quad \perp \quad
  QM_{c,cty} \ge 0
  \tag{PBPMEQ}\label{eq:PBPMEQ}\\[0.4em]
  PB_{c,cty} &\ge PE_{c,cty}
  \quad \perp \quad
  QE_{c,cty} \ge 0
  \tag{PBPEEQ}\label{eq:PBPEEQ}
\end{align}



\subsection{GHG Emission Block}

This subsection describes the equations in the GHG emission block.  
GHG emissions are calculated based on the production quantities of each food commodity.  
When carbon prices are considered, the amount of mitigation technology adoption and its cost are computed from the MAC (Marginal Abatement Cost) curve.  
Emissions from the land-use sector are calculated according to each land area.

\paragraph{GHG Emissions}
\begin{align}
  GHG_{a,src,gas,cty}
  &= QSUP_{a,cty}
     \cdot
     \overline{EF}_{a,src,gas,cty}
     \cdot
     (1 - RR_{a,src,gas,cty})
  \tag{GHGEQ}\label{eq:GHGEQ}
\end{align}

\paragraph{Land-Use Emissions}
\begin{align}
  GHGLU_{src,gas,bas,cty}
  &= \overline{EF\_LU}_{src,gas,cty}
     \cdot QLFRS_{bas,cty}
     + \overline{EF\_LU}_{src,gas,cty}
       \cdot QLGRA_{bas,cty}
     \nonumber\\[-0.3em]
  &\quad
     + \overline{EF\_LU}_{src,gas,cty}
       \cdot QLCRO_{bas,cty}
     + \overline{EF\_LU}_{src,gas,cty}
       \cdot QLPAS_{bas,cty}
  \tag{GHGLUEQ}\label{eq:GHGLUEQ}
\end{align}
\begin{align}
  GHGLU_{"afr","CO_2",bas,cty}
  &= 
  - \sum_{age}
      qlfrs\_age_{bas,cty,age}
      \cdot \overline{CARBONSEQ}_{cty,age}
  \tag{GHGLUAFREQ}\label{eq:GHGLUAFREQ}
\end{align}

\paragraph{Emission Reduction Rate}
\begin{align}
  RR_{a,src,gas,cty}
  &= \overline{MRP}_{src,gas,cty}
     \cdot
     \Bigl(
       1 -
       \bigl[
         1 / (1 + \overline{PRCCAR}_{cty})
       \bigr]^{\sigma_{src,gas,cty}}
     \Bigr)
  \tag{RREQ}\label{eq:RREQ}
\end{align}

\paragraph{Abatement Cost}
\begin{align}
  AC_{a,cty}
  &= \sum_{src,gas}
     \overline{EF}_{a,src,gas,cty}
     \cdot
     \Biggl[
       \frac{\sigma_{src,gas,cty}}
            {(\sigma_{src,gas,cty}-1)}
       \cdot
       \overline{MRP}_{src,gas,cty}
       \nonumber\\[-0.3em]
  &\quad
       \cdot
       \Bigl(
         1 -
         \bigl(
           1 -
           RR_{a,src,gas,cty} /
           \overline{MRP}_{src,gas,cty}
         \bigr)^{(1 - 1/\sigma_{src,gas,cty})}
       \Bigr)
       - RR_{a,src,gas,cty}
     \Biggr]
  \tag{ACEQ}\label{eq:ACEQ}
\end{align}

\paragraph{Carbon Cost on Remaining Emissions}
\begin{align}
  PCB_{a,cty}
  &= \sum_{src,gas}
     \overline{PRCCAR}_{cty}
     \cdot
     \overline{EF}_{a,src,gas,cty}
     \cdot
     (1 - RR_{a,src,gas,cty})
  \tag{PCBEQ}\label{eq:PCBEQ}
\end{align}


\subsection{Complementarity Conditions}

This subsection describes the complementarity conditions introduced to avoid numerical errors in the Logit function when production quantities in specific production units become zero.  
If the production of a unit becomes zero, its land-use area would also become zero, leading to instability.  
Therefore, a minimum positive value is imposed to ensure that production quantities do not fall to zero.

\paragraph{Minimum Production Share for Trade}
\begin{align}
  \frac{QSUP_{c,cty}}{QM_{c,cty}}
  &\ge 0.01
  \quad \perp \quad
  TMCOMP_{c,cty} \ge 0
  \tag{DEPQMEQ}\label{eq:DEPQMEQ}
\end{align}

\paragraph{Minimum Production Share for Supply}
\begin{align}
  \frac{QS_{a,bas,cty}}{QSUP_{a,cty}}
  &\ge 0.01
  \quad \perp \quad
  PPCOMP_{a,bas,cty} \ge 0
  \tag{QSBOUEQ}\label{eq:QSBOUEQ}
\end{align}


\section{Data and Preprocessing}\label{sec:data}
% !TeX root = ../ALPHA_doc.tex
This section explains the calibration and processing of data used in the model.  
Exogenous variables and parameters were estimated using multiple datasets, adjusted to 
obtain equilibrium in the base year, and parameters were estimated by regression analysis.  
\subsection{Data Sources}
Table~\ref{tab:data_sources} lists the datasets used in the model.

\begin{xltabular}{\linewidth}{l X X}
  \caption{List of data sources}\label{tab:data_sources}\\
  \toprule
  \textbf{Data Source} & \textbf{Dataset Name} & \textbf{URL} \\
  \midrule
  \endfirsthead

  \caption[]{List of data sources (continued)}\\
  \toprule
  \textbf{Data Source} & \textbf{Dataset Name} & \textbf{URL} \\
  \midrule
  \endhead

  \midrule
  \multicolumn{3}{r}{\footnotesize Continued on next page} \\
  \endfoot

  \bottomrule
  \endlastfoot

  FAOSTAT & Food Balances   & \nolinkurl{https://www.fao.org/faostat/en/\#data/FBS} \\
  FAOSTAT & Production      & \nolinkurl{https://www.fao.org/faostat/en/\#data/QCL} \\
  FAOSTAT & Trade           & \nolinkurl{https://www.fao.org/faostat/en/\#data/TCL} \\
  FAOSTAT & Producer Prices & \nolinkurl{https://www.fao.org/faostat/en/\#data/PP} \\
  FAOSTAT & Land Use        & \nolinkurl{https://www.fao.org/faostat/en/\#data/RL} \\
  FAOSTAT & Emissions       & \nolinkurl{https://www.fao.org/faostat/en/\#data/GCE} \\
  AQUASTAT & Major hydrological basins of the world & \nolinkurl{https://data.apps.fao.org/catalog//iso/7707086d-af3c-41cc-8aa5-323d8609b2d1} \\
  FAO and IIASA & GAEZ v4   & \nolinkurl{https://gaez.fao.org/} \\
  IIASA & SSP 3.0          & \nolinkurl{https://data.ece.iiasa.ac.at/ssp/\#/about} \\
  GTAP  & GTAP 10          & \nolinkurl{https://jgea.org/ojs/index.php/jgea/article/view/77} \\
  IFPRI & MAPSPAM          & \nolinkurl{https://doi.org/10.7910/DVN/SWPENT} \\
  World Bank & World Bank Open Data & \nolinkurl{https://data.worldbank.org/} \\
  EDGAR & EDGAR v8.0       & \nolinkurl{https://edgar.jrc.ec.europa.eu/dataset_ghg80} \\
  Herrero Acosta et~al.~(2018) & Livestock production systems & \nolinkurl{https://doi.org/10.4225/08/5aa068b33fe06} \\
  Muhammad et~al.~(2011)   & International Evidence on Food Consumption Patterns & \nolinkurl{http://199.135.94.241/publications/pub-details/?pubid=47581} \\
  T.~P.~Robinson et~al.~(2018) & Global distribution of ruminant livestock production systems V5 & \nolinkurl{https://doi.org/10.7910/DVN/WPDSZE} \\
  Harmsen et~al.~(2019)    & CH\textsubscript{4} and N\textsubscript{2}O MAC curves & \nolinkurl{https://doi.org/10.1016/j.envsci.2019.05.013} \\
  Hurtt et~al.~(2019)      & Land Use Harmonization 2 & \nolinkurl{https://luh.umd.edu/index.shtml} \\
\end{xltabular}

\subsection{Calibration and processing of data}
\subsubsection{Adjustment of demand-side data}

First, the adjustment of demand-side data was conducted. 
In the model, if a country–commodity pair has missing data for the base year 2015, it is assumed that consumption does not occur thereafter. 
Therefore, for country–commodity pairs without 2015 data in the Food Balance Sheets (FBS), the average value of subsequent years was used as the 2015 consumption level.

\begin{equation}
QH_{(c,cty,2015)} = \frac{\sum_{yr} QH^{FAO}_{(c,cty,yr)}}{N_{yr}} 
\quad \text{for } QH_{(c,cty,2015)} = \text{NA}
\end{equation}

where \( QH^{FAO} \) is household consumption data from FAO, 
and \( N_{yr} \) is the number of years for which data exist.

In the FBS data, household consumption includes the amount of food wasted at the household level. 
To distinguish this, waste rates by seven commodities and seven regions were obtained from FAO (2011). 
Using these coefficients, household consumption and food waste were estimated as follows:

\begin{align}
QWASTE_{(c,cty)} &= QH^{FAO}_{(c,cty)} \times WASTERATE_{(c,cty)} \\
QH_{(c,cty)} &= QH^{FAO}_{(c,cty)} \times (1 - WASTERATE_{(c,cty)})
\end{align}

For model simplification, the commodities used as livestock feed, processing materials, and biofuel inputs were limited. 
Wheat, rice, maize, other cereals, roots and tubers, sugar crops, pulses, oil crops, and vegetables were used as livestock feed. 
Wheat, rice, maize, other cereals, and fruits were used as raw materials for alcoholic beverages. 
Sugar crops were used as inputs for sugars, oil crops for vegetable oils, and maize, sugar crops, and vegetable oils for biofuels. 
Feed demand, intermediate input demand, and biofuel demand for other commodities were set to zero.

\begin{align}
QF_{(c \notin c_{feed},cty)} &= 0 \\
QBIO_{(c \notin c_{bio},cty)} &= 0 \\
QI_{(c \notin c_{int},cty)} &= 0
\end{align}

As a result of these adjustments, the total domestic demand changed. 
If the sum of all demand components was less than the total demand from the data, 
the residual was added as “other demand”:

\begin{align}
QOTH_{(c,cty)} &= QDEM_{(c,cty)} - 
  \nonumber\\[-0.3em]
  &\quad
  \big(QH_{(c,cty)} + QF_{(c,cty)} + QI_{(c,cty)} + QSEED_{(c,cty)} + 
  \nonumber\\[-0.3em]
  &\quad
  QTOUR_{(c,cty)} + QLOSS_{(c,cty)} + QWASTE_{(c,cty)}\big)
\end{align}


\subsubsection{Adjustment of crop production data}
Next, the adjustment of crop production data was conducted. 
From the FBS data, the production quantities of primary crops for each country were obtained. 
These were disaggregated into production units and production technologies using the GAEZ data as follows:

\begin{equation}
QSWAT_{(acro,bas,cty,water)} = 
QSUP^{FAO}_{(acro,cty)} \times 
\frac{QS^{GAEZ}_{(acro,bas,cty,water)}}{\sum_{(bas,water)} QS^{GAEZ}_{(acro,bas,cty,water)}}
\end{equation}

where \( QSUP^{FAO} \) denotes production quantity data from FAO, 
and \( QS^{GAEZ} \) represents production data from GAEZ.

From the LUH2 dataset, the cultivated land area for each production unit was obtained. 
Using the FAOSTAT data on national land use, these cultivated lands were divided into harvested and fallow lands as follows:

\begin{align}
QLCROCLT_{(bas,cty)} &= 
QLCRO^{LUH2}_{(bas,cty)} \times 
\frac{QLCROCLT^{FAO}_{cty}}{QLCROCLT^{FAO}_{cty} + QLCROFAL^{FAO}_{cty}} \\
QLCROFAL_{(bas,cty)} &= 
QLCRO^{LUH2}_{(bas,cty)} \times 
\frac{QLCROFAL^{FAO}_{cty}}{QLCROCLT^{FAO}_{cty} + QLCROFAL^{FAO}_{cty}}
\end{align}

where \( QLCRO^{LUH2} \) denotes cultivated area data from LUH2, 
\( QLCROCLT^{FAO} \) is harvested land area data from FAO, 
and \( QLCROFAL^{FAO} \) is fallow land area data from FAO.

Furthermore, the harvested land was disaggregated by crop type and production technology 
using GAEZ data as follows:

\begin{equation}
QLWAT_{(acro,bas,cty,water)} = 
QLCROCLT_{(bas,cty)} \times 
\frac{QL^{GAEZ}_{(acro,bas,cty,water)}}{\sum_{(acro,water)} QL^{GAEZ}_{(acro,bas,cty,water)}}
\end{equation}

where \( QLWAT \) is the cropland area by crop type and production technology, 
and \( QL^{GAEZ} \) is the crop-specific land area from GAEZ.

From the MAPSPAM dataset, harvested and physical area data were obtained for each crop, 
production unit, and production technology. 
Using these data, the cropping intensity was estimated as follows:

\begin{equation}
CROPINTESNSITY_{(acro,bas,cty,water)} = 
\frac{QLHARVESTED^{MAPSPAM}_{(acro,bas,cty,water)}}{QLPHYSICAL^{MAPSPAM}_{(acro,bas,cty,water)}}
\end{equation}

where \( QLHARVESTED^{MAPSPAM} \) is the harvested area data from MAPSPAM, 
and \( QLPHYSICAL^{MAPSPAM} \) is the physical area data from MAPSPAM.

Crop yields were calculated using the obtained production quantities, harvested areas, 
and cropping intensities as follows:

\begin{equation}
YLD_{(acro,bas,cty,water)} = 
\frac{QSWAT_{(acro,bas,cty,water)}}{QLWAT_{(acro,bas,cty,water)} \times CROPINTSNSITY_{(acro,bas,cty,water)}}
\end{equation}


\subsubsection{Adjustment of livestock production data}

Next, the adjustment of livestock production data was conducted. 
From FAO data, the numbers of livestock animals and slaughtered (or milked) animals were obtained. 
Using these, the production rate was calculated. 
A minimum threshold of 0.3 was set for the production rate; if it fell below this value, 
it was assumed that a portion of livestock was not involved in production activities.

\begin{align}
ANIMALPRODRATE_{(aliv,cty)} &= 
\frac{ANIMAL^{FAO}_{(aliv,cty)}}{ANIMALSTOCK^{FAO}_{(aliv,cty)}} \\
ANIMALRES_{(aliv,cty)} &= 
ANIMALSTOCK^{FAO}_{(aliv,cty)} - 
\frac{ANIMAL^{FAO}_{(aliv,cty)}}{0.3} 
\quad (\text{if } ANIMALPRODRATE_{(aliv,cty)} < 0.3)
\end{align}

where \( ANIMAL^{FAO} \) is the number of slaughtered (or milked) animals from FAO, 
and \( ANIMALSTOCK^{FAO} \) is the total livestock population.

When non-productive livestock exist, the corresponding unused pasture area was estimated as:

\begin{equation}
QLPASRES_{(aliv,cty)} = 
QLPAS^{LUH2}_{(aliv,cty)} \times 
\frac{ANIMALRES_{(aliv,cty)}}{ANIMALSTOCK^{FAO}_{(aliv,cty)}}
\end{equation}

where \( QLPASRES \) denotes the unused pasture area, 
and \( QLPAS^{LUH2} \) is the pasture area from LUH2.

Country-level livestock production data from the FBS were disaggregated 
into production units, climate zones, and production systems 
using the dataset from Herrero Acosta et al. (2018). 
Similarly, LUH2 pasture area data were disaggregated by livestock type, 
climate zone, and production system.

\begin{align}
QSSYS_{(aliv,bas,cty,cz,sys)} &= 
QSUP^{FAO}_{(aliv,cty)} \times 
\frac{QS^{HERRERO}_{(aliv,bas,cty,cz,sys)}}
{\sum_{(bas,cz,sys)} QS^{HERRERO}_{(aliv,bas,cty,cz,sys)}} \\
QLSYS_{(aliv,bas,cty,cz,sys)} &= 
\big(QLPAS^{LUH2}_{(cty,bas)} - QLPASRES_{(aliv,cty)}\big) 
\times 
\frac{QGRASS^{HERRERO}_{(aliv,bas,cty,cz,sys)}}
{\sum_{(aliv,cz,sys)} QGRASS^{HERRERO}_{(aliv,bas,cty,cz,sys)}}
\end{align}

where \( QS^{HERRERO} \) denotes production data 
and \( QGRASS^{HERRERO} \) represents pasture use data 
from Herrero Acosta et al. (2018).

Next, the feed efficiency and concentrate feed share were estimated as follows:

\begin{align}
FEEDEFFICIENCY_{(aliv,bas,cty,cz,sys)} &= 
\frac{QF^{HERRERO}_{(aliv,bas,cty,cz,sys)} + QGRASS^{HERRERO}_{(aliv,bas,cty,cz,sys)}}
{QS^{HERRERO}_{(aliv,bas,cty,cz,sys)}} \\
FEEDSHARE_{(aliv,bas,cty,cz,sys)} &= 
\frac{QF^{HERRERO}_{(aliv,bas,cty,cz,sys)}}
{QF^{HERRERO}_{(aliv,bas,cty,cz,sys)} + QGRASS^{HERRERO}_{(aliv,bas,cty,cz,sys)}}
\end{align}

where \( QF^{HERRERO} \) is the amount of concentrate feed from Herrero Acosta et al. (2018).

Using these values and the FBS feed demand data, 
the quantities of pasture and concentrate feed use were estimated. 
The amount of concentrate feed input was adjusted with a correction coefficient 
to ensure consistency with the FBS data. 
Consequently, the feed efficiency and concentrate feed share were also adjusted. 
If the estimated pasture use exceeded the available supply, 
it was replaced with the available amount.

\begin{align}
QFEEDSYS^{temp}_{(aliv,cfeed,bas,cty,cz,sys)} &= 
\frac{QSSYS_{(aliv,bas,cty,cz,sys)}}{ANIMALPRODRATE_{(aliv,cty)}} 
\times FEEDEFFICIENCY_{(aliv,bas,cty,cz,sys)} 
\times FEEDSHARE_{(aliv,bas,cty,cz,sys)} 
\times \frac{QF^{FAO}_{(cfeed,cty)}}{\sum_{cfeed} QF^{FAO}_{(cfeed,cty)}} \\
qfcoef_{(cfeed,cty)} &= 
\frac{QF^{FAO}_{(cfeed,cty)}}{\sum_{(aliv,bas,cz,sys)} QFEEDSYS^{temp}_{(aliv,cfeed,bas,cty,cz,sys)}} \\
QFEEDSYS_{(aliv,cfeed,bas,cty,cz,sys)} &= 
QFEEDSYS^{temp}_{(aliv,cfeed,bas,cty,cz,sys)} \times qfcoef_{(cfeed,cty)} \\
QGRASSSYS_{(aliv,bas,cty,cz,sys)} &= 
\frac{QSSYS_{(aliv,bas,cty,cz,sys)}}{ANIMALPRODRATE_{(aliv,cty)}} 
\times FEEDEFFICIENCY_{(aliv,bas,cty,cz,sys)} 
\times (1 - FEEDSHARE_{(aliv,bas,cty,cz,sys)})
\end{align}

If the estimated pasture demand exceeded the available supply, 
it was replaced with the available amount:

\begin{equation}
QGRASSSYS_{(aliv,bas,cty,cz,sys)} = 
QLSYS_{(aliv,bas,cty,cz,sys)} \times GRASSYLD_{(bas,cty)}
\quad \text{if } 
QLSYS_{(aliv,bas,cty,cz,sys)} \times GRASSYLD_{(bas,cty)} 
< QGRASSSYS_{(aliv,bas,cty,cz,sys)}
\end{equation}

where:
\begin{itemize}
\item \( QFEEDSYS^{temp} \): estimated feed demand by production system,  
\item \( QF^{FAO} \): feed demand data from FAO,  
\item \( qfcoef \): adjustment coefficient for consistency with FAO feed demand,  
\item \( QFEEDSYS \): adjusted feed demand consistent with FAO data,  
\item \( QGRASSSYS \): pasture demand by production system.
\end{itemize}


\subsubsection{Adjustment of food trade}

Next, the adjustment of food trade data was conducted. 
In order to express trade through complementary conditions, 
the net import and net export quantities were derived as follows:

\begin{align}
QM_{(c,cty)} &= QM^{FAO}_{(c,cty)} - QE^{FAO}_{(c,cty)} 
\quad (\text{if } QM^{FAO}_{(c,cty)} > QE^{FAO}_{(c,cty)}) \\
QE_{(c,cty)} &= QE^{FAO}_{(c,cty)} - QM^{FAO}_{(c,cty)} 
\quad (\text{if } QE^{FAO}_{(c,cty)} > QM^{FAO}_{(c,cty)})
\end{align}

where \( QM^{FAO} \) denotes the import quantity data from FAO, 
and \( QE^{FAO} \) denotes the export quantity data from FAO.

In the FBS dataset, global trade is not balanced; 
that is, the total export quantity and total import quantity of the world do not match.  
Therefore, the global discrepancy between total exports and imports was treated as a residual.  
This residual can be interpreted as trade with countries or regions 
outside the model scope, as well as losses occurring during international food transport.

\begin{equation}
QRES_{c} = \sum_{cty} QM_{(c,cty)} - \sum_{cty} QE_{(c,cty)}
\end{equation}



\subsubsection{Estimation of land constraints}

Next, the estimation method for land areas unsuitable for agricultural use 
due to soil and topographical constraints is described.  
From the GAEZ dataset, area data for 57 agro-ecological zone (AEZ) classes, 
classified according to climate, soil, and topographical conditions, 
were obtained.  
Among these, land classes with strong constraints were selected 
and considered unsuitable for agricultural use.  
The area of such constrained land was calculated as follows:

\begin{equation}
QLLIM_{(bas,cty)} = 
QLTOT^{LUH2}_{(bas,cty)} \times 
\frac{\sum_{aezlim} QL^{GAEZ}_{(aezlim,bas,cty)}}
{\sum_{aez} QL^{GAEZ}_{(aez,bas,cty)}}
\end{equation}

where:
\begin{itemize}
\item \( QLTOT^{LUH2} \): total land area from LUH2,
\item \( QL^{GAEZ} \): area data by AEZ class from GAEZ,
\item \( aez \): the set of 57 AEZ classes,
\item \( aezlim \in aez \): AEZ classes with severe land-use constraints.
\end{itemize}


\subsubsection{Adjustment of price data}

From the FAO dataset, producer prices, import prices, and export prices 
for each country and commodity were obtained. 
Since the price data contained many outliers and missing values, 
appropriate preprocessing was conducted.  

First, outlier values were treated as missing when the producer price 
was greater than ten times or less than one-tenth of the country-level median.

\begin{equation}
PP_{(c,cty)} \leftarrow \text{NA} 
\quad \text{if } 
\big( PP_{(c,cty)} > 10 \times \text{median}_{cty}(PP_{(c,cty)}) 
\ \cup \ 
PP_{(c,cty)} < 0.1 \times \text{median}_{cty}(PP_{(c,cty)}) \big)
\end{equation}

Next, missing values were complemented.  
A world average price was calculated from the available country-level producer prices, 
weighted by production quantities.  
A country-specific correction coefficient was then derived 
based on the deviation of each country's prices from the world average.  
This coefficient was used to impute the missing prices.  
The same procedure was applied for import and export prices.

\begin{align}
PP_{ave,c} &= 
\frac{\sum_{cty} PP_{(c,cty)} \times QSUP_{(c,cty)}}{\sum_{cty} QSUP_{(c,cty)}} \\
PP_{coef,cty} &= 
\frac{\sum_{c} PP_{(c,cty)}}{PP_{ave,c}} \\
PP_{(c,cty)} &= 
PP_{ave,c} \times PP_{coef,cty}
\end{align}

where \( PP_{ave} \) denotes the global average producer price,  
and \( PP_{coef} \) denotes the country-specific adjustment coefficient.

\bigskip

Next, using the GTAP10 dataset, exogenous variables related to prices were estimated.  
The Producer Support Estimate (PSE) represents the monetary transfers 
from consumers and taxpayers to producers resulting from agricultural policies.  
In this study, the PSE is expressed as the proportionate impact of 
subsidies and taxes on producer prices.  
Similarly, the Consumer Support Estimate (CSE) represents 
the transfers from or to consumers, 
expressed as the proportionate impact of subsidies and taxes on consumer prices.

\begin{align}
PSE_{(i,cty)} &= 
\frac{
OSEP^{GTAP}_{(i,cty)} 
+ \sum_i ISEP^{GTAP}_{(i,ii,cty)} 
- \sum_{endw} FTRV^{GTAP}_{(i,endw,cty)} 
+ \sum_{endw} FBEP^{GTAP}_{(i,endw,cty)}
}{
VOA^{GTAP}_{(i,cty)}
} \\
CSE_{(i,cty)} &= 
\frac{VDPM^{GTAP}_{(i,cty)} - VDPA^{GTAP}_{(i,cty)}}{VDPM^{GTAP}_{(i,cty)}}
\end{align}

where:
\begin{itemize}
\item \( OSEP^{GTAP} \): subsidies/taxes on production,  
\item \( ISEP^{GTAP} \): subsidies/taxes on intermediate inputs,  
\item \( FTRV^{GTAP} \): taxes on factor inputs,  
\item \( FBEP^{GTAP} \): subsidies on factor inputs,  
\item \( VOA^{GTAP} \): production value,  
\item \( VDPM^{GTAP}, VDPA^{GTAP} \): household consumption at market and consumer prices, respectively.
\end{itemize}

\bigskip

To link producer (farm-gate) prices with consumer (market) prices, 
the market margin ratio \( MMJ \) was estimated.  
Intermediate inputs and factor inputs corresponding to market margins 
were identified from the GTAP input–output tables, 
and their total was divided by the production value to obtain the ratio.  
Similarly, trade-related costs \( MMM \) and \( MME \) were estimated from GTAP data,  
using the difference between import values at CIF prices and export values at FOB prices.

\begin{align}
MMJ_{(i,cty)} &= 
\frac{
\sum_{immj} VDFA^{GTAP}_{(i,immj,cty)} 
+ \sum_{immj} VIFA^{GTAP}_{(i,immj,cty)} 
+ \sum_{immj} EVFA^{GTAP}_{(i,immj,cty)}
}{
\Big(
\sum_{ii} VDF^{A,GTAP}_{(i,ii,cty)} 
+ \sum_{ii} VIF^{A,GTAP}_{(i,ii,cty)} 
+ \sum_{endw} EVF^{A,GTAP}_{(i,endw,cty)} 
- 
(\sum_{immj} VDF^{A,GTAP}_{(i,immj,cty)} 
+ \sum_{immj} VIF^{A,GTAP}_{(i,immj,cty)} 
+ \sum_{immj} EVF^{A,GTAP}_{(i,immj,cty)})
\Big)
} \\
MMM_{(i,ccty)} &= 
\frac{\sum_{cty} VIWS^{GTAP}_{(i,cty,ccty)} - \sum_{cty} VXWD^{GTAP}_{(i,cty,ccty)}}
{\sum_{cty} VXWD^{GTAP}_{(i,cty,ccty)}} \\
MME_{(i,cty)} &= 
\frac{\sum_{ccty} VIWS^{GTAP}_{(i,cty,ccty)} - \sum_{ccty} VXWD^{GTAP}_{(i,cty,ccty)}}
{\sum_{ccty} VXWD^{GTAP}_{(i,cty,ccty)}}
\end{align}

where:
\begin{itemize}
\item \( VDFA^{GTAP}, VIFA^{GTAP}, EVFA^{GTAP} \): intermediate and factor input values at consumer prices,  
\item \( VIWS^{GTAP}, VXWD^{GTAP} \): import (CIF) and export (FOB) values,  
\item \( immj \in (i \cup endw) \): goods and factors corresponding to market margins.
\end{itemize}

\bigskip

The import tariff (\( TM \)) and export tariff (\( TE \)) were then calculated.  
The import tariff rate was obtained as the ratio of import value at domestic prices 
to that at international prices, 
and the export tariff rate as the ratio of export value at international prices 
to that at domestic prices.

\begin{align}
TM_{(i,ccty)} &= 
\frac{\sum_{cty} VIMS^{GTAP}_{(i,cty,ccty)}}{\sum_{cty} VIWS^{GTAP}_{(i,cty,ccty)}} \\
TE_{(i,cty)} &= 
\frac{\sum_{ccty} VXWD^{GTAP}_{(i,cty,ccty)}}{\sum_{ccty} VXMD^{GTAP}_{(i,cty,ccty)}}
\end{align}

where:
\begin{itemize}
\item \( VIMS^{GTAP} \): import value at domestic (market) prices,  
\item \( VXMD^{GTAP} \): export value at domestic (market) prices.
\end{itemize}

\bigskip

Finally, coefficients for decomposing producer prices by input factor 
were estimated.  
Each agricultural commodity is produced using land, feed, raw materials, 
and other production factors.  
Using the GTAP input–output tables, the cost share of each factor 
in the producer price was estimated.  
By multiplying the producer price by these shares, 
the base-year prices of each factor were obtained.

\begin{align}
LND_{(i,cty)} &= 
\frac{
\sum_{ilnd} VDF^{A,GTAP}_{(i,ilnd,cty)} 
+ \sum_{ilnd} VIF^{A,GTAP}_{(i,ilnd,cty)} 
+ \sum_{ilnd} EVF^{A,GTAP}_{(i,ilnd,cty)}
}{
\Big(
\sum_{ii} VDF^{A,GTAP}_{(i,ii,cty)} 
+ \sum_{ii} VIF^{A,GTAP}_{(i,ii,cty)} 
+ \sum_{endw} EVF^{A,GTAP}_{(i,endw,cty)} 
-
(\sum_{immj} VDF^{A,GTAP}_{(i,immj,cty)} 
+ \sum_{immj} VIF^{A,GTAP}_{(i,immj,cty)} 
+ \sum_{immj} EVF^{A,GTAP}_{(i,immj,cty)})
\Big)
} \\[6pt]
FED_{(i,cty)} &= 
\frac{
\sum_{ifed} VDF^{A,GTAP}_{(i,ifed,cty)} 
+ \sum_{ifed} VIF^{A,GTAP}_{(i,ifed,cty)} 
+ \sum_{ifed} EVF^{A,GTAP}_{(i,ifed,cty)}
}{
\Big(
\sum_{ii} VDF^{A,GTAP}_{(i,ii,cty)} 
+ \sum_{ii} VIF^{A,GTAP}_{(i,ii,cty)} 
+ \sum_{endw} EVF^{A,GTAP}_{(i,endw,cty)} 
-
(\sum_{immj} VDF^{A,GTAP}_{(i,immj,cty)} 
+ \sum_{immj} VIF^{A,GTAP}_{(i,immj,cty)} 
+ \sum_{immj} EVF^{A,GTAP}_{(i,immj,cty)})
\Big)
} \\[6pt]
IMD_{(i,cty)} &= 
\frac{
\sum_{iimd} VDF^{A,GTAP}_{(i,iimd,cty)} 
+ \sum_{iimd} VIF^{A,GTAP}_{(i,iimd,cty)} 
+ \sum_{iimd} EVF^{A,GTAP}_{(i,iimd,cty)}
}{
\Big(
\sum_{ii} VDF^{A,GTAP}_{(i,ii,cty)} 
+ \sum_{ii} VIF^{A,GTAP}_{(i,ii,cty)} 
+ \sum_{endw} EVF^{A,GTAP}_{(i,endw,cty)} 
-
(\sum_{immj} VDF^{A,GTAP}_{(i,immj,cty)} 
+ \sum_{immj} VIF^{A,GTAP}_{(i,immj,cty)} 
+ \sum_{immj} EVF^{A,GTAP}_{(i,immj,cty)})
\Big)
} \\[6pt]
OFC_{(i,cty)} &= 
1 - (LND_{(i,cty)} + FED_{(i,cty)} + IMD_{(i,cty)})
\end{align}

where:
\begin{itemize}
\item \( LND \): share of land cost in the producer price,  
\item \( FED \): share of feed cost in the producer price,  
\item \( IMD \): share of raw material cost in the producer price,  
\item \( OFC \): share of other production factors in the producer price,  
\item \( ilnd, ifed, iimd \in (i \cup endw) \): goods and factors corresponding to land, feed, and raw materials,  
\item The denominator represents the total value of domestic and imported intermediate inputs 
      and factor inputs, excluding those corresponding to market margin sectors.
\end{itemize}



\section{Parameter Estimation}\label{sec:param}
% !TeX root = ../ALPHA_doc.tex
\subsection{Parameter estimation through regression analysis}

This section presents the results of regression analyses used to estimate the parameters of the model.  
In the following equations, variables with an overline represent the dependent and independent variables used in the regressions.

\bigskip

Since price elasticities in the model were assumed to vary with per-capita GDP, 
the following regression equation was estimated:

\begin{equation}
\overline{elhp}_{(c,cty)} = \alpha^{pe}_{c} + \beta^{pe}_{c} \cdot \log(\overline{GDPPC}_{cty})
\label{eq:pe_reg}
\end{equation}

\bigskip

For livestock products, it was assumed that production systems change with economic development.  
Thus, the parameters determining the production share were estimated using logistic regression analysis.  
In this estimation, dummy variables were introduced to represent climatic zones.

\begin{equation}
\log \left(
\frac{
\overline{SYSSHARE}_{(aliv,bas,cty,cz,sys)}
}{
1 - \overline{SYSSHARE}_{(aliv,bas,cty,cz,sys)}
}
\right)
= 
\delta^{sys}_{(aliv,cz,sys)} + \rho^{sys}_{(aliv,sys)} \cdot \log(\overline{GDPPC}_{cty})
\label{eq:sys_reg}
\end{equation}

\bigskip

It was further assumed that feed requirements and the share of concentrate feed 
also change with the level of economic development.  
Regression analyses were conducted using the following equations:

\begin{align}
\log \left( \overline{FEEDEFFICIENCY}_{(aliv,bas,cty,cz,sys)} \right)
&= 
\alpha^{fe}_{(aliv,cz,sys)} + \beta^{fe}_{(aliv,cz,sys)} \cdot \log(\overline{GDPPC}_{cty}) \label{eq:fe_reg}\\[6pt]
\log \left(
\frac{
\overline{FEEDSHARE}_{(aliv,bas,cty,cz,sys)}
}{
1 - \overline{FEEDSHARE}_{(aliv,bas,cty,cz,sys)}
}
\right)
&=
\delta^{fs}_{(aliv,cz,sys)} + \rho^{fs}_{(aliv,sys)} \cdot \log(\overline{GDPPC}_{cty})
\label{eq:fs_reg}
\end{align}

\bigskip

Following Harmsen et al.\ (2019), the elasticity parameter 
\(\sigma\) related to the adoption of emission-reduction technologies 
was estimated using the following regression equation:

\begin{equation}
\log \left(
1 - 
\frac{
\overline{RR}_{(src,gas,cty)}
}{
\overline{MRP}_{(src,gas,cty)}
}
\right)
=
\sigma_{(src,gas,cty)} \cdot \log(\overline{PRCCAR}_{cty} + 1)
\label{eq:sigma_reg}
\end{equation}

\bigskip

\subsection{Results of parameter estimation}

This section presents the results of the regression analyses conducted to estimate the model parameters.  
Variables with an overline in the equations represent the data used in the regressions.

\subsubsection{Relationship between price elasticity and per-capita GDP}

Figure~\ref{fig:gdp_elasticity} illustrates the relationship between per-capita GDP and price elasticity.  
Table~\ref{tab:price_elasticity} shows the estimated parameters and the coefficient of determination ($R^2$).  

Since the model assumes that price elasticity varies with per-capita GDP, 
the parameters $\alpha^{pe}_c$ and $\beta^{pe}_c$ were estimated using the regression model
given in Eq.~\eqref{eq:pe_reg}.

For all commodities, the estimated elasticities increased with higher levels of per-capita GDP.  
For most commodities, high coefficients of determination ($R^2 \approx 0.8$) were obtained.  
However, for commodities such as sugars, eggs, and spices, $R^2$ values were lower, around 0.5 or 0.3.  
The $p$-values for all regressions were below 0.001, indicating statistically significant relationships 
between the variables.  
Furthermore, the 95\% confidence intervals for $\beta^{pe}$ did not include zero, 
confirming the statistical significance of the estimated coefficients.

\begin{figure}[H]
    \centering
    \includegraphics[width=\textwidth]{fig6_1}
    \caption{Relationship between per-capita GDP and price elasticity}
    \label{fig:gdp_elasticity}
\end{figure}

\bigskip

\begin{table}[H]
    \centering
    \caption{Estimated parameters of price elasticity from regression analysis}
    \label{tab:price_elasticity}

    \csvreader[
        tabular={l r r r r r l},
        table head=\toprule
            Commodity & $\alpha_{pe}$ & $\beta_{pe}$ & \multicolumn{2}{c}{$\beta_{pe}$ (95\% CI)} & $R^2$ & $p$-value \\
            \cmidrule(lr){4-5}
            & & & lower & upper & & \\
        \midrule,
        late after line=\\,
        table foot=\bottomrule
    ]{data/table6_1.csv}%
    {%
      c_list=\commodity,
      alpha=\alphaval,
      beta=\betaval,
      lower=\betalow,
      upper=\betahigh,
      r2=\Rtwo,
      p=\pvalue
    }%
    {%
        \commodity &
        \num[round-mode=places,round-precision=2]{\alphaval} &
        \num[round-mode=places,round-precision=2]{\betaval} &
        \num[round-mode=places,round-precision=2]{\betalow} &
        \num[round-mode=places,round-precision=2]{\betahigh} &
        \num[round-mode=places,round-precision=2]{\Rtwo} &
        $<0.001$%
    }
\end{table}




\bigskip

\subsubsection{Relationship between livestock production system share and per-capita GDP}

Figure~\ref{fig:sys_share_gdp} illustrates the relationship between per-capita GDP and 
the proportion of pasture-based production by livestock type and climatic zone.  
Table~\ref{tab:sys_share_gdp} presents the estimated parameters.

The parameters $\delta^{sys}_{(aliv,cz,sys)}$ and $\rho^{sys}_{(aliv,sys)}$ 
were estimated using the logistic regression model in Eq.~\eqref{eq:sys_reg}.

As per-capita GDP increases, the share of pasture-based production decreases for all livestock products.  
Adjusted coefficients of determination were moderate (0.2–0.3), and all $p$-values were below 0.001,  
indicating statistically significant relationships.  
The 95\% confidence intervals for $\rho^{sys}$ did not include zero,  
confirming the significance of the coefficients.

\begin{figure}[H]
    \centering
    \includegraphics[width=\textwidth]{fig6_2}
    \caption{Relationship between per-capita GDP and share of pasture-based production}
    \label{fig:sys_share_gdp}
\end{figure}

\begin{table}[H]
\centering
\caption{Estimated parameters for pasture-based production share (95\% confidence intervals in parentheses)}
\label{tab:sys_share_gdp}
\csvreader[
    tabular={l l r r r r r l},
    table head=\toprule
        Commodity & $\rho_{sys}$ & \multicolumn{4}{c}{$\delta_{sys}$ by climate zone} & $R^2$ & $p$-value \\
        \cmidrule(lr){3-6}
        & & Arid & Temperate & Humid & HyperArid & & \\
    \midrule,
    late after line=\\,
    table foot=\bottomrule
]{data/table6_2.csv}%
{%
  c_list=\commodity,
  ro=\rhoval,
  delta_A=\arid,
  delta_T=\temp,
  delta_H=\humid,
  delta_Y=\hyper,
  r2=\Rtwo,
  lower=\lowerci,
  upper=\upperci,
  p=\pval
}%
{%
    \commodity &
    % rho_sys with 95% CI, rounded
    $\num[round-mode=places,round-precision=2]{\rhoval}$%
    \,($\num[round-mode=places,round-precision=2]{\lowerci}$--$\num[round-mode=places,round-precision=2]{\upperci}$) &
    \num[round-mode=places,round-precision=2]{\arid} &
    \num[round-mode=places,round-precision=2]{\temp} &
    \num[round-mode=places,round-precision=2]{\humid} &
    \num[round-mode=places,round-precision=2]{\hyper} &
    \num[round-mode=places,round-precision=2]{\Rtwo} &
    $<0.001$%
}
\end{table}



\bigskip

\subsubsection{Relationship between feed efficiency and per-capita GDP}

Figure~\ref{fig:feed_eff_gdp} shows the relationship between per-capita GDP and feed efficiency 
by livestock type, climate zone, and production system.  
Table~\ref{tab:feed_eff_gdp} presents the estimated regression parameters,
obtained using the model in Eq.~\eqref{eq:fe_reg}.

Higher per-capita GDP is associated with lower feed requirements.  
For ruminants, the adjusted $R^2$ ranged from 0.66 to 0.97, indicating high explanatory power,  
while for poultry and other livestock it was relatively low (0.22 and 0.05, respectively).  
The coefficients $\beta^{fe}$ were smaller for ruminants, suggesting stronger efficiency improvements 
with economic development.  
All $p$-values were below 0.001, and the 95\% confidence intervals for $\alpha^{fe}$ 
did not include zero, confirming statistical significance.


\begin{figure}[H]
    \centering
    \includegraphics[width=\textwidth]{fig6_3}
    \caption{Relationship between per-capita GDP and feed efficiency}
    \label{fig:feed_eff_gdp}
\end{figure}

\begin{table}[H]
\centering
\scriptsize
\setlength{\tabcolsep}{3pt}
\caption{Estimated parameters of feed efficiency (95\% confidence intervals in parentheses)}
\label{tab:feed_eff_gdp}
\resizebox{\textwidth}{!}{%
\csvreader[
    tabular={@{} l l l r r r r r r l @{}},
    table head=\toprule
        Commodity & System & $\beta_{fe}$ & \multicolumn{5}{c}{$\alpha_{fe}$ by climate zone} & $R^2$ & $p$-value \\
        \cmidrule(lr){4-8}
        & & & Arid & Temperate & Humid & HyperArid & Total & & \\
    \midrule,
    late after line=\\,
    table foot=\bottomrule
]{data/table6_3.csv}% 
{%
  c_list   = \commodity,
  beta     = \betaval,
  alpha_A  = \alphaA,
  alpha_T  = \alphaT,
  alpha_H  = \alphaH,
  alpha_Y  = \alphaY,
  alpha_tot= \alphaTot,
  r2       = \Rtwo,
  lower    = \lowerci,
  upper    = \upperci,
  p        = \pval,
  sys      = \system
}% 
{%
    \commodity &
    \system &
    % beta_fe with 95% CI
    $\num[round-mode=places,round-precision=2]{\betaval}$%
    \,($\num[round-mode=places,round-precision=2]{\lowerci}$--$\num[round-mode=places,round-precision=2]{\upperci}$) &
    % alpha_fe by climate zone 
    \IfStrEq{\alphaA}{NA}{}{%
      \num[round-mode=places,round-precision=2]{\alphaA}} &
    \IfStrEq{\alphaT}{NA}{}{%
      \num[round-mode=places,round-precision=2]{\alphaT}} &
    \IfStrEq{\alphaH}{NA}{}{%
      \num[round-mode=places,round-precision=2]{\alphaH}} &
    \IfStrEq{\alphaY}{NA}{}{%
      \num[round-mode=places,round-precision=2]{\alphaY}} &
    \IfStrEq{\alphaTot}{NA}{}{%
      \num[round-mode=places,round-precision=2]{\alphaTot}} &
    \num[round-mode=places,round-precision=2]{\Rtwo} &
    $<0.001$%
}
}% end resizebox
\end{table}

\bigskip

\subsubsection{Relationship between concentrate feed share and per-capita GDP}

Figure~\ref{fig:feedshare_gdp} shows the relationship between per-capita GDP and 
the share of concentrate feed by livestock type, climatic zone, and production system.  
Table~\ref{tab:feedshare_gdp} reports the estimated regression parameters,
based on the logistic specification in Eq.~\eqref{eq:fs_reg}.  
Poultry and other livestock are excluded from the analysis because they do not consume pasture.

The results indicate that higher per-capita GDP is associated with 
a higher share of concentrate feed and a lower share of pasture.  
Adjusted coefficients of determination range from 0.74 to 0.90, 
indicating a relatively high explanatory power.  
All $p$-values are below 0.001, suggesting statistically significant relationships 
between the variables.  
Furthermore, the 95\% confidence intervals for $\rho^{fs}$ do not include zero, 
which confirms the statistical significance of the estimated coefficients.

\begin{figure}[H]
    \centering
    \includegraphics[width=\textwidth]{fig6_4}
    \caption{Relationship between per-capita GDP and concentrate feed share}
    \label{fig:feedshare_gdp}
\end{figure}

\begin{table}[H]
\centering
\caption{Estimated parameters of concentrate feed share 
(95\% confidence intervals in parentheses)}
\label{tab:feedshare_gdp}

\scriptsize
\setlength{\tabcolsep}{3pt}%
\resizebox{0.95\textwidth}{!}{%
\csvreader[%
    tabular={l l l r r r r r l},
    table head=\toprule
        Commodity & System & $\rho_{fs}$ & \multicolumn{4}{c}{$\delta_{fs}$ by climate zone} & $R^2$ & $p$-value \\
        \cmidrule(lr){4-7}
        & & & Arid & Temperate & Humid & HyperArid & & \\
    \midrule,
    late after line=\\,
    table foot=\bottomrule
]{data/table6_4.csv}%
{%
  c_list  = \commodity,
  sys     = \system,
  beta    = \rhoval, 
  alpha_A = \deltaA,
  alpha_T = \deltaT,
  alpha_H = \deltaH,
  alpha_Y = \deltaY,
  r2      = \Rtwo,
  lower   = \lowerci,
  upper   = \upperci,
  p       = \pval
}%
{%
    \commodity &
    \system &
    % rho_fs with 95% CI
    $\num[round-mode=places,round-precision=2]{\rhoval}$%
    \,($\num[round-mode=places,round-precision=2]{\lowerci}$--$\num[round-mode=places,round-precision=2]{\upperci}$) &
    \num[round-mode=places,round-precision=2]{\deltaA} &
    \num[round-mode=places,round-precision=2]{\deltaT} &
    \num[round-mode=places,round-precision=2]{\deltaH} &
    \num[round-mode=places,round-precision=2]{\deltaY} &
    \num[round-mode=places,round-precision=2]{\Rtwo} &
    $<0.001$%
}
}
\end{table}



\section[Model Calculation Flow and Execution]{\huge Model Calculation Flow and Execution}\label{sec:calc}
% !TeX root = ../ALPHA_doc.tex
\subsection{Model Calculation Flow}

Figure~\ref{fig:model_flow} shows the processing flow of the AIM-ALPHA model.  
The model begins with the processing of the original data sources, 
including extraction of necessary information, aggregation of gridded datasets, 
and conversion of data formats.  
Next, the processed data are adjusted through parameter estimation via regression analysis, 
data calibration to ensure equilibrium in the base year, 
and coefficient calibration for model consistency.  

Using these prepared data, the model then performs future scenario calculations.  
For each year, equilibrium solutions are iteratively computed based on 
the base-year data, the results of the previous year, and the scenario assumptions.  
The aggregated annual results provide the final outputs for each scenario.  
Additionally, model results are converted into the submission templates 
of AgMIP (Agricultural Model Intercomparison and Improvement Project) 
and IAMC (Integrated Assessment Modeling Consortium) 
to facilitate data exchange and comparison.

\bigskip

\begin{figure}[H]
    \centering
    \includegraphics[width=\textwidth]{fig7_1}
    \caption{Processing flow of the AIM-ALPHA model}
    \label{fig:model_flow}
\end{figure}

\clearpage

\subsection{Programs within the Model}

This section provides an overview of the major programs used in the AIM-ALPHA model.  
Each component plays a specific role in data processing, parameter calibration, and scenario simulation.

\medskip

% Refined tcolorbox style for a more formal academic tone
\newtcolorbox{programbox}[2][]{
  enhanced,
  sharp corners,
  colback=Accent!3!white,          
  colframe=Accent!70!black,        
  boxrule=0.5pt,                   
  left=7pt, right=7pt, top=5pt, bottom=5pt,
  title=\texttt{#2},
  coltitle=Primary!85!black,       
  fonttitle=\ttfamily\bfseries\small,
  attach boxed title to top left={xshift=0pt,yshift*=-1mm},
  boxed title style={
    colback=Accent!12!white,       
    colframe=Accent!60!black,
    boxrule=0pt,
    sharp corners,
    top=1pt, bottom=1pt,
  },
  before skip=8pt, after skip=8pt,
  #1
}


\begin{programbox}{0\_csv\_to\_gdx.R}
Converts datasets in \texttt{csv} or \texttt{xlsx} format into \texttt{gdx} format.  
This script processes raw datasets such as FAOSTAT and MAC Curve sources.
\end{programbox}

\begin{programbox}{0\_grid\_data.R}
Aggregates gridded datasets by country and production unit, and exports the results in \texttt{gdx} format.  
It processes geospatial data from sources such as GAEZ, Livestock System, and LUH2.
\end{programbox}

\begin{programbox}{1\_data\_import.gms}
Integrates and reformats the \texttt{gdx}-formatted datasets into structures compatible with the model’s computation framework.  
It performs country code conversions, consolidates data into AIM-ALPHA’s food and land-use classifications,  
and includes submodules that output historical data in AgMIP and IAMC template formats.
\end{programbox}

\begin{programbox}{1-5\_regression.gms}
Performs parameter regression analyses based on the original datasets to estimate elasticities and adjustment coefficients.
\end{programbox}

\begin{programbox}{2\_basetear.gms}
Adjusts baseline datasets to ensure equilibrium consistency in the base year.
\end{programbox}

\begin{programbox}{3\_equilibrium.gms}
Solves for annual equilibrium states for future years, using base-year data and previous results as inputs.
\end{programbox}

\begin{programbox}{4\_combine\_gdx.gms}
Aggregates annual simulation outputs and compiles them into scenario-level result files.
\end{programbox}

\begin{programbox}{5\_scenario\_combine.gms}
Combines outputs from all scenarios to produce the comprehensive simulation dataset.
\end{programbox}

\begin{programbox}{Visualization.R}
Generates graphical outputs and visualization figures from the simulation results.
\end{programbox}


\clearpage

\subsection{Execution Method}

The AIM-ALPHA model can be executed by running the shell script 
\texttt{AIMALPHA/shell/execution.sh} on either Cygwin or Ubuntu environments. \\
Model configurations are defined in \texttt{AIMALPHA/shell/model\_setting.sh}.  
As summarized in Table~\ref{tab:model_setting}, the configurable items include 
the selection of executable programs, scenario settings, and computational environment settings.

\subsubsection*{Basic execution}

To run the model, move to the \texttt{AIMALPHA/shell} directory and execute:

\begin{lstlisting}[language=bash]
cd AIMALPHA/shell
bash execution.sh
\end{lstlisting}

When \texttt{execution.sh} is called without any arguments, 
the default configuration file \texttt{model\_setting.sh} is loaded automatically 
and the model is executed using those settings.

\subsubsection*{Using a custom setting file}

It is also possible to specify an alternative setting file as an argument.  
In this case, the configuration defined in the specified file is used instead of 
\texttt{model\_setting.sh}:

\begin{lstlisting}[language=bash]
cd AIMALPHA/shell
bash execution.sh mysetting
\end{lstlisting}

Here, \texttt{mysetting.sh} denotes a user-defined setting file 
(e.g.\ a copy of \texttt{model\_setting.sh} with modified options).  
Each run of \texttt{bash execution.sh <setting\_file>} loads the corresponding setting file 
and executes the model according to those configurations.


\begin{table}[H]
\centering
\caption{Model configuration parameters in \texttt{model\_setting.sh}}
\label{tab:model_setting}
\scriptsize
\begin{xltabular}{\textwidth}{@{} l X l @{}}
\toprule
\textbf{Setting item} & \textbf{Description} & \textbf{Default} \\
\midrule
\multicolumn{3}{@{}l}{\textbf{Model switches}} \\[2pt]
\midrule
\texttt{data\_download} & Executes a script to download data from submodules when the repository is first cloned.  
Once downloaded, this can be turned off. & on \\
\texttt{data\_import} & Executes \texttt{1\_data\_import.gms} and \texttt{1-5\_regression.gms}. & on \\
\texttt{baseyear\_run} & Executes \texttt{2\_baseyear.gms}. & on \\
\texttt{model\_run} & Executes \texttt{3\_equilibirium.gms}. & on \\
\texttt{combine\_gdx} & Executes \texttt{4\_combine\_gdx.gms}. & on \\
\texttt{scenario\_combine} & Executes \texttt{5\_scenario\_combine.gms}. & on \\
\texttt{Visualization} & Executes \texttt{Visualization.R}. & on \\
\texttt{data\_update} & Executes \texttt{0\_csv\_to\_gdx.R}.  
If no data update is needed, this may be turned off. & off \\
\texttt{grid\_data} & Executes \texttt{0\_grid\_data.R}.  
If no data update is needed, this may be turned off.  
If executed, the folder \texttt{L/ModelData/AIMALPHA/grid} must be copied to \texttt{AIMALPHA/data}. & off \\
\texttt{Scenario\_compare} & Enables comparison across scenarios in plots. & off \\
\texttt{Model\_compare} & Enables comparison across models in plots. & off \\
\texttt{historical\_check} & Compares simulated and historical data in plots. & off \\
\texttt{aggregate} & Performs aggregation of results into 17-region, 10-region, and 5-region groupings. & on \\
\texttt{AGMIP} & Outputs results in AGMIP and IAMC\_Template formats. & on \\
\midrule
\multicolumn{3}{@{}l}{\textbf{Scenario settings}} \\[2pt]
\midrule
\texttt{socioeconomic\_setting} & Specifies the socioeconomic scenario.  
A file with the scenario name should exist in \texttt{AIMALPHA/setting/socioeconomic}. & SSP2\_BaU \\
\texttt{climate\_setting} & Specifies the climate scenario.  
A file with the scenario name should exist in \texttt{AIMALPHA/setting/climate}. & NoCC \\
\texttt{runlist} & Allows individual scenario specification without using the socioeconomic–climate matrix.  
Scenario names are defined as combined strings (e.g., \texttt{SSP2\_BaU-NoCC}).  
If this item is specified, the two settings above are ignored. & -- \\
\texttt{start\_year} & Starting year of simulation. & 2015 \\
\texttt{target\_year} & Ending year of simulation. & 2100 \\
\midrule
\multicolumn{3}{@{}l}{\textbf{Execution environment}} \\[2pt]
\midrule
\texttt{run\_paralell} & Runs multiple scenarios in parallel. & on \\[2pt]
\texttt{iteration\_number} & 
Number of iterations per year.  
This setting is applied only when \texttt{run\_paralell} is \texttt{off}.  
Larger values improve convergence but increase computation time. & 0 \\[2pt]
\texttt{MultiSol} & 
Used only when \texttt{run\_paralell} is \texttt{on}.  
An integer between 1 and 5.  
Specifies the number of internally divided iteration settings to be solved in parallel 
(e.g., if set to 3, iterations such as 1, 3, and 5 are executed concurrently).  
This helps the model converge more efficiently.  
Note that the specified number corresponds to the number of threads required per scenario  
(e.g., 3 scenarios with \texttt{MultiSol}=3 require 9 threads in total). & 3 \\[2pt]
\texttt{NCPU} & Maximum number of CPUs used for parallel computation. & 8 \\[2pt]
\texttt{mem\_check} & Performs memory usage checks (Windows only). & off \\[2pt]
\texttt{mem\_threshold} & 
Memory usage threshold (\%) at which parallel computation pauses 
when \texttt{mem\_check} is enabled. & 80 \\[2pt]
\texttt{YEAR\_TIMEOUT} & 
Maximum computation time allowed per simulation year (in seconds).  
If this time limit is exceeded, the scenario run is aborted. & 1800 \\ 
\bottomrule
\end{xltabular}
\end{table}

\clearpage

\subsection{Execution Environment Requirements}

The model can be executed under either Cygwin or Ubuntu with the following dependencies installed:

\begin{itemize}[leftmargin=2.5em]
  \item \textbf{Cygwin} (required packages: \texttt{zip}, \texttt{unzip})
  \item \textbf{Ubuntu}
  \item \textbf{R} (version $\geq$ 4.3.2; required libraries: 
    \texttt{"tidyverse"}, \texttt{"gdxrrw"}, \texttt{"openxlsx"}, \texttt{"readxl"},
    \texttt{"ggplot2"}, \texttt{"ggpmisc"}, \texttt{"RColorBrewer"},
    \texttt{"purrr"}, \texttt{"furrr"}, \texttt{"scales"},
    \texttt{"patchwork"}, \texttt{"sf"}, \texttt{"raster"}, \texttt{"ncdf4"},
    \texttt{"exactextractr"}, \texttt{"rmapshaper"}, \\
    \texttt{"rnaturalearth"},\texttt{"rnaturalearthdata"}, \texttt{"grDevices"})
  \item \textbf{GAMS} (version $\geq$ 38, with the \texttt{PATH} solver available)
  \item \textbf{GitHub}
\end{itemize}


\section{Update History}
% !TeX root = ../ALPHA_doc.tex
\begin{changelogentry}{2025-11-02}{R. Totake}{1.0}
Initial release of AIM-ALPHA documentation.
\end{changelogentry}


\printbibliography



% -------------------------------------------------------------------------------------------

% \section{Overview}
% \begin{refsection}
% This document specifies the mathematics and implementation details of a numerical model. Use \Cref{sec:gov,sec:num} for governing equations and numerics; \cref{sec:val} for verification/validation.
% This document specifies the mathematics and implementation details of a numerical model. Use \Cref{sec:gov,sec:num} for governing equations and numerics; \cref{sec:val} for verification/validation.
% According to \cite{smith2020model}, the model performs well under uncertainty.
% ... as shown in previous work \parencite{hasegawa2022climate}.


% \printbibliography[heading=subbibliography]
% \end{refsection}

% \section{Notation}
% Vectors $\vect{x}$ (bold lowercase), matrices $\mat{A}$ (bold uppercase), domain $\Omega\subset\R^d$, time $t\in\R_{\ge0}$. Norms $\norm{\cdot}$ and absolute values $\abs{\cdot}$ use auto-sized delimiters.

% \section{Governing Equations and BCs}\label{sec:gov}
% Example diffusion model (replace as needed):
% \begin{equation}\label{eq:pde}
% \frac{\partial u}{\partial t} - \nabla\!\cdot\!\bigl(D\nabla u\bigr) = f \quad \text{in } \Omega\times(0,T].
% \end{equation}
% Boundary and initial conditions:
% \begin{align}\label{eq:bc}
% \begin{cases}
%  u = g_D & \text{on } \Gamma_D, \\
%  -D\,\nabla u\cdot\vect{n} = g_N & \text{on } \Gamma_N,
% \end{cases}\qquad
% u: u(\vect{x},0)=u_0(\vect{x}).
% \end{align}
% Weak form for test function $v$:
% \begin{align}
% \int_\Omega \frac{\partial u}{\partial t}\,v\,\dd\vect{x}
% +\int_\Omega D\,\nabla u\cdot\nabla v\,\dd\vect{x}
% = \int_\Omega f v\,\dd\vect{x} + \int_{\Gamma_N} g_N v\,\dd s.
% \end{align}

% \section{Numerical Methods}\label{sec:num}
% \subsection{Time Discretization (\texorpdfstring{$\theta$}{theta}-scheme)}
% \begin{equation}\label{eq:theta}
% \frac{u^{n+1}-u^{n}}{\Delta t} - \nabla\!\cdot\!\Bigl(D\,[\theta\nabla u^{n+1} + (1-\theta)\nabla u^{n}]\Bigr) = f^{n+\theta},\qquad \theta\in[0,1].
% \end{equation}
% \subsection{Linear System}
% After spatial discretization (FEM/FDM/FVM):
% \begin{equation}
% \mat{A}(\theta)\,\vect{u}^{n+1} = \vect{b}(\vect{u}^{n}).
% \end{equation}
% \begin{algorithm}[H]
% \caption{Time Marching (\texorpdfstring{$\theta$}{theta}-scheme)}\label{alg:time}
% \begin{algorithmic}[1]
% \Require $\vect{u}^0$, $\Delta t$, steps $N$ \Ensure $\vect{u}^N$
% \For{$n=0,1,\dots,N-1$}
%   \State assemble $\vect{b}(\vect{u}^n)$ and $\mat{A}(\theta)$
%   \State solve $\mat{A}\,\vect{u}^{n+1}=\vect{b}$ with preconditioning
% \EndFor
% \end{algorithmic}
% \end{algorithm}

% \section{Parameters and Units}
% \begin{table}[H]
%   \centering
%   \caption{Key parameters}
%   \sisetup{table-number-alignment = center}
%   \begin{tabular}{@{} l S[table-format=1.2] l l @{} }
%     \toprule
%     Symbol & {Value} & Unit & Description \\
%     \midrule
%     $D$ & 1.25 & \si{\meter\squared\per\second} & Diffusion coefficient \\
%     $\Delta t$ & 0.01 & \si{\second} & Time step \\
%     $T$ & 10 & \si{\second} & Final time \\
%     \bottomrule
%   \end{tabular}
% \end{table}

% \section{Results (Figure Example)}
% \begin{figure}[H]
%   \centering
%   \includegraphics[width=.75\linewidth]{fig1}
%   \caption{Snapshot of numerical result (example).}
%   \label{fig:snapshot}
% \end{figure}

% \section{Verification and Validation}\label{sec:val}
% If an analytic solution $u_\text{ana}$ is known,
% \begin{equation}\label{eq:error}
% E_{L^2}=\Bigl(\int_\Omega \abs{u_\text{num}-u_\text{ana}}^2\,\dd\vect{x}\Bigr)^{1/2},\qquad
% E_\infty=\max_{\vect{x}\in\Omega}\abs{u_\text{num}-u_\text{ana}}.
% \end{equation}
% Report observed order by varying mesh size $h$ and time step $\Delta t$.

% \appendix
% \section{Boxed Statements and Extras}
% \begin{theobox}{Maximum Principle}{th:max}
% Under suitable conditions on $D$ and $f$, the solution $u$ satisfies ...
% \end{theobox}

% \begin{lemabox}{Stability}{lem:stab}
% The scheme \eqref{eq:theta} is unconditionally stable for $\theta\ge 1/2$.
% \end{lemabox}

% \begin{defbox}{Weighted Norm}{def:wn}
% For $w>0$, define $\norm{v}_{w}^2=\int_\Omega w\,v^2\,\dd\vect{x}$.
% \end{defbox}

% % \begin{table}[h]
% %   \centering
% %   \caption{Parameters from CSV}
% %   \csvautotabular{data/params.csv}
% % \end{table}

% \paragraph{Listing example}
% \begin{lstlisting}[language=C++]
% for (int n = 0; n < N; ++n) {
%   assemble_rhs(u);
%   assemble_matrix(theta);
%   solve(A, u);
% }
% \end{lstlisting}

% \bigskip
% \noindent\textbf{Cross-ref:} Equations~\cref{eq:pde,eq:theta}, Table, Figure~\cref{fig:snapshot}, Algorithm~\cref{alg:time}.

\end{document}
